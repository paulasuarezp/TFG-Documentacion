\subsection{Análisis de sistemas similares}
En este apartado, se aborda un análisis detallado de las plataformas y sistemas que actualmente exhiben un comportamiento similar al del sistema que se pretende desarrollar para 'BidMon Universe'. 

La realización de este análisis es crucial, ya que proporciona una base sólida para la toma de decisiones estratégicas. Al examinar detenidamente tanto las ventajas como las desventajas de los sistemas existentes, se identifican estrategias exitosas y se reconocen posibles fallos o áreas susceptibles de mejora. Este enfoque no solo informa las decisiones de diseño y tecnología, sino que también las fundamenta en un conocimiento práctico y bien contextualizado. Así, se garantiza que el desarrollo de 'BidMon Universe' esté en consonancia con las tendencias actuales y adopte las soluciones más efectivas y adecuadas para su propósito.

\subsubsection{EA Sports FC}
EA Sports FC constituye una destacada franquicia de videojuegos de fútbol, los cuales se encuentran disponibles en diversas plataformas. Dentro de estos juegos, los usuarios tienen la oportunidad de adquirir sobres que contienen cartas de futbolistas y otros elementos complementarios. Además, cuentan con acceso a un mercado virtual donde pueden llevar a cabo transferencias de jugadores mediante un sistema de subastas.

El sistema de subastas puede variar según la plataforma del juego, sin embargo, todos comparten una serie de características que sera continuación.

\subsubsubsection{Ventajas de EA Sports FC}
En el contexto de EA Sports FC, destacan las siguientes ventajas:
\begin{itemize}
    \item Los usuarios pueden marcar una carta para efectuar un seguimiento y recibir notificaciones en caso de una disminución en su precio.
    \item Se proporcionan estadísticas detalladas de cada carta, incluyendo su evolución de precio en las últimas semanas, así como los valores más bajos y más altos a los que se está vendiendo actualmente.
    \item Existe la opción de vender una carta directamente, evitando la necesidad de utilizar el sistema de subastas. 
    \item Cuenta con un buscador de ofertas que incorpora varios filtros.
    \item Los usuarios tienen la posibilidad de editar o cancelar sus pujas en curso.
    \item El sistema de subastas implementado opera bajo un mecanismo de puja ciega.
\end{itemize}

\subsubsubsection{Desventajas de EA Sports FC}
Sin embargo, es importante mencionar algunas desventajas de EA Sports FC, que incluyen:
\begin{itemize}
    \item Un número limitado de órdenes de canje simultáneas como, por ejemplo, el máximo de 25 permitido en EA Sports FC Mobile.
    \item Las pujas solo se pueden realizar por un valor igual o superior al establecido por el juego. 
\end{itemize}

\subsubsubsection{Ventajas del nuevo sistema respecto EA Sports FC}
El sistema en desarrollo presenta varias características que mejoran la experiencia del usuario en comparación con EA Sports FC.

\begin{itemize}
    \item El acceso a la aplicación es completamente gratuito.
    \item Ofrece una mayor personalización en el proceso de puja, brindando a los usuarios una experiencia más flexible y adaptada a sus preferencias.
    \item El usuario tiene la posibilidad de acceder a un histórico exhaustivo en el que se detallan todas las transacciones realizadas.
\end{itemize}

\subsubsection{LaLiga Fantasy}
\coloredUnderline{\href{https://laligafantasy.relevo.com/}{LaLiga Fantasy}} un juego basado en la liga de fútbol española, conocida como LaLiga. En este juego, los usuarios tienen la capacidad de crear equipos que se componen de jugadores cuyo rendimiento se correlaciona con el desempeño real en los partidos de LaLiga. Además, existe la posibilidad de competir por premios reales en determinadas instancias del juego.
El juego dispone de un mercado en el que los usuarios pueden vender o adquirir jugadores a través de un sistema de subastas, por lo que estamos ante un escenario similar al anterior.


\subsubsubsection{Ventajas de LaLiga Fantasy}
Dentro del marco de LaLiga Fantasy, se destacan las siguientes características:
\begin{itemize}
    \item El juego renueva constantemente el mercado de transferibles.
    \item El juego cuenta con la capacidad de generar ofertas automáticas por los futbolistas que se encuentran en venta. Estas ofertas se establecen a partir de un valor aleatorio que oscila entre el valor de mercado del jugador, con un margen del 10\% tanto por encima como por debajo de dicho valor.
    \item Recientemente se ha incorporado el mercado de ``clausulazos'', donde los usuarios pueden adquirir un jugador pagando su cláusula, que será más elevada que el valor de mercado, sin tener que depender del propietario del jugador.
    \item Se pueden realizar ofertas por jugadores a otros usuarios.
\end{itemize}

\subsubsubsection{Desventajas de LaLiga Fantasy}
Se pueden identificar las siguientes desventajas:
\begin{itemize}
    \item La limitación a un máximo de 24 jugadores en la plantilla, lo que puede ocasionar la pérdida de oportunidades en subastas.
    \item La plataforma brinda escasa flexibilidad en lo que respecta a la personalización al momento de poner un jugador en venta.
\end{itemize}

\subsubsubsection{Ventajas del nuevo sistema respecto LaLiga Fantasy}
\begin{itemize}
    \item LaLiga Fantasy ofrece una suscripción de 0,99€/mes para poder jugar sin anuncios mientras que el sistema que se desarrollará carece de anuncios.
    \item Se proporciona un historial de transacciones completo para que los usuarios puedan rastrear las compras y ventas.
    \item Se implementa un sistema de subastas en tiempo real que, además, brinda una mayor personalización al usuario en aspectos como la duración de la subasta y los valores de venta, entre otros.
    \item Además de acceder al mercado, los usuarios tienen la opción de adquirir sobres de cartas, lo que añade un elemento de emoción a la aplicación.
\end{itemize}

\subsubsection{eBay}
\coloredUnderline{\href{https://www.ebay.es/}{eBay}} es una plataforma de comercio online que permite a los usuarios comprar y vender productos a través de subastas o ventas directas. 
\subsubsubsection{Ventajas de eBay}
Dentro del marco de LaLiga Fantasy, se destacan las siguientes características:
\begin{itemize}
    \item Para cada producto subastado, la plataforma ofrece la posibilidad de visualizar información detallada que incluye el número de pujas, la cantidad de pujadores, las retractaciones, el tiempo restante en la subasta, y proporciona un historial completo de las pujas realizadas en ese producto. Este historial incluye datos relevantes sobre las pujas, como su valor y la fecha en la que se efectuaron. Además, se brinda acceso a información pertinente sobre los pujadores involucrados.
    \item eBay proporciona a los usuarios la capacidad de configurar pujas automáticas. En este proceso, el comprador define el precio máximo que está dispuesto a pagar por el producto, y la plataforma aumenta automáticamente la oferta en su nombre, siempre que sea necesario, para mantener al comprador como el principal postor hasta alcanzar el límite previamente establecido.
    \item Los usuarios tienen la posibilidad de examinar el perfil del vendedor, explorar otros productos que este tenga en venta y establecer contacto directo con él.
    \item Como vendedor, la plataforma te brinda la capacidad de definir el valor inicial de la puja, decidir si deseas recibir ofertas, establecer la fecha de inicio de la subasta, determinar su duración y habilitar la opción de compra directa a un precio específico de tu elección.
\end{itemize}

\subsubsubsection{Desventajas de eBay}
A continuación, es posible señalar las siguientes desventajas:
\begin{itemize}
    \item La interfaz de usuario muestra una gran cantidad de información, lo que puede resultar confuso para alguien que no está acostumbrado a ella.
    \item Cuando se va a realizar una puja, no aparece una ventana emergente de confirmación de puja o similar por lo que es fácil introducir una puja errónea y, posteriormente, puede resultar complicado retirarla.
    \item Las subastas tienen incrementos de puja predefinidos, lo que significa que no puedes especificar un valor concreto por el que desees pujar.
\end{itemize}

\subsubsubsection{Ventajas del nuevo sistema respecto eBay}
\begin{itemize}
    \item El sistema en desarrollo presentará una interfaz simple e intuitiva, que proporcionará todos los datos esenciales para efectuar una puja con confianza, sin abrumar al usuario con una sobrecarga de información.
    \item El sistema que se desarrollará proporcionará al usuario información sobre las condiciones para retirar una puja antes de su confirmación.
\end{itemize}


% datos de facturación de laliga y juegos fantasy
% https://www.palco23.com/entorno/los-fantasy-se-preparan-para-su-boom-el-negocio-rebasara-416-millones-en-2024
% ebay https://www.ebayinc.com/company/

\subsection{Valoración de alternativas de solución}
En el presente apartado, se realiza un examen detallado de las diversas alternativas tecnológicas y arquitectónicas disponibles que satisfacen los requisitos previamente establecidos para el proyecto. 

Este análisis implica una evaluación rigurosa de las ventajas y desventajas asociadas a cada opción. La finalidad es identificar la solución más apropiada que no solo cumpla con los requisitos funcionales y no funcionales del proyecto, sino que también se alinee óptimamente con las restricciones y objetivos globales del mismo.

\subsubsection{Valoración de alternativas para la arquitectura}
A continuación, se presenta una comparación de varias alternativas arquitectónicas para el desarrollo del sistema.
\subsubsubsection{Arquitectura Monolítica}
La arquitectura monolítica es un enfoque de desarrollo de software en el que una aplicación se construye como una sola unidad. En este caso, todos los componentes del sistema se diseñan y se implementan como un único bloque, que se ejecuta como un único proceso.
Este enfoque en un proyecto pequeño puede ser beneficioso, ya que simplifica el proceso de desarrollo y sobretodo de despliegue. Sin embargo, a medida que el proyecto crece, la arquitectura monolítica se vuelve cada vez más compleja y difícil de mantener. Además, la escalabilidad de la aplicación se ve limitada por la necesidad de escalar el sistema en su conjunto, en lugar de poder escalar componentes individuales de forma independiente.

\subsubsubsection{Arquitectura de Microservicios}
La arquitectura de microservicios es un enfoque de desarrollo de software en el que una aplicación se construye como un conjunto de servicios pequeños, independientes y altamente escalables. Cada servicio se ejecuta como un proceso separado y se comunica con otros servicios mediante mecanismos ligeros, como una API REST.
Este enfoque permite que los servicios se desarrollen, desplieguen y escalen de forma independiente, lo que facilita la gestión de proyectos complejos. Sin embargo, la arquitectura de microservicios también introduce una mayor complejidad en el desarrollo y la gestión de la aplicación, ya que requiere la implementación de mecanismos de comunicación entre los servicios, así como la gestión de la escalabilidad de cada uno de ellos.

\subsubsubsection{Arquitectura de REST API y WepApp}    
Se caracteriza por una clara división entre cliente y servidor, encapsulados respectivamente en WepApp (frontend) y REST API (backend). Esta separación promueve una organización modular, facilitando el mantenimiento del proyecto al separar de forma clara las distintas responsabilidades. 
Este enfoque es un punto intermedio entre las dos arquitecturas anteriores, ya que permite una mayor flexibilidad en el desarrollo y la gestión de la aplicación, sin introducir una complejidad excesiva. 

\subsubsubsection{Comparativa de alternativas arquitectónicas}
Mediante la siguiente tabla, se presenta una comparación de las alternativas arquitectónicas previamente mencionadas.

\begin{table}[htb]
    \centering
    \caption{Comparación de Arquitecturas de Software}
    \label{tabla:comparacion_arquitecturas}
    \hypertarget{table:comparacion_arquitecturas}{}
    \begin{tabular}{
       >{\columncolor{rowcolor}\raggedright\arraybackslash}p{4cm}
       >{\raggedright\arraybackslash}p{3cm}
       >{\raggedright\arraybackslash}p{3cm}
       >{\raggedright\arraybackslash}p{3cm} }
    \rowcolor{lightgreen}
    \toprule
    \textbf{Criterio} & \textbf{Arquitectura Monolítica} & \textbf{Arquitectura de Microservicios} & \textbf{API REST y WebApp} \\
    \midrule
    Complejidad Inicial & Baja & Alta & Moderada \\
    \midrule
    Escalabilidad & Limitada & Alta & Moderada \\
    \midrule
    Facilidad de Mantenimiento & Alta en proyectos pequeños & Moderada a baja & Moderada si cada módulo tiene una estructura interna clara \\
    \midrule
    Despliegue & Sencillo & Complejo & Moderado \\
    \midrule
    Gestión de Proyectos & Simple en proyectos pequeños & Requiere coordinación compleja & Balanceada \\
    \midrule
    Independencia de Componentes & No & Sí & Parcial \\
    \midrule
    Adaptabilidad a Cambios & Baja & Alta & Moderada \\
    \midrule
    Recomendado para & Proyectos pequeños y simples & Proyectos grandes y escalables & Proyectos con necesidad de separación clara entre frontend y backend \\
    \bottomrule
    \end{tabular}
\end{table}


\subsubsubsection{Decisión final de la arquitectura}
Tras un análisis exhaustivo de las alternativas disponibles, se ha optado por implementar una arquitectura de API REST y WebApp.

Esta decisión permite alcanzar un equilibrio entre la simplicidad inherente a la arquitectura monolítica y la escalabilidad ofrecida por la arquitectura de microservicios. Además, esta elección facilita la gestión del proyecto mediante una separación clara entre el frontend y el backend. Tal distinción posibilita el desarrollo y despliegue independiente de cada componente, contribuyendo a la minimización de riesgos asociados a fallos en cadena. 
Además, cada módulo, operando de manera aislada, mejora significativamente la disponibilidad y confiabilidad del sistema.



\subsubsection{Valoración de alternativas para el Backend}
Es crucial seleccionar una tecnología de backend que no solo soporte eficientemente las operaciones en tiempo real sino que también se integre de manera óptima con el frontend.
A continuación, se presenta un análisis de varias alternativas para el desarrollo del backend teniendo en cuenta estos requisitos.

\subsubsubsection{Java con Spring Boot}
Java es un lenguaje de programación orientado a objetos que destaca por su portabilidad y robustez, siendo ampliamente utilizado en el desarrollo de aplicaciones empresariales. 
La combinación con Spring Boot permite un desarrollo ágil con una amplia gama de herramientas como Spring Data y Spring Security entre otras. 
Entre sus ventajas, además de las herramientas mencionadas, destacan su escalabilidad, su robustez en el manejo de transacciones y su amplia comunidad. 
Sin embargo, Java con Spring Boot presenta una curva de aprendizaje significativa y, aunque Spring Boot puede manejar aplicaciones en tiempo real mediante Spring WebFlux, su rendimiento en este ámbito puede ser inferior comparado con otras tecnologías. Además, la integración con el frontend, podría no ser la más rápida en términos de desarrollo.

\subsubsubsection{Node.js con Express}
Node.js es un entorno de ejecución para JavaScript en el lado del servidor, conocido por su modelo de I/O no bloqueante y su eficiencia en el manejo de múltiples conexiones simultáneas
La integración de Node.js con Express, un framework ligero y flexible, ofrece una solución óptima para desarrollar aplicaciones web dinámicas.
Esta combinación destaca por su capacidad para gestionar operaciones en tiempo real de manera eficiente y su sinergia natural con tecnologías frontend basadas en JavaScript, facilitando un desarrollo cohesivo y ágil entre el backend y el frontend.
Como desventajas, cabe mencionar que Node.js puede no ser la opción más adecuada para tareas que requieren un alto uso de CPU debido a su naturaleza de ejecución de un solo hilo, ya que su rendimiento en este ámbito puede ser inferior al de otras tecnologías.

\subsubsubsection{Python con Django}
Python, un lenguaje de programación de alto nivel, se caracteriza por su versatilidad y amplia gama de bibliotecas. Utilizado ampliamente en aplicaciones empresariales combinado con Django, un framework de alto nivel que se enfoca en el desarrollo rápido y eficiente.
Como ventajas, destacan su desarrollo rápido, sus excelentes capacidades de seguridad y su buena documentación.
Aunque Django puede ser configurado para soportar aplicaciones en tiempo real mediante Django Channels, su rendimiento en estos escenarios puede no ser tan óptimo como el de Node.js. Además, aunque Django asegura una buena integración con el frontend, puede no ser la opción más ágil para aplicaciones que requieren una interacción constante.



\subsubsubsection{Comparativa de alternativas para el Backend}
En la siguiente tabla, se presenta una comparación de las alternativas para el desarrollo del Backend.
\begin{table}[H]
    \centering
    \begin{tabular}{ 
       >{\columncolor{rowcolor}\raggedright\arraybackslash}p{3cm} 
       >{\raggedright\arraybackslash}p{3cm} 
       >{\raggedright\arraybackslash}p{3cm} 
       >{\raggedright\arraybackslash}p{3cm} }
        \rowcolor{lightgreen}
    \toprule
    \textbf{Criterio} & \textbf{Java con Spring Boot} & \textbf{Node.js con Express} & \textbf{Python con Django} \\
    \midrule
    \textbf{Modelo de Programación} & Orientado a objetos, con énfasis en inyección de dependencias. & Basado en eventos y callbacks. & Orientado a componentes/modelos con énfasis en la reutilización de código. \\
    \midrule
    \textbf{Funcionalidad en tiempo real} & Buen manejo de transacciones pero menos óptimo para tiempo real. & Excelente para operaciones en tiempo real gracias a su eficiencia en I/O. & Posible pero requiere más configuración para tiempo real. \\
    \midrule
    \textbf{Integración con Frontend} & Buena, puede requerir esfuerzos adicionales. & Natural y fluida. & Buena, pero puede necesitar configuraciones extra. \\
    \midrule
    \textbf{Escalabilidad} & Alta, pero con escalabilidad vertical. & Alta, con facilidad para escalar horizontalmente. & Moderada, con algunas limitaciones en escalabilidad. \\
    \midrule
    \textbf{Seguridad} & Fuertes capacidades de seguridad. & Requiere implementaciones adicionales para seguridad. & Seguridad integrada y robusta. \\
    \bottomrule
    \end{tabular}
    \caption{Comparación de Tecnologías para el Backend}
    \label{tabla:comparacion_backend}
    \hypertarget{table:comparacion_backend}{}
    \end{table}

    
\subsubsection{Decisión final del Backend}
Considerando las necesidades del sistema a desarrollar, para soportar subastas en tiempo real y una integración fluida con el frontend la opción más adecuada es Node.js con Express.


\subsubsection{Valoración de alternativas para el Frontend}
Por último, se presenta un análisis de alternativas para el desarrollo del frontend.

\subsubsubsection{React}
React es una biblioteca de JavaScript desarrollada y mantenida por Facebook, centrada en la construcción de interfaces de usuario a través de componentes reutilizables. Se caracteriza por su virtual DOM y su enfoque declarativo.

Entre sus ventajas, destacan su amplia comunidad, su rendimiento optimizado con Virtual DOM y su flexibilidad en la elección de estilos y componentes. 
Como desventajas cabe mencionar las actualizaciones frecuentes que implican mantenerse al día con los cambios y que necesita integración con otras herramientas para ser una solución completa.


\subsubsubsection{Angular}
Angular es un framework de desarrollo web mantenido por Google, conocido por su enfoque en aplicaciones de página única (SPA). Utiliza TypeScript como lenguaje principal y proporciona un entorno robusto y completo para el desarrollo.

Entre sus ventajas, destaca su ecosistema completo, promueve un estilo de desarrollo coherente y mantenible y, además, cuenta con una comunidad activa y un amplio soporte de Google, lo que garantiza actualizaciones y soporte continuo.
Sin embargo, Angular puede ser excesivo para proyectos pequeños, ya que su curva de aprendizaje es pronunciada y su configuración inicial puede ser compleja.

\subsubsubsection{Vue.js}
Vue es un framework progresivo para la construcción de interfaces de usuario, creado por Evan You. Se destaca por su facilidad de adopción, su sistema reactividad y su enfoque en la simplicidad.

Entre sus ventajas, destacan su sintaxis sencilla, su buena documentación y su facilidad de integración en proyectos existentes.
Aunque su uso va en aumento, su comunidad es más pequeña en comparación con React o Angular, lo que se traduce en menos librerías y recursos disponibles.

\subsubsubsection{Comparativa de alternativas para el Frontend}
A continuación, se presenta una comparación de las alternativas para el desarrollo del Frontend.

\begin{table}[H]
    \centering
    \begin{tabular}{ 
       >{\columncolor{rowcolor}\raggedright\arraybackslash}p{2.5cm} 
       >{\raggedright\arraybackslash}p{3.5cm} 
       >{\raggedright\arraybackslash}p{3.5cm} 
       >{\raggedright\arraybackslash}p{3.5cm} }
        \rowcolor{lightgreen}
    \toprule
    
    \textbf{Criterio} & \textbf{React} & \textbf{Angular} & \textbf{Vue.js} \\
    \midrule
    \textbf{Velocidad y Rendimiento} & Alto con Virtual DOM. Optimizado para cambios dinámicos de UI. & Buen rendimiento, pero puede ser más lento en proyectos grandes debido a su complejidad. & Rendimiento similar a React, con optimizaciones en la actualización de la UI. \\
    \midrule
    \textbf{Mantener el Estado} & Requiere bibliotecas adicionales como Redux para manejo complejo del estado. & Gestión de estado integrada, adecuada para aplicaciones complejas. & Sistema reactividad sencillo, para manejo de estado complejo requiere la biblioteca Vuex. \\
    \midrule
    \textbf{Comunidad} & Muy grande y activa, con un ecosistema extenso. & Amplia y soportada por Google, con muchas empresas adoptándolo. & Creciente y entusiasta, con un enfoque en la facilidad de uso. \\
    \midrule
    \textbf{Curva de Aprendizaje} & Moderada; JSX y el ecosistema pueden requerir tiempo de aprendizaje. & Elevada; TypeScript y su arquitectura completa requieren más tiempo para aprender. & Baja; Vue es considerado fácil de aprender, especialmente para principiantes. \\
    \midrule
    \textbf{Escalabilidad} & Muy escalable con un enfoque modular y reutilizable. & Diseñado para aplicaciones empresariales escalables y complejas. & Escalable, pero más adecuado para proyectos de tamaño mediano. \\
    \midrule
    \textbf{Ecosistemas y Módulos} & Rico ecosistema con una gran cantidad de módulos y herramientas. & Ecosistema completo con soluciones integradas para muchas necesidades. & Ecosistema más pequeño aunque en crecimiento, con módulos y librerías en aumento. \\
    \bottomrule
    \end{tabular}
    \caption{Análisis Comparativo de Frameworks y Bibliotecas de Frontend}
    \label{tabla:comparacion_frontend}
    \hypertarget{table:comparacion_frontend}{}
\end{table}

\subsubsubsection{Decisión final del Frontend}
Tras una evaluación detallada de las distintas opciones disponibles y en función de los requisitos específicos del sistema a desarrollar, se ha determinado que React es la tecnología más adecuada para el frontend.

Esta decisión se basa, principalmente, en la gran comunidad de React y en la rica colección de recursos disponles.