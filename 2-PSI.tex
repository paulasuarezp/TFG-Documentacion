
\section{INICIO DEL PLAN DE SISTEMAS DE INFORMACIÓN}
\subsection{Análisis de la Necesidad del Plan de Sistemas de Información}
Con la tecnología, la forma en la que nos relacionamos con nuestros pasatiempos ha cambiado. La popularidad de plataformas como EA Sports FC y su modo Ultimate Team\cite{artsEASPORTSFC2023}, 
deja en evidencia que el mercado de coleccionismo digital está en pleno crecimiento \cite{sanmartin000MillonesDolares2021}. 
Sin embargo, aún existe un nicho de mercado específico por explorar: una plataforma dedicada exclusivamente al coleccionismo de cartas digitales.

BidMon Universe trata de dar respuesta a esta necesidad, ofreciendo a los coleccionistas digitales una forma segura e innovadora de comprar y vender activos digitales mediante un sistema de pujas ciego.

El Plan de Sistemas de Información para este proyecto se aborda desde una perspectiva que considera tanto la arquitectura técnica como la experiencia del usuario final. 
De esta manera se establece un marco de trabajo que garantiza que, además de cumplir con las expectativas del mercado actual, el proyecto proporciona características y experiencias novedosas 
que enriquecen la dinámica del coleccionismo digital.

\subsection{Identificación del Alcance del Plan de Sistemas de Información} \hypertarget{sec:2_identificacion_alcance_PSI}{}
La aplicación web a desarrollar deberá de proporcionar al usuario final las siguientes funcionalidades de alto nivel:
\begin{itemize}
    \item Todo usuario final puede acceder a la aplicación web y consultar información genérica sobre la misma.
    \begin{itemize}
        \item Información sobre el servicio que se ofrece.
        \item Información sobre el equipo de desarrollo.
    \end{itemize}
    \item Cualquier usuario, mayor de 18 años, podrá crear una cuenta de usuario en la aplicación.
    \item Cualquier usuario registrado en la aplicación puede iniciar sesión en esta.
    \item Los usuarios autenticados tendrán una colección de cartas digitales que podrán consultar.
    \item Los usuarios autenticados pueden recargar el saldo de su cuenta.
    \item Los usuarios autenticados tendrán acceso a un catálogo de sobres de cartas que podrán adquirir si cuentan con saldo suficiente.
    \item Los usuarios autenticados podrán realizar subastas y pujas ciegas sobre las cartas.
    \item Los usuarios autenticados podrán consultar el histórico de transacciones realizadas.
    \item Los usuarios autenticados podrán modificar su perfil.
    \item Los usuarios autenticados podrán consultar el perfil de las personas que están subastando cartas. Se mostrará la siguiente información:
    \begin{itemize}
        \item Cartas destacadas del usuario. Por ejemplo, si este posee una carta rara podrá seleccionarla como destacada para mostrarla en su perfil.
        \item Número de pujas en las que ha participado.
        \item Número de subastas que ha realizado.
        \item Subastas activas.
    \end{itemize}
    \item Un usuario con rol de administrador podrá acceder al histórico de subastas y cancelar las que considere fraudulentas, el sistema notificará al usuario que la ha iniciado.
\end{itemize}

En base a estas funcionalidades se extraen los siguientes \textbf{objetivos estratégicos} a lograr para alcanzar el éxito del proyecto:

\begin{itemize}
    \item Implementar un sistema de autenticación de usuarios.
    \item Implementar un sistema de venta de cartas.
    \item Implementar un sistema de subastas ciegas que permita a los usuarios comprar y vender cartas.
    \item Implememntar un sistema de gestión de subastas por parte de un usuario autenticado.
    \item Implememntar un sistema de gestión de subastas por parte de un usuario administrador.
    \item Implementar un sistema de notificaciones que informe a los usuarios sobre el estado de las subastas en las que participan.
    \item Interfaz intuitiva que permita al usuario realizar consultas sobre el estado de las pujas y subastas, además de visualizar la colección de cartas adquirida.
\end{itemize}

\subsection{Determinación de Responsables}
A continuación, se detallan las personas que participan en el proyecto junto con sus funciones.
\begin{itemize}
    \item El tutor del trabajo se encargará de la supervisión del proyecto.
    \item El alumno se encargará del desarrollo del proyecto y documentación de este. 
\end{itemize}

\newpage
\section{DEFINICIÓN Y ORGANIZACIÓN DEL PLAN DE SISTEMAS DE INFORMACIÓN}
 

\subsection{Especificación del Ámbito y Alcance} 
El proyecto BidMon Universe tiene como objetivo principal el desarrollo de una plataforma que permitirá a los usuarios coleccionar activos digitales, específicamente cartas coleccionables. 
Para adquirir estas cartas, los usuarios podrán realizar compras directas dentro de la aplicación o participar en subastas.

La plataforma incorporará una moneda virtual que facilitará las transacciones de compra y subasta de cartas. Los usuarios podrán incrementar su saldo utilizando una pasarela de pago integrada. 
Cada usuario autenticado tendrá la capacidad de comprar cartas y, si lo desea, ofrecerlas en subastas ciegas, donde no se revelarán las ofertas de otros participantes hasta el cierre de la misma. 
Además, cada usuario mantendrá una colección privada de cartas, con la opción de destacar ciertas cartas en su perfil.

En cuanto a la mecánica de subastas, el usuario que ofrezca la puja más alta al finalizar el período establecido ganará la carta, y el monto ofrecido se deducirá de su saldo virtual. 
El vendedor podrá establecer condiciones específicas para la subasta, como la duración y el precio inicial, y tendrá la opción de cancelar la subasta en cualquier momento.

Para mantener informados a los usuarios, la plataforma contará con un sistema de notificaciones que proporcionará actualizaciones sobre las subastas en curso, tanto para las iniciadas 
como para aquellas en las que el usuario participe. Asimismo, se mantendrá un registro detallado de todas las transacciones, incluidas compras, subastas y pujas, asegurando la trazabilidad y 
transparencia de todas las operaciones. Las propias cartas también tendrán un historial de transacciones asociado, que se mostrará en la interfaz de usuario.

Los usuarios tendrán acceso a un panel de control donde podrán revisar su historial de transacciones y modificar aspectos de su perfil. 
Por otro lado, los administradores tendrán herramientas específicas para supervisar y gestionar el historial de subastas, incluida la capacidad de cancelar aquellas que consideren fraudulentas.

Se establecen limitaciones significativas para la primera versión de la plataforma debido a restricciones de tiempo.
No se incluirá un sistema de mensajería interna, ni se permitirá la formación de conexiones sociales directas entre usuarios. Además, no se incluirán funcionalidades de jugabilidad con las cartas, 
ni se podrán exportar las colecciones de cartas a formatos externos ni convertir el saldo virtual en dinero real.

Finalmente, los datos iniciales de las cartas se cargarán una única vez y no se permitirá la adición de nuevas cartas a través de la interfaz de usuario. De igual forma, no se incluirá la opción de
añadir nuevos sobres de cartas a la tienda virtual desde la interfaz de usuario. Estas modificaciones se podrán realizar directamente en la base de datos.


\subsection{Organización del Plan de Sistemas de Información }

