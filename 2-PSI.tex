
\section{INICIO DEL PLAN DE SISTEMAS DE INFORMACIÓN}
En esta sección se determina la necesidad del trabajo de fin de grado a constuituir y llevar a cabo su arranque formal. 
Para este fin, se analizan las expectativas del cliente, se identifican los objetivos estratégicos del proyecto y se determinan los responsables del mismo.
 
\subsection{Identificación del Alcance del Plan de Sistemas de Información }
El auge de los juegos en línea y el coleccionismo digital ha transformado la manera en que interactuamos con nuestros hobbies y pasiones. Con la popularidad de plataformas como EA Sports FC y su modo Ultimate Team, queda claro que el mercado de coleccionismo digital está en pleno crecimiento. Sin embargo, aún existe un nicho específico por explorar: una plataforma dedicada exclusivamente al coleccionismo de cartas digitales, que ofrezca una experiencia de usuario única centrada en la comunidad, la seguridad y la innovación.

Frente a este panorama, se propone el desarrollo de una plataforma de coleccionismo de cartas digitales que no solo permita a los usuarios comprar sobres de cartas, sino también venderlas en un mercado digital mediante un sistema de pujas ciego. Este enfoque busca mejorar la interacción entre coleccionistas, ofreciendo un sistema más justo y emocionante para la adquisición de cartas raras y buscadas.

La necesidad de un Plan de Sistemas de Información (PSI) para este proyecto es evidente. Debe abordarse desde una perspectiva holística que considere tanto la arquitectura técnica como la experiencia del usuario final. Este PSI permitirá establecer un marco de trabajo sólido para el desarrollo, lanzamiento y mantenimiento de la plataforma, asegurando que esta no solo cumpla con las expectativas del mercado actual, sino que también ofrezca nuevas funcionalidades y experiencias que enriquezcan el coleccionismo digital.

Además, el PSI jugará un papel crucial en la diferenciación de nuestra plataforma frente a competidores establecidos. Al enfocarnos en características únicas como el sistema de pujas ciego y una comunidad digital fuertemente integrada, podremos captar la atención de un segmento de mercado apasionado por el coleccionismo que busca nuevas formas de interactuar con sus colecciones.

En conclusión, el desarrollo de este PSI es un paso fundamental para asegurar que nuestra plataforma de coleccionismo de cartas digitales se alinee con las necesidades y tendencias actuales del mercado, ofreciendo una solución innovadora que destaque en el ámbito del entretenimiento digital. Este plan nos guiará a través de las fases críticas del proyecto, desde la concepción hasta la implementación, garantizando el éxito y la sostenibilidad a largo plazo de la plataforma.



\subsection{Determinación de Responsables}


\newpage
\section{DEFINICIÓN Y ORGANIZACIÓN DEL PLAN DE SISTEMAS DE INFORMACIÓN}
 

\subsection{Especificación del Ámbito y Alcance} 


\subsection{Organización del Plan de Sistemas de Información }



\newpage
\section{ESTUDIO DE LA INFORMACIÓN RELEVANTE}
 
\subsection{Selección y Análisis de Antecedentes} 
