
\subsection{Caso de Uso 1} 

\textcolor[rgb]{0.65,0.16,0}{Ejemplo de tabla para análisis de casos de usos}

\begin{table}[H]
  \centering
  \vspace{-5mm}
  \caption{Análisis del Caso de Uso 1}
    \begin{tabular}{p{7.5em}p{24.145em}}
    \toprule
    \rowcolor[rgb]{ .871,  .918,  .965} \multicolumn{2}{p{31.645em}}{\textbf{Registro}} \\
    \midrule
    \rowcolor[rgb]{ .906,  .902,  .902} \textbf{Precondiciones} & \cellcolor[rgb]{ 1,  1,  1}El usuario no debe haber iniciado sesión nunca. \\
    \midrule
    \rowcolor[rgb]{ .906,  .902,  .902} \textbf{Postcondiciones} & \cellcolor[rgb]{ 1,  1,  1}- \\
    \midrule
    \rowcolor[rgb]{ .906,  .902,  .902} \textbf{Actores} & \cellcolor[rgb]{ 1,  1,  1}Usuario no registrado \\
    \midrule
    \rowcolor[rgb]{ .906,  .902,  .902} \textbf{Descripción} & \cellcolor[rgb]{ 1,  1,  1}El usuario accederá a la pantalla principal de la aplicación cuando no se está registrado, y seleccionará el botón de inicio de sesión, que, al ser la primera vez, registrará.Seleccionará la cuenta de Google con la que desee registrarse y el sistema completará el resto del registro. \\
    \midrule
    \rowcolor[rgb]{ .906,  .902,  .902} \textbf{Escenarios          Secundarios} & \cellcolor[rgb]{ 1,  1,  1} El usuario no tiene cuenta de Google: escenario que puede ser posible si accede a la aplicación a través del App Market. En este caso se le solicitará crear una cuenta. \\
    \bottomrule
    \end{tabular}%
\end{table}%
 
\subsection{Caso de Uso 2}

