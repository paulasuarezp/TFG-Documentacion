
En este apartado se detalla la arquitectura de los subsistemas de análisis. Como se ha visto en el apartado anterior, el sistema se divide en dos subsistemas principales: 
\textbf{restapi} y \textbf{frontend}. El subsistema restapi contiene las clases que implementan la API REST y la lógica de negocio, mientras que el subsistema frontend abarca la interfaz de usuario.

Primero, se presentará el diagrama de paquetes para ofrecer una visión general de la estructura del sistema. Este diagrama ilustrará la organización y 
agrupación de los distintos elementos del sistema en paquetes, permitiendo una comprensión clara de su disposición y jerarquía.

A continuación, se analizará cada subsistema en detalle mediante un diagrama de componentes por cada subsistema. Se ha optado por un diagrama de componentes debido a que el sistema sigue una arquitectura MERN 
(MongoDB, Express, React, Node.js) y este tipo de diagrama aporta más valor que un diagrama de clases al reflejar más fielmente la estructura y relaciones entre los distintos módulos del sistema.

Finalmente, se describirán los componentes identificados en el diagrama, explicando su función específica dentro del sistema y las relaciones que mantienen con otros componentes. 
Esta descripción detallada permitirá entender el papel de cada componente en el funcionamiento global del sistema y cómo colaboran entre sí para cumplir con los objetivos del sistema.

Por último, una vez descritos en detalle cada subsistema, se explicará el diagrama de despliegue. Este diagrama mostrará cómo se distribuyen físicamente los componentes del sistema en el entorno de 
ejecución, especificando las configuraciones de hardware y software necesarias, así como las interconexiones entre los distintos nodos del sistema.


\subsection{Diagrama de paquetes} 
El sistema se divide en dos paquetes principales:
\begin{itemize}
    \item \textbf{restapi}: se corresponde con el \textit{backend} del sistema, contiene las clases que implementan la API REST del sistema y la lógica de negocio.
    \item \textbf{frontend}: se corresponde con el \textit{frontend} del sistema, contiene las clases que implementan la interfaz de usuario.
\end{itemize}

En la \coloredUnderline{\hyperlink{fig:6_5_Diagrama-Paquetes}{Figura \ref*{fig:6_5_Diagrama-Paquetes}: \nameref*{fig:6_5_Diagrama-Paquetes}}} se muestra el diagrama de paquetes del sistema.
\begin{figure}[H]
    \hypertarget{fig:6_5_Diagrama-Paquetes}{}
    \centering
    \includegraphics[width=0.8\linewidth]{figures/6-Analisis/6-Clases/6_5-vista_general-paquetes.png}
    \caption{Diagrama de Paquetes del Sistema}
    \label{fig:6_5_Diagrama-Paquetes}
\end{figure}

\subsection{Descripción de los Paquetes}
\subsubsection{restapi}
El paquete \textbf{restapi} contiene las clases que implementan la API REST del sistema y la lógica de negocio. Este paquete se divide en los siguientes subpaquetes:
\begin{itemize}
    \item \textbf{controllers}: contiene las clases que implementan los controladores de la API REST, se encargan de gestionar las peticiones HTTP y las respuestas. Se comunica con la base de datos.
    \item \textbf{models}: contiene las clases que implementan los modelos de datos del sistema.
    \item \textbf{routes}: contiene las clases que implementan las rutas de la API REST, se encargan de definir las rutas y los métodos HTTP asociados.
    \item \textbf{middlewares}: contiene las clases que implementan los middlewares de la API REST, se encargan de gestionar la autenticación y la autorización de los usuarios.
    \item \textbf{scripts}: contiene las clases que implementan los scripts de inicialización de la base de datos.
    \item \textbf{tests}: contiene las clases que implementan las pruebas unitarias de las clases de los otros subpaquetes.
\end{itemize}


\subsubsection{frontend}
El paquete \textbf{frontend} contiene los archivos que implementan la interfaz de usuario. Este paquete se divide en los siguientes subpaquetes:
\begin{itemize} 
    \item \textbf{src}: contiene los archivos que implementan la lógica de la interfaz de usuario.
    \begin{itemize}
        \item \textbf{api}: contiene los archivos que implementan la API del frontend, se encargan de gestionar las peticiones HTTP/HTTPS al backend.
        \item \textbf{views}: contiene los componentes que implementan las vistas de la interfaz de usuario. A su vez, se divide en los siguientes subpaquetes:
        \begin{itemize}
            \item \textbf{components}: contiene los archivos que implementan los componentes de la interfaz de usuario.
            \item \textbf{pages}: contiene las archivos que implementan las páginas de la interfaz de usuario.
        \end{itemize}
        \item \textbf{redux}: contiene las archivos que implementan los estados de Redux.
        \item \textbf{socket}: contiene las archivos que implementan la conexión con Socket.io.
        \item \textbf{shared}: contiene las archivos que implementan los tipos de datos compartidos entre las distintas partes de la interfaz de usuario.
        \item \textbf{utils}: contiene las archivos que implementan utilidades de la interfaz de usuario.
    \end{itemize}
    \item \textbf{public}: contiene los archivos estáticos de la interfaz de usuario.
    \item \textbf{tests}: contiene las clases que implementan las pruebas de la interfaz de usuario.
\end{itemize}

\subsection{Diagramas de Componentes}
En el estándar UML se define como componente: 
\begin{quote}
    \"Un Componente representa una parte modular de un sistema que encapsula su contenido y cuya manifestación es reemplazable dentro de su entorno.
[...]
Un Componente especifica un contrato formal de los servicios que proporciona a sus clientes y aquellos que requiere de otros Componentes o servicios en el sistema en términos de sus Interfaces proporcionadas y requeridas.
Un Componente es una unidad sustituible que puede ser reemplazada en tiempo de diseño o en tiempo de ejecución por un Componente que ofrece una funcionalidad equivalente basada en la compatibilidad de sus Interfaces. 
Siempre que el entorno sea totalmente compatible con las Interfaces proporcionadas y requeridas de un Componente, este podrá interactuar con dicho entorno."
\end{quote}

\begin{flushright}
    \cite[p. 209]{UMLomg2017}
\end{flushright}

En el contexto de este proyecto, un componente es una parte modular del sistema que encapsula su contenido y, que en un futuro, su funcionalidad podría ser ampliada o reemplazada por otra siempre que cumpla con las interfaces proporcionadas y requeridas.

\subsubsection{Diagrama de componentes del subsistema restapi}
A continuación, se presenta el diagrama de componentes del subsistema \textbf{restapi}. En este diagrama se muestran los componentes que forman parte del subsistema y las relaciones entre ellos.

    \begin{landscape}
    \begin{figure}[H]
        \hypertarget{fig:6_5_Diagrama-Componentes-restapi}{}
        \centering
        \includegraphics[width=1\linewidth]{figures/6-Analisis/6-Clases/6_5-Componentes-restapi.png}
        \caption{Diagrama de componentes del subsistema restapi}
        \label{fig:6_5_Diagrama-Componentes-restapi}
    \end{figure}
    \end{landscape}

\newpage

Por cada componente identificado en la \coloredUnderline{\hyperlink{fig:6_5_Diagrama-Componentes-restapi}{Figura \ref*{fig:6_5_Diagrama-Componentes-restapi}: \nameref*{fig:6_5_Diagrama-Componentes-restapi}}},
se ha creado una tabla con su descripción detallada, explicando su función específica dentro del sistema y las relaciones que mantiene con otros componentes.

\subsubsubsection{Descripción de componentes del subsistema restapi. \textit{Server} y \textit{App}}
%--- SERVER ---
\begin{longtable}{
    >{\columncolor{lightgreen!20}}p{4cm}
    p{12cm}
    }
    \caption{Descripción del componente:  Server} \label{table:descripcion_server} \\
    \toprule
    \rowcolor{darkgreen!50}
    \textbf{Componente} & \multicolumn{1}{>{\columncolor{darkgreen!50}\centering\arraybackslash}p{12cm}}{\textbf{SERVER}} \\
    \endfirsthead
    
    \multicolumn{2}{c}%
    {{ \tablename\ \thetable{} Descripción del componente:  Server -- continuación de la página anterior}} \\
    \toprule
    \rowcolor{darkgreen!50}
    \textbf{Componente} & \multicolumn{1}{>{\columncolor{darkgreen!50}\centering\arraybackslash}p{12cm}}{\textbf{SERVER}} \\
    \midrule
    \endhead
    
    \midrule
    \multicolumn{2}{r}{{Continúa en la siguiente página...}} \\ 
    \endfoot
    
    \bottomrule
    \endlastfoot
    
    \midrule
    Descripción & Este componente configura y gestiona los servidores HTTP y HTTPS, la conexión a MongoDB, y la gestión de conexiones de sockets mediante Socket.IO. También incluye la configuración de variables de entorno y el manejo de errores. \\
    \midrule
    Métodos & \begin{itemize}[nosep,leftmargin=*]
      \item \textbf{config()}: void, configura las variables de entorno.
      \item \textbf{createServers()}: void, crea y configura los servidores HTTP y HTTPS.
      \item \textbf{connectToDatabase()}: void, conecta a la base de datos MongoDB.
      \item \textbf{startServers()}: void, inicia los servidores HTTP y HTTPS.
      \item \textbf{setupSocketIO()}: void, configura el middleware de autenticación de sockets y maneja eventos de conexión y desconexión.
      \item \textbf{closeServer()}: Promise<void>, cierra los servidores HTTP, HTTPS y la conexión a la base de datos.
    \end{itemize} \\
    \midrule
    Interfaces requeridas & \begin{itemize}[nosep,leftmargin=*]
      \item \textbf{App}: Usa el componente App, que configura las rutas de la API REST.
      \item \textbf{AuthSocket}: Usa el middleware de autenticación de sockets.
      \item \textbf{HTTP/HTTPS}: Para la conexión a los servidores HTTP y HTTPS.
    \end{itemize} \\
    \midrule
    Interfaces proporcionadas & \begin{itemize}[nosep,leftmargin=*]
      \item \textbf{HTTP/HTTPS}: Proporciona la conexión a los servidores HTTP y HTTPS.
    \end{itemize} \\
\end{longtable}

%--- APP ---

\begin{longtable}{
    >{\columncolor{lightgreen!20}}p{4cm}
    p{12cm}
    }
    \caption{Descripción del componente:  App} \label{table:descripcion_app} \\
    \toprule
    \rowcolor{darkgreen!50}
    \textbf{Componente} & \multicolumn{1}{>{\columncolor{darkgreen!50}\centering\arraybackslash}p{12cm}}{\textbf{APP}} \\
    \endfirsthead
    
    \multicolumn{2}{c}%
    {{ \tablename\ \thetable{} Descripción del componente:  App -- continuación de la página anterior}} \\
    \toprule
    \rowcolor{darkgreen!50}
    \textbf{Componente} & \multicolumn{1}{>{\columncolor{darkgreen!50}\centering\arraybackslash}p{12cm}}{\textbf{APP}} \\
    \midrule
    \endhead
    
    \midrule
    \multicolumn{2}{r}{{Continúa en la siguiente página...}} \\ 
    \endfoot
    
    \bottomrule
    \endlastfoot
    
    \midrule
    Descripción & Este componente configura y gestiona las políticas CORS, las rutas de la API y el middleware para el manejo de errores. \\
    \midrule
    Métodos & \begin{itemize}[nosep,leftmargin=*]
      \item \textbf{use()}: void, permite configurar las rutas y los middlewares de la aplicación.
      \item \textbf{listen()}: void, inicia el servidor en el puerto especificado.
      \item \textbf{errorHandler()}: void, middleware para manejar errores en la aplicación.
    \end{itemize} \\
    \midrule
    Interfaces requeridas & \begin{itemize}[nosep,leftmargin=*]
      \item \textbf{Endpoints}: Utiliza todas las rutas de la API REST, definidas en el paquete \textit{routes}. Estas son:
        \begin{itemize}[nosep,leftmargin=*]
        \item \textbf{AuctionRouter}: Rutas que gestionan las subastas.
        \item \textbf{BidRouter}: Rutas que gestionan las pujas.
        \item \textbf{CardPackRouter}: Rutas que gestionan los sobres de cartas.
        \item \textbf{CardRouter}: Rutas que gestionan las cartas.
        \item \textbf{DeckRouter}: Rutas que gestionan los mazos de cartas.
        \item \textbf{NotificationRouter}: Rutas que gestionan las notificaciones.
        \item \textbf{PaypalRouter}: Rutas que gestionan las transacciones de PayPal.
        \item \textbf{PurchasesRouter}: Rutas que gestionan las compras.
        \item \textbf{TransactionRouter}: Rutas que gestionan las transacciones propias de la aplicación.
        \item \textbf{UserCardRouter}: Rutas que gestionan las cartas de usuario.
        \item \textbf{UserRouter}: Rutas que gestionan los usuarios.
        \end{itemize}
    \end{itemize} \\
    \midrule
    Interfaces proporcionadas & \begin{itemize}[nosep,leftmargin=*]
      \item \textbf{Configured endpoints}: Proporciona la aplicación de Express con las rutas y middlewares configurados.   
    \end{itemize} \\
\end{longtable}


\subsubsubsection{Descripción de componentes del subsistema restapi. Paquete \textit{middlewares}}\label{sec:descripcion_authmiddleware}
%--- AUTHMIDDLEWARE ---
\begin{longtable}{
    >{\columncolor{lightgreen!20}}p{4cm}
    p{12cm}
    }
    \caption{Descripción del componente:  AuthMiddleware} \label{table:descripcion_authmiddleware} \\
    \toprule
    \rowcolor{darkgreen!50}
    \textbf{Componente} & \multicolumn{1}{>{\columncolor{darkgreen!50}\centering\arraybackslash}p{12cm}}{\textbf{AUTHMIDDLEWARE}} \\
    \endfirsthead
    
    \multicolumn{2}{c}%
    {{ \tablename\ \thetable{} Descripción del componente:  AuthMiddleware -- continuación de la página anterior}} \\
    \toprule
    \rowcolor{darkgreen!50}
    \textbf{Componente} & \multicolumn{1}{>{\columncolor{darkgreen!50}\centering\arraybackslash}p{12cm}}{\textbf{AUTHMIDDLEWARE}} \\
    \midrule
    \endhead
    
    \midrule
    \multicolumn{2}{r}{{Continúa en la siguiente página...}} \\ 
    \endfoot
    
    \bottomrule
    \endlastfoot
    
    \midrule
    Descripción & Este componente proporciona middleware para la autenticación y autorización de usuarios mediante tokens JWT. Incluye la verificación de tokens y la verificación de roles de administrador. \\
    \midrule
    Métodos & \begin{itemize}[nosep,leftmargin=*]
      \item \textbf{auth(req: Request, res: Response, next: any)}: void, middleware para verificar la autenticidad del token JWT en las peticiones.
      \item \textbf{verifyAdmin(req: Request, res: Response, next: any)}: void, middleware para verificar que el usuario tiene rol de administrador.
    \end{itemize} \\
    \midrule
    Interfaces requeridas &  \\
    \midrule
    Interfaces proporcionadas & \begin{itemize}[nosep,leftmargin=*]
      \item \textbf{User Authentication}: Proporciona middleware para la autenticación de usuarios, verificando la validez del token JWT y, ofreciendo la posibilidad de verificar roles de administrador.
    \end{itemize} \\
    \end{longtable}

%--- AUTHSOCKET ---
\begin{longtable}{
    >{\columncolor{lightgreen!20}}p{4cm}
    p{12cm}
    }
    \caption{Descripción del componente:  AuthSocket} \label{table:descripcion_authsocket} \\
    \toprule
    \rowcolor{darkgreen!50}
    \textbf{Componente} & \multicolumn{1}{>{\columncolor{darkgreen!50}\centering\arraybackslash}p{12cm}}{\textbf{AUTHSOCKET}} \\
    \endfirsthead
    
    \multicolumn{2}{c}%
    {{ \tablename\ \thetable{} Descripción del componente:  AuthSocket -- continuación de la página anterior}} \\
    \toprule
    \rowcolor{darkgreen!50}
    \textbf{Componente} & \multicolumn{1}{>{\columncolor{darkgreen!50}\centering\arraybackslash}p{12cm}}{\textbf{AUTHSOCKET}} \\
    \midrule
    \endhead
    
    \midrule
    \multicolumn{2}{r}{{Continúa en la siguiente página...}} \\ 
    \endfoot
    
    \bottomrule
    \endlastfoot
    
    \midrule
    Descripción & Este componente proporciona el middleware para la autenticación de conexiones de sockets mediante tokens JWT. Verifica la validez y el formato del token proporcionado en el handshake de la conexión del socket. \\
    \midrule
    Métodos & \begin{itemize}[nosep,leftmargin=*]
      \item \textbf{authSocket(socket: Socket, next: (err?: Error) => void)}: void, middleware para verificar la autenticidad del token JWT en las conexiones de sockets.
    \end{itemize} \\
    \midrule
    Interfaces requeridas &  \\
    \midrule
    Interfaces proporcionadas & \begin{itemize}[nosep,leftmargin=*]
      \item \textbf{Socket Authentication}: Proporciona middleware para la autenticación de conexiones de sockets, verificando la validez del token JWT.
    \end{itemize} \\
    \end{longtable}


\subsubsection{Descripción de componentes del subsistema restapi. Paquete \textit{routes}}
%--- AUCTIONROUTER ---
\begin{longtable}{
    >{\columncolor{lightgreen!20}}p{4cm}
    p{12cm}
    }
    \caption{Descripción del componente:  AuctionRouter} \label{table:descripcion_auctionrouter} \\
    \toprule
    \rowcolor{darkgreen!50}
    \textbf{Componente} & \multicolumn{1}{>{\columncolor{darkgreen!50}\centering\arraybackslash}p{12cm}}{\textbf{AUCTIONROUTER}} \\
    \endfirsthead
    
    \multicolumn{2}{c}%
    {{ \tablename\ \thetable{} Descripción del componente:  AuctionRouter -- continuación de la página anterior}} \\
    \toprule
    \rowcolor{darkgreen!50}
    \textbf{Componente} & \multicolumn{1}{>{\columncolor{darkgreen!50}\centering\arraybackslash}p{12cm}}{\textbf{AUCTIONROUTER}} \\
    \midrule
    \endhead
    
    \midrule
    \multicolumn{2}{r}{{Continúa en la siguiente página...}} \\ 
    \endfoot
    
    \bottomrule
    \endlastfoot
    
    \midrule
    Descripción & Este componente configura y gestiona las rutas relacionadas con las subastas en la aplicación Express. Incluye la autenticación mediante middleware y validaciones para las peticiones. \\
    \midrule
    Atributos & \begin{itemize}[nosep,leftmargin=*]
      \item \textbf{auctionRouter}: Router, instancia del enrutador de Express para las subastas.
    \end{itemize} \\
    \midrule
    Métodos & \begin{itemize}[nosep,leftmargin=*]
      \item \textbf{getAuctions(req: Request, res: Response)}: void, maneja la obtención de todas las subastas.
      \item \textbf{getAuction(req: Request, res: Response)}: void, maneja la obtención de una subasta por su ID.
      \item \textbf{getActiveAuctions(req: Request, res: Response)}: void, maneja la obtención de todas las subastas activas.
      \item \textbf{getActiveAuctionsByUser(req: Request, res: Response)}: void, maneja la obtención de todas las subastas activas de un usuario.
      \item \textbf{putUserCardUpForAuction(req: Request, res: Response)}: void, maneja la puesta en subasta de una carta de usuario.
      \item \textbf{withdrawnUserCardFromAuction(req: Request, res: Response)}: void, maneja la retirada de una carta de usuario de una subasta.
      \item \textbf{checkAllActiveAuctions(req: Request, res: Response)}: void, verifica todas las subastas activas y actualiza su estado si es necesario.
    \end{itemize} \\
    \midrule
    Interfaces requeridas &  \begin{itemize}[nosep,leftmargin=*]
      \item \textbf{User Authentication}: Middleware para la autenticación de usuarios.
      \item \textbf{Auction MGMT.}: Utiliza los métodos definidos en el controlador de subastas, \textit{AuctionController}.
    \end{itemize} \\
    \midrule
    Interfaces proporcionadas & \begin{itemize}[nosep,leftmargin=*]
      \item \textbf{Auction Router}: Proporciona middleware para la autenticación de usuarios, verificando la validez del token JWT y, ofreciendo la posibilidad de verificar roles de administrador.
    \end{itemize} \\
    \end{longtable}

%--- BIDROUTER ---
\begin{longtable}{
    >{\columncolor{lightgreen!20}}p{4cm}
    p{12cm}
    }
    \caption{Descripción del componente:  BidRouter} \label{table:descripcion_bidrouter} \\
    \toprule
    \rowcolor{darkgreen!50}
    \textbf{Componente} & \multicolumn{1}{>{\columncolor{darkgreen!50}\centering\arraybackslash}p{12cm}}{\textbf{BIDROUTER}} \\
    \endfirsthead
    
    \multicolumn{2}{c}%
    {{ \tablename\ \thetable{} Descripción del componente:  BidRouter -- continuación de la página anterior}} \\
    \toprule
    \rowcolor{darkgreen!50}
    \textbf{Componente} & \multicolumn{1}{>{\columncolor{darkgreen!50}\centering\arraybackslash}p{12cm}}{\textbf{BIDROUTER}} \\
    \midrule
    \endhead
    
    \midrule
    \multicolumn{2}{r}{{Continúa en la siguiente página...}} \\ 
    \endfoot
    
    \bottomrule
    \endlastfoot
    
    \midrule
    Descripción & Esta clase configura y gestiona las rutas relacionadas con las pujas en la aplicación Express. Incluye la autenticación mediante middleware y validaciones para las peticiones. \\
    \midrule
    Atributos & \begin{itemize}[nosep,leftmargin=*]
      \item \textbf{bidRouter}: Router, instancia del enrutador de Express para las pujas.
    \end{itemize} \\
    \midrule
    Métodos & \begin{itemize}[nosep,leftmargin=*]
      \item \textbf{getBidById(req: Request, res: Response)}: void, maneja la obtención de una puja por su ID.
      \item \textbf{createBid(req: Request, res: Response)}: void, maneja la creación de una nueva puja.
      \item \textbf{getActiveBidsByUser(req: Request, res: Response)}: void, maneja la obtención de todas las pujas activas de un usuario.
      \item \textbf{withdrawBid(req: Request, res: Response)}: void, maneja la retirada de una puja.
    \end{itemize} \\
    \midrule
    Relaciones & \begin{itemize}[nosep,leftmargin=*]
      \item \textbf{AuthMiddleware}: Middleware para la autenticación de usuarios.
      \item \textbf{BidController}: Importa y utiliza métodos del controlador de pujas.
    \end{itemize} \\
    \midrule
    Interfaces requeridas &  \\
    \midrule
    Interfaces proporcionadas & \begin{itemize}[nosep,leftmargin=*]
      \item \textbf{User Authentication}: Proporciona middleware para la autenticación de usuarios, verificando la validez del token JWT y, ofreciendo la posibilidad de verificar roles de administrador.
    \end{itemize} \\
    \end{longtable}

\subsubsubsection{Descripción del componente:  CardPackRouter} \label{sec:descripcion_cardpackrouter}
\begin{longtable}{
    >{\columncolor{lightgreen!20}}p{4cm}
    p{12cm}
    }
    \caption{Descripción del componente:  CardPackRouter} \label{table:descripcion_cardpackrouter} \\
    \toprule
    \rowcolor{darkgreen!50}
    \textbf{Componente} & \multicolumn{1}{>{\columncolor{darkgreen!50}\centering\arraybackslash}p{12cm}}{\textbf{CARDPACKROUTER}} \\
    \endfirsthead
    
    \multicolumn{2}{c}%
    {{ \tablename\ \thetable{} Descripción del componente:  CardPackRouter -- continuación de la página anterior}} \\
    \toprule
    \rowcolor{darkgreen!50}
    \textbf{Componente} & \multicolumn{1}{>{\columncolor{darkgreen!50}\centering\arraybackslash}p{12cm}}{\textbf{CARDPACKROUTER}} \\
    \midrule
    \endhead
    
    \midrule
    \multicolumn{2}{r}{{Continúa en la siguiente página...}} \\ 
    \endfoot
    
    \bottomrule
    \endlastfoot
    
    \midrule
    Descripción & Esta clase configura y gestiona las rutas relacionadas con los sobres de cartas en la aplicación Express. Incluye la autenticación mediante middleware. \\
    \midrule
    Atributos & \begin{itemize}[nosep,leftmargin=*]
      \item \textbf{cardPackRouter}: Router, instancia del enrutador de Express para los sobres de cartas.
    \end{itemize} \\
    \midrule
    Métodos & \begin{itemize}[nosep,leftmargin=*]
      \item \textbf{getCardPacks(req: Request, res: Response)}: void, maneja la obtención de todos los sobres de cartas.
    \end{itemize} \\
    \midrule
    Relaciones & \begin{itemize}[nosep,leftmargin=*]
      \item \textbf{AuthMiddleware}: Middleware para la autenticación de usuarios.
      \item \textbf{CardPackController}: Importa y utiliza el método del controlador de sobres de cartas.
    \end{itemize} \\
    \midrule
    Interfaces requeridas &  \\
    \midrule
    Interfaces proporcionadas & \begin{itemize}[nosep,leftmargin=*]
      \item \textbf{User Authentication}: Proporciona middleware para la autenticación de usuarios, verificando la validez del token JWT y, ofreciendo la posibilidad de verificar roles de administrador.
    \end{itemize} \\
    \end{longtable}


\subsubsubsection{Descripción del componente:  CardRouter} \label{sec:descripcion_cardrouter}
\begin{longtable}{
    >{\columncolor{lightgreen!20}}p{4cm}
    p{12cm}
    }
    \caption{Descripción del componente:  CardRouter} \label{table:descripcion_cardrouter} \\
    \toprule
    \rowcolor{darkgreen!50}
    \textbf{Componente} & \multicolumn{1}{>{\columncolor{darkgreen!50}\centering\arraybackslash}p{12cm}}{\textbf{CARDROUTER}} \\
    \endfirsthead
    
    \multicolumn{2}{c}%
    {{ \tablename\ \thetable{} Descripción del componente:  CardRouter -- continuación de la página anterior}} \\
    \toprule
    \rowcolor{darkgreen!50}
    \textbf{Componente} & \multicolumn{1}{>{\columncolor{darkgreen!50}\centering\arraybackslash}p{12cm}}{\textbf{CARDROUTER}} \\
    \midrule
    \endhead
    
    \midrule
    \multicolumn{2}{r}{{Continúa en la siguiente página...}} \\ 
    \endfoot
    
    \bottomrule
    \endlastfoot
    
    \midrule
    Descripción & Esta clase configura y gestiona las rutas relacionadas con las cartas en la aplicación Express. Incluye la autenticación mediante middleware y validaciones para las peticiones. \\
    \midrule
    Atributos & \begin{itemize}[nosep,leftmargin=*]
      \item \textbf{cardRouter}: Router, instancia del enrutador de Express para las cartas.
    \end{itemize} \\
    \midrule
    Métodos & \begin{itemize}[nosep,leftmargin=*]
      \item \textbf{getCard(req: Request, res: Response)}: void, maneja la obtención de una carta por su ID.
    \end{itemize} \\
    \midrule
    Relaciones & \begin{itemize}[nosep,leftmargin=*]
      \item \textbf{AuthMiddleware}: Middleware para la autenticación de usuarios.
      \item \textbf{CardController}: Importa y utiliza el método del controlador de cartas.
    \end{itemize} \\
    \midrule
    Interfaces requeridas &  \\
    \midrule
    Interfaces proporcionadas & \begin{itemize}[nosep,leftmargin=*]
      \item \textbf{User Authentication}: Proporciona middleware para la autenticación de usuarios, verificando la validez del token JWT y, ofreciendo la posibilidad de verificar roles de administrador.
    \end{itemize} \\
    \end{longtable}

\subsubsubsection{Descripción del componente:  NotificationRouter} \label{sec:descripcion_notificationrouter}
\begin{longtable}{
    >{\columncolor{lightgreen!20}}p{4cm}
    p{12cm}
    }
    \caption{Descripción del componente:  NotificationRouter} \label{table:descripcion_notificationrouter} \\
    \toprule
    \rowcolor{darkgreen!50}
    \textbf{Componente} & \multicolumn{1}{>{\columncolor{darkgreen!50}\centering\arraybackslash}p{12cm}}{\textbf{NOTIFICATIONROUTER}} \\
    \endfirsthead
    
    \multicolumn{2}{c}%
    {{ \tablename\ \thetable{} Descripción del componente:  NotificationRouter -- continuación de la página anterior}} \\
    \toprule
    \rowcolor{darkgreen!50}
    \textbf{Componente} & \multicolumn{1}{>{\columncolor{darkgreen!50}\centering\arraybackslash}p{12cm}}{\textbf{NOTIFICATIONROUTER}} \\
    \midrule
    \endhead
    
    \midrule
    \multicolumn{2}{r}{{Continúa en la siguiente página...}} \\ 
    \endfoot
    
    \bottomrule
    \endlastfoot
    
    \midrule
    Descripción & Esta clase configura y gestiona las rutas relacionadas con las notificaciones en la aplicación Express. Incluye la autenticación mediante middleware y validaciones para las peticiones. \\
    \midrule
    Atributos & \begin{itemize}[nosep,leftmargin=*]
      \item \textbf{notificationRouter}: Router, instancia del enrutador de Express para las notificaciones.
    \end{itemize} \\
    \midrule
    Métodos & \begin{itemize}[nosep,leftmargin=*]
      \item \textbf{getNotifications(req: Request, res: Response)}: void, maneja la obtención de todas las notificaciones de un usuario.
      \item \textbf{markAsRead(req: Request, res: Response)}: void, maneja el marcado de una notificación como leída.
      \item \textbf{markAllAsRead(req: Request, res: Response)}: void, maneja el marcado de todas las notificaciones de un usuario como leídas.
      \item \textbf{hasUnreadNotifications(req: Request, res: Response)}: void, verifica si un usuario tiene notificaciones no leídas.
    \end{itemize} \\
    \midrule
    Relaciones & \begin{itemize}[nosep,leftmargin=*]
      \item \textbf{AuthMiddleware}: Middleware para la autenticación de usuarios.
      \item \textbf{NotificationController}: Importa y utiliza métodos del controlador de notificaciones.
    \end{itemize} \\
    \midrule
    Interfaces requeridas &  \\
    \midrule
    Interfaces proporcionadas & \begin{itemize}[nosep,leftmargin=*]
      \item \textbf{User Authentication}: Proporciona middleware para la autenticación de usuarios, verificando la validez del token JWT y, ofreciendo la posibilidad de verificar roles de administrador.
    \end{itemize} \\
    \end{longtable}

    \subsubsubsection{Descripción del componente:  PaypalRouter} \label{sec:descripcion_paypalrouter}
\begin{longtable}{
    >{\columncolor{lightgreen!20}}p{4cm}
    p{12cm}
    }
    \caption{Descripción del componente:  PaypalRouter} \label{table:descripcion_paypalrouter} \\
    \toprule
    \rowcolor{darkgreen!50}
    \textbf{Componente} & \multicolumn{1}{>{\columncolor{darkgreen!50}\centering\arraybackslash}p{12cm}}{\textbf{PAYPALROUTER}} \\
    \endfirsthead
    
    \multicolumn{2}{c}%
    {{ \tablename\ \thetable{} Descripción del componente:  PaypalRouter -- continuación de la página anterior}} \\
    \toprule
    \rowcolor{darkgreen!50}
    \textbf{Componente} & \multicolumn{1}{>{\columncolor{darkgreen!50}\centering\arraybackslash}p{12cm}}{\textbf{PAYPALROUTER}} \\
    \midrule
    \endhead
    
    \midrule
    \multicolumn{2}{r}{{Continúa en la siguiente página...}} \\ 
    \endfoot
    
    \bottomrule
    \endlastfoot
    
    \midrule
    Descripción & Esta clase configura y gestiona las rutas relacionadas con las órdenes de PayPal en la aplicación Express. Incluye validaciones para las peticiones y manejo de errores. \\
    \midrule
    Atributos & \begin{itemize}[nosep,leftmargin=*]
      \item \textbf{paypalRouter}: Router, instancia del enrutador de Express para las órdenes de PayPal.
    \end{itemize} \\
    \midrule
    Métodos & \begin{itemize}[nosep,leftmargin=*]
      \item \textbf{createOrder(req: Request, res: Response)}: void, maneja la creación de una nueva orden de PayPal.
      \item \textbf{updateOrder(req: Request, res: Response)}: void, maneja la actualización del saldo de un usuario después de completar un pago.
    \end{itemize} \\
    \midrule
    Relaciones & \begin{itemize}[nosep,leftmargin=*]
      \item \textbf{PaypalController}: Importa y utiliza el método del controlador de PayPal para crear órdenes.
      \item \textbf{User}: Importa y utiliza el modelo de usuario para actualizar el saldo.
    \end{itemize} \\
    \midrule
    Interfaces requeridas &  \\
    \midrule
    Interfaces proporcionadas & \begin{itemize}[nosep,leftmargin=*]
      \item \textbf{User Authentication}: Proporciona middleware para la autenticación de usuarios, verificando la validez del token JWT y, ofreciendo la posibilidad de verificar roles de administrador.
    \end{itemize} \\
    \end{longtable}




\subsubsubsection{Descripción de los Componentes del Subsistema restapi}


\subsubsection{webapp}
A continuación, se presenta el diagrama de componentes del subsistema \textbf{webapp}. En este diagrama se muestran los componentes que forman parte del subsistema.
Se han omitido las relaciones entre los componentes para simplificar el diagrama. Las relaciones entre los componentes se explicarán en la descripción de los componentes.
\begin{landscape}
    \begin{figure}[H]
        \hypertarget{fig:6_5_Diagrama-Componentes-webapp}{}
        \centering
        \includegraphics[width=1\linewidth]{figures/6-Analisis/6-Clases/6_5-Componentes-webapp.png}
        \caption{Diagrama de componentes del subsistema webapp}
        \label{fig:6_5_Diagrama-Componentes-webapp}
    \end{figure}
    \end{landscape}

\newpage

\subsubsubsection{Descripción de los Componentes del Subsistema webapp}
En este apartado se describirá la funcionalidad de los componentes del subsistema webapp. 
\begin{itemize}
    \item \textbf{Index}: Es el punto de entrada de la aplicación. Este componente se encarga del renderizado del componente \textit{App}.
    \item \textbf{App}: es el componente principal de la aplicación. Este componente define el diagrama de navegabilidad de la aplicación y se encarga de renderizar los componentes de la aplicación en función de la ruta actual. 
    Se relaciona, por lo tanto, con el paquete \textit{pages} y con el componente \textit{Theme} para definir el tema de la aplicación.
    \item \textbf{Theme}: es el componente que define el tema de la aplicación. Este componente se encarga de definir los colores y tipografías de la aplicación.
    \item \textbf{API}: es el componente que se encarga de realizar las peticiones a la API. Este componente se relaciona con el paquete \textit{shared} para definir los tipos de datos compartidos entre las distintas partes de la aplicación.
    \item \textbf{Redux}: es el componente que se encarga de gestionar el estado de la aplicación. Gestiona los siguientes estados:
    \begin{itemize}
        \item \textbf{user}: contiene la información del usuario autenticado.
        \item \textbf{notification}: estado para gestionar las visualizaciones de las notificaciones en tiempo real.
        \item \textbf{update}: estado para gestionar las actualizaciones de los datos de la aplicación.
    \end{itemize}
    \item \textbf{Socket}: es el componente que se encarga de gestionar la conexión con Socket.io. Concretamente se utiliza para que los usuarios puedan recibir notificaciones en tiempo real.
    \item \textbf{Shared}: es el componente que se encarga de definir los tipos de datos compartidos entre las distintas partes de la aplicación.
    \item \textbf{Utils}: es el componente que se encarga de definir utilidades de la aplicación. Contiene las siguientes utilidades:
    \begin{itemize}
        \item \textbf{cardData}: Colección de cartas de ejemplo para la aplicación. Se utiliza en un componente de la aplicación para mostrar cartas de ejemplo.
        \item \textbf{fieldsValidation}: utilidad para validar los campos de un formulario.
        \item \textbf{PrivateRoute}: componente que implementa una ruta privada de la aplicación. Verifica que un usuario esté autenticado antes de renderizar un componente.
        \item \textbf{RouteRedirector}: componente que implementa un redireccionador de rutas de la aplicación. Redirige a la ruta solicitada si el usuario cumple con el rol requerido, en caso contrario redirige a la ruta por defecto.
        \item \textbf{utils}: funciones de utilidad para la aplicación, como conversión de fechas y generación de mensajes.
    \end{itemize}
    \item \textbf{views}: contiene los componentes que implementan las vistas de la aplicación, y se divide en los siguientes subpaquetes:
    \begin{itemize}
        \item \textbf{components}: contiene los archivos que implementan los componentes de la aplicación. 
        Estos se usan para definir componentes más complejos que se reutilizan en distintas partes de la aplicación.
        \begin{itemize}
            \item \textbf{Button}: componente que implementa un botón de la aplicación. En este componente se definen los distintos tipos de botones que se utilizan en la aplicación.
            \item \textbf{Calendar}: componente que implementa un calendario de la aplicación.
            \item \textbf{Card}: componente que implementa una carta de la aplicación. 
            \item \textbf{CardDetail}: componente que implementa el modelo base de un detalle de carta.
            \item \textbf{Cardpack}: componente que implementa un sobre de cartas de la aplicación.
            \item \textbf{Container}: componente que implementa distintos tipos de contenedores de la aplicación.
            \item \textbf{Form}: componente que define los formularios de la aplicación.
            \item \textbf{Footer}: componente que implementa el pie de página de la aplicación.
            \item \textbf{Header}: componente que implementa la cabecera de la aplicación.
            \item \textbf{LogoBox}: componente que implementa el logotipo de la aplicación.
            \item \textbf{Menu}: componente que implementa los distintos menús de la aplicación.
            \item \textbf{Messages}: componente que implementa los distintos mensajes informativos de la aplicación.
            \item \textbf{Ornament}: componente que implementa distintos adornos de la aplicación.
            \item \textbf{Paper}: componente que implementa un contenedor con sombra, se utiliza para mostrar información en la aplicación.
            \item \textbf{Switch}: componente que implementa un interruptor de la aplicación. Concretamente, se utiliza para cambiar entre los modos claro y oscuro de la temática de la aplicación.
            \item \textbf{Table}: componente que implementa una tabla de la aplicación.
        \end{itemize}
        \item \textbf{pages}: contiene los archivos que implementan las páginas de la aplicación.
        \begin{itemize}
            \item \textbf{BasePage}: página que implementa la estructura base de una página de la aplicación.
            \item \textbf{BasePageWithNav}: página que implementa la estructura base de una página de la aplicación con navegación.
            \item \textbf{NotFoundPage}: página que implementa la página de error 404.
            \item En el paquete \textit{guest} se encuentran las páginas que puede ver cualquier usuario:
            \begin{itemize}
                \item \textbf{Home}: página principal de la aplicación.
                \item \textbf{Login}: página de inicio de sesión.
                \item \textbf{SignUp}: página de registro de usuario.
                \item \textbf{About}: página de información sobre la aplicación.
            \end{itemize}
             \item En el paquete \textit{admin} se encuentran las páginas que puede ver un usuario con rol de administrador:
            \begin{itemize}
                \item \textbf{AdminPage}: página de administración de la aplicación.
                \item \textbf{AdminTransactions}: página de administración de usuarios.
                \item \textbf{AdminAuctionDetail}: página de administración de transacciones.
                \item \textbf{AuctionsAdmin}: página de administración de subastas.
            \end{itemize}
            \item En el paquete \textit{standard} se encuentran las páginas que puede ver un usuario autenticado con rol de usuario no administrador:
            \begin{itemize}
                \item \textbf{Logueado}: página de inicio del usuario autenticado.
                \item \textbf{EditProfile}: página de perfil del usuario autenticado.
                \item \textbf{CardDetail}: página de detalle de carta.
                \item \textbf{Shop}: página de tienda, en ella el usuario puede adquirir sobres de cartas.
                \item \textbf{Auctions}: página de subastas, en ella el usuario puede participar en subastas.
                \item \textbf{Transactions}: página de transacciones, en ella el usuario puede consultar las subastas realizadas.
            \end{itemize}
        \end{itemize}
    \end{itemize}
\end{itemize}
