Por cada componente identificado en la \coloredUnderline{\hyperlink{fig:6_5_Diagrama-Componentes-restapi}{Figura \ref*{fig:6_5_Diagrama-Componentes-restapi}: \nameref*{fig:6_5_Diagrama-Componentes-restapi}}},
se ha creado una tabla con su descripción detallada, explicando su función específica dentro del sistema y las relaciones que mantiene con otros componentes.

\subsubsubsection{Descripción de componentes del subsistema restapi. \textit{Server} y \textit{App}}
%--- SERVER ---
\begin{longtable}{
    >{\columncolor{lightgreen!20}}p{4cm}
    p{12cm}
    }
    \caption{Descripción del componente:  Server} \label{table:descripcion_server} \\
    \toprule
    \rowcolor{darkgreen!50}
    \textbf{Componente} & \multicolumn{1}{>{\columncolor{darkgreen!50}\centering\arraybackslash}p{12cm}}{\textbf{SERVER}} \\
    \endfirsthead
    
    \multicolumn{2}{c}%
    {{ \tablename\ \thetable{} Descripción del componente:  Server -- continuación de la página anterior}} \\
    \toprule
    \rowcolor{darkgreen!50}
    \textbf{Componente} & \multicolumn{1}{>{\columncolor{darkgreen!50}\centering\arraybackslash}p{12cm}}{\textbf{SERVER}} \\
    \midrule
    \endhead
    
    \midrule
    \multicolumn{2}{r}{{Continúa en la siguiente página...}} \\ 
    \endfoot
    
    \bottomrule
    \endlastfoot
    
    \midrule
    Descripción & Este componente configura y gestiona los servidores HTTP y HTTPS, la conexión a MongoDB, y la gestión de conexiones de sockets mediante Socket.IO. También incluye la configuración de variables de entorno y el manejo de errores. \\
    \midrule
    Métodos & \begin{itemize}[nosep,leftmargin=*]
      \item \textbf{config()}: void, configura las variables de entorno.
      \item \textbf{createServers()}: void, crea y configura los servidores HTTP y HTTPS.
      \item \textbf{connectToDatabase()}: void, conecta a la base de datos MongoDB.
      \item \textbf{startServers()}: void, inicia los servidores HTTP y HTTPS.
      \item \textbf{setupSocketIO()}: void, configura el middleware de autenticación de sockets y maneja eventos de conexión y desconexión.
      \item \textbf{closeServer()}: Promise<void>, cierra los servidores HTTP, HTTPS y la conexión a la base de datos.
    \end{itemize} \\
    \midrule
    Interfaces requeridas & \begin{itemize}[nosep,leftmargin=*]
      \item \textbf{App}: Usa el componente App, que configura las rutas de la API REST.
      \item \textbf{AuthSocket}: Usa el middleware de autenticación de sockets.
      \item \textbf{HTTP/HTTPS}: Para la conexión a los servidores HTTP y HTTPS.
    \end{itemize} \\
    \midrule
    Interfaces proporcionadas & \begin{itemize}[nosep,leftmargin=*]
      \item \textbf{HTTP/HTTPS}: Proporciona la conexión a los servidores HTTP y HTTPS.
    \end{itemize} \\
\end{longtable}

%--- APP ---

\begin{longtable}{
    >{\columncolor{lightgreen!20}}p{4cm}
    p{12cm}
    }
    \caption{Descripción del componente:  App} \label{table:descripcion_app} \\
    \toprule
    \rowcolor{darkgreen!50}
    \textbf{Componente} & \multicolumn{1}{>{\columncolor{darkgreen!50}\centering\arraybackslash}p{12cm}}{\textbf{APP}} \\
    \endfirsthead
    
    \multicolumn{2}{c}%
    {{ \tablename\ \thetable{} Descripción del componente:  App -- continuación de la página anterior}} \\
    \toprule
    \rowcolor{darkgreen!50}
    \textbf{Componente} & \multicolumn{1}{>{\columncolor{darkgreen!50}\centering\arraybackslash}p{12cm}}{\textbf{APP}} \\
    \midrule
    \endhead
    
    \midrule
    \multicolumn{2}{r}{{Continúa en la siguiente página...}} \\ 
    \endfoot
    
    \bottomrule
    \endlastfoot
    
    \midrule
    Descripción & Este componente configura y gestiona las políticas CORS, las rutas de la API y el middleware para el manejo de errores. \\
    \midrule
    Métodos & \begin{itemize}[nosep,leftmargin=*]
      \item \textbf{use()}: void, permite configurar las rutas y los middlewares de la aplicación.
      \item \textbf{listen()}: void, inicia el servidor en el puerto especificado.
      \item \textbf{errorHandler()}: void, middleware para manejar errores en la aplicación.
    \end{itemize} \\
    \midrule
    Interfaces requeridas & \begin{itemize}[nosep,leftmargin=*]
      \item \textbf{Endpoints}: Utiliza todas las rutas de la API REST, definidas en el paquete \textit{routes}. Estas son:
        \begin{itemize}[nosep,leftmargin=*]
        \item \textbf{AuctionRouter}: Rutas que gestionan las subastas.
        \item \textbf{BidRouter}: Rutas que gestionan las pujas.
        \item \textbf{CardPackRouter}: Rutas que gestionan los sobres de cartas.
        \item \textbf{CardRouter}: Rutas que gestionan las cartas.
        \item \textbf{DeckRouter}: Rutas que gestionan los mazos de cartas.
        \item \textbf{NotificationRouter}: Rutas que gestionan las notificaciones.
        \item \textbf{PaypalRouter}: Rutas que gestionan las transacciones de PayPal.
        \item \textbf{PurchasesRouter}: Rutas que gestionan las compras.
        \item \textbf{TransactionRouter}: Rutas que gestionan las transacciones propias de la aplicación.
        \item \textbf{UserCardRouter}: Rutas que gestionan las cartas de usuario.
        \item \textbf{UserRouter}: Rutas que gestionan los usuarios.
        \end{itemize}
    \end{itemize} \\
    \midrule
    Interfaces proporcionadas & \begin{itemize}[nosep,leftmargin=*]
      \item \textbf{Configured endpoints}: Proporciona la aplicación de Express con las rutas y middlewares configurados.   
    \end{itemize} \\
\end{longtable}


%------------------------- PAQUETE MIDDLWARES -------------------------
\subsubsubsection{Descripción de componentes del subsistema restapi. Paquete \textit{middlewares}}\label{sec:descripcion_authmiddleware}
%--- AUTHMIDDLEWARE ---
\begin{longtable}{
    >{\columncolor{lightgreen!20}}p{4cm}
    p{12cm}
    }
    \caption{Descripción del componente:  AuthMiddleware} \label{table:descripcion_authmiddleware} \\
    \toprule
    \rowcolor{darkgreen!50}
    \textbf{Componente} & \multicolumn{1}{>{\columncolor{darkgreen!50}\centering\arraybackslash}p{12cm}}{\textbf{AUTHMIDDLEWARE}} \\
    \endfirsthead
    
    \multicolumn{2}{c}%
    {{ \tablename\ \thetable{} Descripción del componente:  AuthMiddleware -- continuación de la página anterior}} \\
    \toprule
    \rowcolor{darkgreen!50}
    \textbf{Componente} & \multicolumn{1}{>{\columncolor{darkgreen!50}\centering\arraybackslash}p{12cm}}{\textbf{AUTHMIDDLEWARE}} \\
    \midrule
    \endhead
    
    \midrule
    \multicolumn{2}{r}{{Continúa en la siguiente página...}} \\ 
    \endfoot
    
    \bottomrule
    \endlastfoot
    
    \midrule
    Descripción & Este componente proporciona middleware para la autenticación y autorización de usuarios mediante tokens JWT. Incluye la verificación de tokens y la verificación de roles de administrador. \\
    \midrule
    Métodos & \begin{itemize}[nosep,leftmargin=*]
      \item \textbf{auth(req: Request, res: Response, next: any)}: void, middleware para verificar la autenticidad del token JWT en las peticiones.
      \item \textbf{verifyAdmin(req: Request, res: Response, next: any)}: void, middleware para verificar que el usuario tiene rol de administrador.
    \end{itemize} \\
    \midrule
    Interfaces requeridas &  \\
    \midrule
    Interfaces proporcionadas & \begin{itemize}[nosep,leftmargin=*]
      \item \textbf{User Authentication}: Proporciona middleware para la autenticación de usuarios, verificando la validez del token JWT y, ofreciendo la posibilidad de verificar roles de administrador.
    \end{itemize} \\
    \end{longtable}

%--- AUTHSOCKET ---
\begin{longtable}{
    >{\columncolor{lightgreen!20}}p{4cm}
    p{12cm}
    }
    \caption{Descripción del componente:  AuthSocket} \label{table:descripcion_authsocket} \\
    \toprule
    \rowcolor{darkgreen!50}
    \textbf{Componente} & \multicolumn{1}{>{\columncolor{darkgreen!50}\centering\arraybackslash}p{12cm}}{\textbf{AUTHSOCKET}} \\
    \endfirsthead
    
    \multicolumn{2}{c}%
    {{ \tablename\ \thetable{} Descripción del componente:  AuthSocket -- continuación de la página anterior}} \\
    \toprule
    \rowcolor{darkgreen!50}
    \textbf{Componente} & \multicolumn{1}{>{\columncolor{darkgreen!50}\centering\arraybackslash}p{12cm}}{\textbf{AUTHSOCKET}} \\
    \midrule
    \endhead
    
    \midrule
    \multicolumn{2}{r}{{Continúa en la siguiente página...}} \\ 
    \endfoot
    
    \bottomrule
    \endlastfoot
    
    \midrule
    Descripción & Este componente proporciona el middleware para la autenticación de conexiones de sockets mediante tokens JWT. Verifica la validez y el formato del token proporcionado en el handshake de la conexión del socket. \\
    \midrule
    Métodos & \begin{itemize}[nosep,leftmargin=*]
      \item \textbf{authSocket(socket: Socket, next: (err?: Error) => void)}: void, middleware para verificar la autenticidad del token JWT en las conexiones de sockets.
    \end{itemize} \\
    \midrule
    Interfaces requeridas &  \\
    \midrule
    Interfaces proporcionadas & \begin{itemize}[nosep,leftmargin=*]
      \item \textbf{Socket Authentication}: Proporciona middleware para la autenticación de conexiones de sockets, verificando la validez del token JWT.
    \end{itemize} \\
    \end{longtable}

%------------------------- PAQUETE ROUTES -------------------------
\subsubsection{Descripción de componentes del subsistema restapi. Paquete \textit{routes}}
En el diagrama se muestra una relación de dependencia entre \textit{App} y los componentes del paquete \textit{routes}. 
En la práctica, cada componente del paquete \textit{routes} proporciona sus rutas a \textit{App} a través su propia instancia de \textit{Router} de Express. 
Se ha decidido simplificar la representación para facilitar la comprensión del diagrama, dado que todos los componentes del paquete \textit{routes} mantienen la misma relación con \textit{App}.

%--- AUCTIONROUTER ---
\begin{longtable}{
    >{\columncolor{lightgreen!20}}p{4cm}
    p{12cm}
    }
    \caption{Descripción del componente:  AuctionRouter} \label{table:descripcion_auctionrouter} \\
    \toprule
    \rowcolor{darkgreen!50}
    \textbf{Componente} & \multicolumn{1}{>{\columncolor{darkgreen!50}\centering\arraybackslash}p{12cm}}{\textbf{AUCTIONROUTER}} \\
    \endfirsthead
    
    \multicolumn{2}{c}%
    {{ \tablename\ \thetable{} Descripción del componente:  AuctionRouter -- continuación de la página anterior}} \\
    \toprule
    \rowcolor{darkgreen!50}
    \textbf{Componente} & \multicolumn{1}{>{\columncolor{darkgreen!50}\centering\arraybackslash}p{12cm}}{\textbf{AUCTIONROUTER}} \\
    \midrule
    \endhead
    
    \midrule
    \multicolumn{2}{r}{{Continúa en la siguiente página...}} \\ 
    \endfoot
    
    \bottomrule
    \endlastfoot
    
    \midrule
    Descripción & Este componente configura y gestiona las rutas relacionadas con las subastas en la aplicación Express. Incluye la autenticación mediante middleware y validaciones para las peticiones. \\
    \midrule
    Atributos & \begin{itemize}[nosep,leftmargin=*]
      \item \textbf{auctionRouter}: Router, instancia del enrutador de Express para las subastas.
    \end{itemize} \\
    \midrule
    Métodos & \begin{itemize}[nosep,leftmargin=*]
      \item \textbf{getAuctions(req: Request, res: Response)}: void, maneja la obtención de todas las subastas.
      \item \textbf{getAuction(req: Request, res: Response)}: void, maneja la obtención de una subasta por su ID.
      \item \textbf{getActiveAuctions(req: Request, res: Response)}: void, maneja la obtención de todas las subastas activas.
      \item \textbf{getActiveAuctionsByUser(req: Request, res: Response)}: void, maneja la obtención de todas las subastas activas de un usuario.
      \item \textbf{putUserCardUpForAuction(req: Request, res: Response)}: void, maneja la puesta en subasta de una carta de usuario.
      \item \textbf{withdrawnUserCardFromAuction(req: Request, res: Response)}: void, maneja la retirada de una carta de usuario de una subasta.
      \item \textbf{checkAllActiveAuctions(req: Request, res: Response)}: void, verifica todas las subastas activas y actualiza su estado si es necesario.
    \end{itemize} \\
    \midrule
    Interfaces requeridas &  \begin{itemize}[nosep,leftmargin=*]
      \item \textbf{User Authentication}: Middleware para la autenticación de usuarios.
      \item \textbf{Auction MGMT.}: Utiliza los métodos definidos en el controlador de subastas, \textit{AuctionController}.
    \end{itemize} \\
    \midrule
    Interfaces proporcionadas & \begin{itemize}[nosep,leftmargin=*]
      \item \textbf{Auction Router}: Proporciona las rutas para la gestión de subastas.
    \end{itemize} \\
    \end{longtable}

%--- BIDROUTER ---
\begin{longtable}{
    >{\columncolor{lightgreen!20}}p{4cm}
    p{12cm}
    }
    \caption{Descripción del componente:  BidRouter} \label{table:descripcion_bidrouter} \\
    \toprule
    \rowcolor{darkgreen!50}
    \textbf{Componente} & \multicolumn{1}{>{\columncolor{darkgreen!50}\centering\arraybackslash}p{12cm}}{\textbf{BIDROUTER}} \\
    \endfirsthead
    
    \multicolumn{2}{c}%
    {{ \tablename\ \thetable{} Descripción del componente:  BidRouter -- continuación de la página anterior}} \\
    \toprule
    \rowcolor{darkgreen!50}
    \textbf{Componente} & \multicolumn{1}{>{\columncolor{darkgreen!50}\centering\arraybackslash}p{12cm}}{\textbf{BIDROUTER}} \\
    \midrule
    \endhead
    
    \midrule
    \multicolumn{2}{r}{{Continúa en la siguiente página...}} \\ 
    \endfoot
    
    \bottomrule
    \endlastfoot
    
    \midrule
    Descripción & Este componente configura y gestiona las rutas relacionadas con las pujas en la aplicación Express. Incluye la autenticación mediante middleware y validaciones para las peticiones. \\
    \midrule
    Atributos & \begin{itemize}[nosep,leftmargin=*]
      \item \textbf{bidRouter}: Router, instancia del enrutador de Express para las pujas.
    \end{itemize} \\
    \midrule
    Métodos & \begin{itemize}[nosep,leftmargin=*]
      \item \textbf{getBidById(req: Request, res: Response)}: void, maneja la obtención de una puja por su ID.
      \item \textbf{createBid(req: Request, res: Response)}: void, maneja la creación de una nueva puja.
      \item \textbf{getActiveBidsByUser(req: Request, res: Response)}: void, maneja la obtención de todas las pujas activas de un usuario.
      \item \textbf{withdrawBid(req: Request, res: Response)}: void, maneja la retirada de una puja.
    \end{itemize} \\
    \midrule
    Relaciones & \begin{itemize}[nosep,leftmargin=*]
      \item \textbf{AuthMiddleware}: Middleware para la autenticación de usuarios.
      \item \textbf{BidController}: Importa y utiliza métodos del controlador de pujas.
    \end{itemize} \\
    \midrule
    Interfaces requeridas & \begin{itemize}[nosep,leftmargin=*]
      \item \textbf{User Authentication}: Middleware para la autenticación de usuarios.
      \item \textbf{Bid MGMT.}: Utiliza los métodos definidos en el controlador de pujas, \textit{BidController}.
    \end{itemize} \\
    \midrule
    Interfaces proporcionadas & \begin{itemize}[nosep,leftmargin=*]
      \item \textbf{Bid Router}: Proporciona las rutas para la gestión de pujas.
    \end{itemize} \\
    \end{longtable}

\subsubsubsection{Descripción del componente:  CardPackRouter} \label{sec:descripcion_cardpackrouter}
\begin{longtable}{
    >{\columncolor{lightgreen!20}}p{4cm}
    p{12cm}
    }
    \caption{Descripción del componente:  CardPackRouter} \label{table:descripcion_cardpackrouter} \\
    \toprule
    \rowcolor{darkgreen!50}
    \textbf{Componente} & \multicolumn{1}{>{\columncolor{darkgreen!50}\centering\arraybackslash}p{12cm}}{\textbf{CARDPACKROUTER}} \\
    \endfirsthead
    
    \multicolumn{2}{c}%
    {{ \tablename\ \thetable{} Descripción del componente:  CardPackRouter -- continuación de la página anterior}} \\
    \toprule
    \rowcolor{darkgreen!50}
    \textbf{Componente} & \multicolumn{1}{>{\columncolor{darkgreen!50}\centering\arraybackslash}p{12cm}}{\textbf{CARDPACKROUTER}} \\
    \midrule
    \endhead
    
    \midrule
    \multicolumn{2}{r}{{Continúa en la siguiente página...}} \\ 
    \endfoot
    
    \bottomrule
    \endlastfoot
    
    \midrule
    Descripción & Este componente configura y gestiona las rutas relacionadas con los sobres de cartas en la aplicación Express. Incluye la autenticación mediante middleware. \\
    \midrule
    Métodos & \begin{itemize}[nosep,leftmargin=*]
      \item \textbf{getCardPacks(req: Request, res: Response)}: void, maneja la obtención de todos los sobres de cartas.
    \end{itemize} \\
    \midrule
    Interfaces requeridas & \begin{itemize}[nosep,leftmargin=*]
      \item \textbf{User Authentication}: Middleware para la autenticación de usuarios.
      \item \textbf{CardPack MGMT.}: Utiliza los métodos definidos en el controlador de sobres de cartas, \textit{CardPackController}.
    \end{itemize} \\
    \midrule
    Interfaces proporcionadas & \begin{itemize}[nosep,leftmargin=*]
      \item \textbf{CardPack Router}: Proporciona las rutas para la gestión de sobres de cartas.
    \end{itemize} \\
    \end{longtable}


%--- CARDROUTER ---
\begin{longtable}{
    >{\columncolor{lightgreen!20}}p{4cm}
    p{12cm}
    }
    \caption{Descripción del componente:  CardRouter} \label{table:descripcion_cardrouter} \\
    \toprule
    \rowcolor{darkgreen!50}
    \textbf{Componente} & \multicolumn{1}{>{\columncolor{darkgreen!50}\centering\arraybackslash}p{12cm}}{\textbf{CARDROUTER}} \\
    \endfirsthead
    
    \multicolumn{2}{c}%
    {{ \tablename\ \thetable{} Descripción del componente:  CardRouter -- continuación de la página anterior}} \\
    \toprule
    \rowcolor{darkgreen!50}
    \textbf{Componente} & \multicolumn{1}{>{\columncolor{darkgreen!50}\centering\arraybackslash}p{12cm}}{\textbf{CARDROUTER}} \\
    \midrule
    \endhead
    
    \midrule
    \multicolumn{2}{r}{{Continúa en la siguiente página...}} \\ 
    \endfoot
    
    \bottomrule
    \endlastfoot
    
    \midrule
    Descripción & Esta clase configura y gestiona las rutas relacionadas con las cartas en la aplicación Express. Incluye la autenticación mediante middleware y validaciones para las peticiones. \\
    \midrule
    Métodos & \begin{itemize}[nosep,leftmargin=*]
      \item \textbf{getCard(req: Request, res: Response)}: void, maneja la obtención de una carta por su ID.
    \end{itemize} \\
    \midrule
    Interfaces requeridas  & \begin{itemize}[nosep,leftmargin=*]
      \item \textbf{User Authentication}: Middleware para la autenticación de usuarios.
      \item \textbf{Card MGMT.}: Utiliza los métodos definidos en el controlador de cartas, \textit{CardController}.
    \end{itemize} \\
    \midrule
    Interfaces proporcionadas & \begin{itemize}[nosep,leftmargin=*]
        \item \textbf{Card Router}: Proporciona las rutas para la gestión de cartas.
    \end{itemize} \\
    \end{longtable}

% --- NOTIFICATIONROUTER ---
\begin{longtable}{
    >{\columncolor{lightgreen!20}}p{4cm}
    p{12cm}
    }
    \caption{Descripción del componente:  NotificationRouter} \label{table:descripcion_notificationrouter} \\
    \toprule
    \rowcolor{darkgreen!50}
    \textbf{Componente} & \multicolumn{1}{>{\columncolor{darkgreen!50}\centering\arraybackslash}p{12cm}}{\textbf{NOTIFICATIONROUTER}} \\
    \endfirsthead
    
    \multicolumn{2}{c}%
    {{ \tablename\ \thetable{} Descripción del componente:  NotificationRouter -- continuación de la página anterior}} \\
    \toprule
    \rowcolor{darkgreen!50}
    \textbf{Componente} & \multicolumn{1}{>{\columncolor{darkgreen!50}\centering\arraybackslash}p{12cm}}{\textbf{NOTIFICATIONROUTER}} \\
    \midrule
    \endhead
    
    \midrule
    \multicolumn{2}{r}{{Continúa en la siguiente página...}} \\ 
    \endfoot
    
    \bottomrule
    \endlastfoot
    
    \midrule
    Descripción & Este componente configura y gestiona las rutas relacionadas con las notificaciones en la aplicación Express. Incluye la autenticación mediante middleware y validaciones para las peticiones. \\
    \midrule
    Atributos & \begin{itemize}[nosep,leftmargin=*]
      \item \textbf{notificationRouter}: Router, instancia del enrutador de Express para las notificaciones.
    \end{itemize} \\
    \midrule
    Métodos & \begin{itemize}[nosep,leftmargin=*]
      \item \textbf{getNotifications(req: Request, res: Response)}: void, maneja la obtención de todas las notificaciones de un usuario.
      \item \textbf{markAsRead(req: Request, res: Response)}: void, maneja el marcado de una notificación como leída.
      \item \textbf{markAllAsRead(req: Request, res: Response)}: void, maneja el marcado de todas las notificaciones de un usuario como leídas.
      \item \textbf{hasUnreadNotifications(req: Request, res: Response)}: void, verifica si un usuario tiene notificaciones no leídas.
    \end{itemize} \\
    \midrule
    Interfaces requeridas & \begin{itemize}[nosep,leftmargin=*]
      \item \textbf{User Authentication}: Middleware para la autenticación de usuarios.
      \item \textbf{Notification MGMT.}: Utiliza los métodos definidos en el controlador de notificaciones, \textit{NotificationController}.
    \end{itemize} \\
    \midrule
    Interfaces proporcionadas & \begin{itemize}[nosep,leftmargin=*]
        \item \textbf{Notification Router}: Proporciona las rutas para la gestión de notificaciones.
    \end{itemize} \\
    \end{longtable}

%--- PAYPALROUTER ---
\begin{longtable}{
    >{\columncolor{lightgreen!20}}p{4cm}
    p{12cm}
    }
    \caption{Descripción del componente:  PaypalRouter} \label{table:descripcion_paypalrouter} \\
    \toprule
    \rowcolor{darkgreen!50}
    \textbf{Componente} & \multicolumn{1}{>{\columncolor{darkgreen!50}\centering\arraybackslash}p{12cm}}{\textbf{PAYPALROUTER}} \\
    \endfirsthead
    
    \multicolumn{2}{c}%
    {{ \tablename\ \thetable{} Descripción del componente:  PaypalRouter -- continuación de la página anterior}} \\
    \toprule
    \rowcolor{darkgreen!50}
    \textbf{Componente} & \multicolumn{1}{>{\columncolor{darkgreen!50}\centering\arraybackslash}p{12cm}}{\textbf{PAYPALROUTER}} \\
    \midrule
    \endhead
    
    \midrule
    \multicolumn{2}{r}{{Continúa en la siguiente página...}} \\ 
    \endfoot
    
    \bottomrule
    \endlastfoot
    
    \midrule
    Descripción & Este componente configura y gestiona las rutas relacionadas con las órdenes de PayPal en la aplicación Express. Incluye validaciones para las peticiones y manejo de errores. \\
    \midrule
    Métodos & \begin{itemize}[nosep,leftmargin=*]
      \item \textbf{createOrder(req: Request, res: Response)}: void, maneja la creación de una nueva orden de PayPal.
      \item \textbf{updateOrder(req: Request, res: Response)}: void, maneja la actualización del saldo de un usuario después de completar un pago.
    \end{itemize} \\
    \midrule
    Interfaces requeridas & \begin{itemize}[nosep,leftmargin=*]
      \item \textbf{Paypal MGMT.}: Utiliza los métodos definidos en el controlador de PayPal, \textit{PaypalController}.
    \end{itemize} \\
    \midrule
    Interfaces proporcionadas & \begin{itemize}[nosep,leftmargin=*]
      \item \textbf{Paypal Router}: Proporciona las rutas para la gestión de órdenes de PayPal.
    \end{itemize} \\
    \end{longtable}

%------------------------- PAQUETE CONTROLLERS -------------------------
