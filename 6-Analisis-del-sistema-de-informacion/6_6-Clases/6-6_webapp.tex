\subsubsection{Descripción de los Componentes del Subsistema webapp}

Este apartado expone detalladamente los componentes del subsistema webapp, detallando su funcionalidad y su interrelación dentro de la arquitectura de la aplicación.

\paragraph{Componente \textbf{public}}
Este componente almacena los archivos estáticos esenciales para la interfaz de usuario, incluyendo recursos gráficos y el archivo \textit{index.html}. Este último establece la estructura fundamental de la aplicación, configurando el entorno inicial en el que se cargarán los demás componentes dinámicos.

\paragraph{Paquete \textbf{src}}
El núcleo de la lógica de la interfaz de usuario se gestiona dentro de este paquete, donde cada componente tiene una función específica:

\begin{itemize}
    \item \textbf{Index}: Actúa como el punto de entrada de la aplicación, encargado de inicializar el componente principal \textit{App} y de configurar el entorno de ejecución.
    \item \textbf{App}: Es el componente central que gestiona la navegabilidad y el diagrama de flujo de la aplicación. Coordina el renderizado de diferentes componentes basado en la ruta actual y establece conexiones con los paquetes \textit{pages} y \textit{Theme}.
    \item \textbf{Theme}: Define y aplica la configuración visual global de la aplicación, incluyendo esquemas de color y tipografías, para mantener una coherencia estética en toda la interfaz de usuario.
    \item \textbf{API}: Administra todas las interacciones con las APIs, permitiendo la comunicación y el intercambio de datos entre el frontend y el backend.
    \item \textbf{Redux}: Es el gestor de estado de la aplicación, manteniendo y actualizando los estados globales como:
    \begin{itemize}
        \item \textbf{user}: Almacena y gestiona la información del usuario autenticado.
        \item \textbf{notification}: Controla la presentación y gestión de notificaciones en tiempo real.
        \item \textbf{update}: Se encarga de las actualizaciones periódicas de los datos de la aplicación.
    \end{itemize}
    \item \textbf{Socket}: Gestiona la comunicación en tiempo real con el servidor a través de Socket.io. Define la funcionalidad para gestionar las notificaciones en tiempo real.
    \item \textbf{Shared}: Define los tipos de datos y estructuras compartidas utilizadas a lo largo de la aplicación para asegurar la consistencia y la eficiencia en el manejo de la información.
    \item \textbf{Utils}: Provee una serie de utilidades y herramientas que facilitan operaciones comunes dentro de la aplicación, tales como validación de formularios, gestión de rutas privadas y funciones de ayuda general.
    \item \textbf{Views}: Incluye la implementación de las interfaces de usuario. Se divide en \textit{components} y \textit{pages} para organizar los elementos visuales de manera coherente y modular.

    \begin{itemize}
        \item \textbf{Components}: Este subpaquete agrupa los componentes reutilizables que conforman la base de las interfaces de usuario en la aplicación:

\begin{itemize}
    \item \textbf{Button}: Implementa diferentes tipos de botones que se utilizan a través de la aplicación, adaptándose a diversas funciones y estilos según las necesidades de la interfaz.
    \item \textbf{Calendar}: Proporciona una representación visual de un calendario para la selección de fechas.
    \item \textbf{Card}: Define una estructura base para las cartas coleccionables, utilizadas en diferentes contextos de la aplicación.
    \item \textbf{CardDetail}: Muestra información detallada de una carta.
    \item \textbf{Cardpack}: Representa sobres de cartas, es el componente principal para la compra de cartas en la tienda.
    \item \textbf{Container}: Generaliza contenedores que encapsulan diferentes tipos de contenido.
    \item \textbf{Form}: Define los distintos tipos de formularios utilizados en la aplicación, con validación y gestión de eventos integrados.
    \item \textbf{Footer}: Define el pie de página de la aplicación, conteniendo enlaces e información relevante de contacto.
    \item \textbf{Header}: Encabeza las páginas de la aplicación, incluyendo elementos de navegación y acceso al perfil del usuario.
    \item \textbf{LogoBox}: Presenta el logotipo de la aplicación, utilizado generalmente en la cabecera de los formularios de inicio de sesión y registro.
    \item \textbf{Menu}: Organiza las opciones de navegación disponibles para los usuarios, adaptándose a los diferentes roles y estados de autenticación.
    \item \textbf{Messages}: Administra la presentación de mensajes informativos o de error, mejorando la interactividad y experiencia del usuario.
    \item \textbf{Ornament}: Añade elementos decorativos que mejoran la estética sin cargar la funcionalidad principal de los componentes.
    \item \textbf{Paper}: Ofrece un componente estilizado para mostrar contenido en forma de tarjeta elevada, utilizada para destacar información relevante.
    \item \textbf{Switch}: Permite a los usuarios alternar entre el modo claro y oscuro de la aplicación, adaptándose a sus preferencias visuales.
    \item \textbf{Table}: Proporciona estructuras de tabla para la visualización de datos complejos, con funciones de ordenación, paginación y filtrado integradas.
\end{itemize}
        \item \textbf{Pages}: Detalla las implementaciones específicas de las vistas de la aplicación, que se construyen utilizando los componentes anteriores. 
        Cada página está diseñada para satisfacer las necesidades de diferentes tipos de usuarios y situaciones:

\begin{itemize}
    \item \textbf{BasePage}: Ofrece una plantilla base para las páginas de la aplicación, garantizando la uniformidad y coherencia en el diseño.
    \item \textbf{BasePageWithNav}: Similar a BasePage, pero incluye componentes de navegación específicos para facilitar el acceso a otras secciones.
    \item \textbf{NotFoundPage}: Proporciona una respuesta visual para rutas no encontradas, mejorando la navegación del usuario en casos de error.
    \item El subpaquete \textbf{guest} contiene páginas accesibles sin necesidad de autenticación:
    \begin{itemize}
        \item \textbf{Home}: La página principal de la aplicación, presentando información general y acceso al inicio de sesión o registro.
        \item \textbf{Login}: Facilita el inicio de sesión para los usuarios.
        \item \textbf{SignUp}: Permite a nuevos usuarios registrarse en la aplicación.
        \item \textbf{About}: Ofrece información sobre la aplicación y sus creadores.
    \end{itemize}
    \item El subpaquete \textbf{admin} agrupa las páginas destinadas a usuarios con roles de administrador:
    \begin{itemize}
        \item \textbf{AdminPage}: Página principal del panel de administración.
        \item \textbf{AdminTransactions}: Permite consultar todas las transacciones realizadas en la aplicación por los usuarios.
        \item \textbf{AdminAuctionDetail}: Detalla las subastas activas y permite su gestión directa.
        \item \textbf{AuctionsAdmin}: Permite a los administradores consultar y gestionar las subastas activas.
    \end{itemize}
    \item El subpaquete \textbf{standard}: Para usuarios autenticados sin privilegios administrativos, proporciona acceso a funcionalidades estándar de la aplicación:
    \begin{itemize}
        \item \textbf{Logueado}: Muestra la interfaz principal para usuarios autenticados, ofreciendo un resumen de su actividad y acceso directo a funcionalidades comunes.
        \item \textbf{EditProfile}: Permite a los usuarios modificar su perfil.
        \item \textbf{CardDetail}: Muestra la información de una carta de la colección del usuario.
        \item \textbf{AuctionCardDetail}: Presenta información detallada de una carta en subasta.
        \item \textbf{BidCardDetail}: Muestra el detalle de una puja realizada por el usuario.
        \item \textbf{Shop}: Tienda virtual donde los usuarios pueden adquirir sobres de cartas para ampliar su colección.
        \item \textbf{ActiveAuctions}: Lista las subastas en curso, permitiendo a los usuarios participar en ellas.
        \item \textbf{UserTransactions}: Resumen de las transacciones realizadas por el usuario, ofreciendo un historial detallado.
        \item \textbf{InBox}: Gestiona las notificaciones recibidas, asegurando que el usuario esté informado de eventos importantes.
        \item \textbf{MyCollection}: Permite al usuario visualizar y gestionar su colección de cartas.
        \item \textbf{RechargeBalance}: Integra métodos para recargar el saldo del usuario, enlazando con sistemas de pago externos como PayPal.
    \end{itemize}
\end{itemize}

    \end{itemize}
\end{itemize}

\bigskip
\textbf{Nota:} Cada componente en el diagrama puede representar uno o más elementos funcionales dentro de la aplicación. 
Por ejemplo, en el caso del componente \textit{Button}, en la práctica se implementan varios componentes que se adaptan para cubrir distintas funcionalidades en el sistema.


