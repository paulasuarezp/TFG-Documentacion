\subsubsubsection{Descripción de los Componentes del Subsistema webapp}
En este apartado se describirá la funcionalidad de los componentes del subsistema webapp. 
El componente \textbf{public} contiene los archivos estáticos de la interfaz de usuario, como las imágenes y el archivo \textit{index.html} que define la estructura base de la aplicación.
Por otro lado el paquete \textbf{src} contiene los archivos que implementan la lógica de la interfaz de usuario, detallados a continuación:
\begin{itemize}
    \item \textbf{Index}: Es el punto de entrada de la aplicación. Este componente se encarga del renderizado del componente \textit{App}.
    \item \textbf{App}: es el componente principal de la aplicación. Este componente define el diagrama de navegabilidad de la aplicación y se encarga de renderizar los componentes de la aplicación en función de la ruta actual. 
    Se relaciona, por lo tanto, con el paquete \textit{pages} y con el componente \textit{Theme} para definir el tema de la aplicación.
    \item \textbf{Theme}: es el componente que define el tema de la aplicación. Este componente se encarga de definir los colores y tipografías de la aplicación.
    \item \textbf{API}: es el componente que se encarga de realizar las peticiones a la API. Este componente se relaciona con el paquete \textit{shared} para definir los tipos de datos compartidos entre las distintas partes de la aplicación.
    \item \textbf{Redux}: es el componente que se encarga de gestionar el estado de la aplicación. Gestiona los siguientes estados:
    \begin{itemize}
        \item \textbf{user}: contiene la información del usuario autenticado.
        \item \textbf{notification}: estado para gestionar las visualizaciones de las notificaciones en tiempo real.
        \item \textbf{update}: estado para gestionar las actualizaciones de los datos de la aplicación.
    \end{itemize}
    \item \textbf{Socket}: es el componente que se encarga de gestionar la conexión con Socket.io. Concretamente se utiliza para que los usuarios puedan recibir notificaciones en tiempo real.
    \item \textbf{Shared}: es el componente que se encarga de definir los tipos de datos compartidos entre las distintas partes de la aplicación.
    \item \textbf{Utils}: es el componente que se encarga de definir utilidades de la aplicación. Contiene las siguientes utilidades:
    \begin{itemize}
        \item \textbf{cardData}: Colección de cartas de ejemplo para la aplicación. Se utiliza en un componente de la aplicación para mostrar cartas de ejemplo.
        \item \textbf{fieldsValidation}: utilidad para validar los campos de un formulario.
        \item \textbf{PrivateRoute}: componente que implementa una ruta privada de la aplicación. Verifica que un usuario esté autenticado antes de renderizar un componente.
        \item \textbf{RouteRedirector}: componente que implementa un redireccionador de rutas de la aplicación. Redirige a la ruta solicitada si el usuario cumple con el rol requerido, en caso contrario redirige a la ruta por defecto.
        \item \textbf{utils}: funciones de utilidad para la aplicación, como conversión de fechas y generación de mensajes.
    \end{itemize}
    \item \textbf{views}: contiene los componentes que implementan las vistas de la aplicación, y se divide en los siguientes subpaquetes:
    \begin{itemize}
        \item \textbf{components}: contiene los archivos que implementan los componentes de la aplicación. 
        Estos se usan para definir componentes más complejos que se reutilizan en distintas partes de la aplicación.
        \begin{itemize}
            \item \textbf{Button}: componente que implementa un botón de la aplicación. En este componente se definen los distintos tipos de botones que se utilizan en la aplicación.
            \item \textbf{Calendar}: componente que implementa un calendario de la aplicación.
            \item \textbf{Card}: componente que implementa una carta de la aplicación. 
            \item \textbf{CardDetail}: componente que implementa el modelo base de un detalle de carta.
            \item \textbf{Cardpack}: componente que implementa un sobre de cartas de la aplicación.
            \item \textbf{Container}: componente que implementa distintos tipos de contenedores de la aplicación.
            \item \textbf{Form}: componente que define los formularios de la aplicación.
            \item \textbf{Footer}: componente que implementa el pie de página de la aplicación.
            \item \textbf{Header}: componente que implementa la cabecera de la aplicación.
            \item \textbf{LogoBox}: componente que implementa el logotipo de la aplicación.
            \item \textbf{Menu}: componente que implementa los distintos menús de la aplicación.
            \item \textbf{Messages}: componente que implementa los distintos mensajes informativos de la aplicación.
            \item \textbf{Ornament}: componente que implementa distintos adornos de la aplicación.
            \item \textbf{Paper}: componente que implementa un contenedor con sombra, se utiliza para mostrar información en la aplicación.
            \item \textbf{Switch}: componente que implementa un interruptor de la aplicación. Concretamente, se utiliza para cambiar entre los modos claro y oscuro de la temática de la aplicación.
            \item \textbf{Table}: componente que implementa una tabla de la aplicación.
        \end{itemize}
        \item \textbf{pages}: contiene los archivos que implementan las páginas de la aplicación. Las páginas definen la estructura de las distintas vistas de la aplicación.
        Estas se crean utilizando los componentes definidos en el paquete \textit{components} junto con otros componentes de React.
        \begin{itemize}
            \item \textbf{BasePage}: página que implementa la estructura base de una página de la aplicación.
            \item \textbf{BasePageWithNav}: página que implementa la estructura base de una página de la aplicación con navegación.
            \item \textbf{NotFoundPage}: página que implementa la página de error 404.
            \item En el paquete \textit{guest} se encuentran las páginas que puede ver cualquier usuario:
            \begin{itemize}
                \item \textbf{Home}: página principal de la aplicación.
                \item \textbf{Login}: página de inicio de sesión.
                \item \textbf{SignUp}: página de registro de usuario.
                \item \textbf{About}: página de información sobre la aplicación.
            \end{itemize}
             \item En el paquete \textit{admin} se encuentran las páginas que puede ver un usuario con rol de administrador:
            \begin{itemize}
                \item \textbf{AdminPage}: página de administración de la aplicación.
                \item \textbf{AdminTransactions}: página de administración de usuarios.
                \item \textbf{AdminAuctionDetail}: página de administración de transacciones.
                \item \textbf{AuctionsAdmin}: página de administración de subastas.
            \end{itemize}
            \item En el paquete \textit{standard} se encuentran las páginas que puede ver un usuario autenticado con rol de usuario no administrador:
            \begin{itemize}
                \item \textbf{Logueado}: página de inicio del usuario autenticado.
                \item \textbf{EditProfile}: página de perfil del usuario autenticado.
                \item \textbf{CardDetail}: página de detalle de carta.
                \item \textbf{AuctionCardDetail}: página que muestra el detalle de una carta en subasta.
                \item \textbf{BidCardDetail}: página que muestra el detalle de una puja realizada a una determinada carta.
                \item \textbf{Shop}: página de tienda, en ella el usuario puede adquirir sobres de cartas.
                \item \textbf{ActiveAuctions}: página de subastas, en ella el usuario puede participar en subastas.
                \item \textbf{UserTransactions}: página de transacciones, en ella el usuario puede consultar las subastas realizadas.
                \item \textbf{InBox}: página de notificaciones, en ella el usuario puede consultar las notificaciones recibidas.
                \item \textbf{MyCollection}: página de colección, en ella el usuario puede consultar las cartas que posee.
                \item \textbf{RechargeBalance}: página de recarga de saldo, en ella el usuario puede recargar su saldo. Se comunica por HTTPS con la pasarela de pago, en este caso con la API de PayPal.
            \end{itemize}
        \end{itemize}
    \end{itemize}
\end{itemize}

% Nota adicional sobre los componentes
\bigskip
\textbf{Nota:} En el diagrama de componentes, una caja no necesariamente representa un único componente; puede representar múltiples componentes. 
Por ejemplo, en el caso del componente \textit{Button}, en la práctica se han implementado varios componentes que representan distintos tipos de botones de la aplicación.
