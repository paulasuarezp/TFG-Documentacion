
\newlist{RFGestionUsuarios}{enumerate}{5}
\setlist[RFGestionUsuarios,1]{label=\textbf{RGU-\arabic*.}, leftmargin=*, align=left, font=\fontsize{10pt}{11pt}\selectfont}
\setlist[RFGestionUsuarios,2]{label*=\textbf{\arabic*.},font=\fontsize{10pt}{11pt}\selectfont}
\setlist[RFGestionUsuarios,3]{label*=\textbf{\arabic*.},font=\fontsize{9pt}{11pt}\selectfont}
\setlist[RFGestionUsuarios,4]{label*=\textbf{\arabic*.},font=\fontsize{8pt}{11pt}\selectfont}
\setlist[RFGestionUsuarios,5]{label*=\textbf{\arabic*.},font=\fontsize{8pt}{11pt}\selectfont}


\subsubsection*{Registro e inicio de sesión}
\begin{RFGestionUsuarios}
  \item Un usuario deberá poder registrarse en el sistema.\label{req_registro} %RF-1
  \begin{RFGestionUsuarios}
    
    \item Un usuario deberá registrarse con unas credenciales pertenecientes a una cuenta de Google.
    \begin{RFGestionUsuarios}
      \item El sistema no permitirá que un usuario se registre más de una vez.
      \item El sistema pedirá la confirmación de los credenciales a Google.
      \begin{RFGestionUsuarios}
      	\item En caso de que los credenciales sean correctos, se accederá a la aplicación por primera vez.
      	\item En caso de que los credenciales sean erróneos, se informará al usuario de dicho error.
       \end{RFGestionUsuarios}
    \end{RFGestionUsuarios}
    
    \item Un usuario deberá proporcionar el número de teléfono correspondiente al dispositivo que esté utilizando.
    \begin{RFGestionUsuarios}
    	\item El sistema comprobará que el número de teléfono corresponde al dispositivo con el que se esté realizando el registro.
    	\begin{RFGestionUsuarios}
     		\item El sistema emitirá un mensaje SMS a dicho dispositivo que se detectará automáticamente.
      		\item El sistema validará el código de SMS recibido con el de la base de datos.
    	\end{RFGestionUsuarios}
   	\end{RFGestionUsuarios} 
   	
   	\item Una vez finalizado el registro, el sistema automáticamente:
   		\begin{RFGestionUsuarios}
   		\item Escribirá los datos recibidos del registro en la base de datos:
    		\begin{RFGestionUsuarios}
      			\item Nombre
      			\item Apellidos
      			\item Email
      			\item Número de teléfono
      			\item UID de Google
      			\item URL de la imagen de perfil
      			\item Locale
      			\item Fecha de creación
      			\item Fecha de último inicio de sesión
    		\end{RFGestionUsuarios}
    		
    	\item Redirigirá al usuario a la pantalla principal.
    	\end{RFGestionUsuarios}
  \end{RFGestionUsuarios}
  
  \item Un usuario deberá poder iniciar sesión en el sistema. %RF-2
    \begin{RFGestionUsuarios}%RF2.1
      \item Un usuario deberá iniciar sesión mediante las credenciales de Google.
      \begin{RFGestionUsuarios}%RF2.1.1
      	\item El sistema comprobará que las credenciales sean correctas.
      	\item El sistema verificará que el usuario existe en la base de datos.
      		\begin{RFGestionUsuarios}%RF2.1.1.1
      			\item En caso de no existir, el sistema automáticamente:
      			\begin{RFGestionUsuarios}%RF2.1.1.1.1
      				\item Registrará al usuario siguiendo \textit{\ref{req_registro}}.
      				\item Notificará a usuario.
      			\end{RFGestionUsuarios}
      			\item En caso de existir, el sistema automáticamente actualizará la información del usuario en la base de datos:
      			\begin{RFGestionUsuarios}
      				\item Fecha de último inicio de sesión
      			\end{RFGestionUsuarios}
      		\end{RFGestionUsuarios}
      \end{RFGestionUsuarios}
      
      \item Una vez finalizado el inicio de sesión, el sistema automáticamente redirigirá al usuario a la pantalla principal.
    \end{RFGestionUsuarios}
\end{RFGestionUsuarios}