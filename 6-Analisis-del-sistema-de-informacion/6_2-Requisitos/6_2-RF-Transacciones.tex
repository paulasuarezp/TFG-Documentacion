\newlist{RFTransacciones}{enumerate}{5}
\setlist[RFTransacciones,1]{label=\textbf{RT-\arabic*.}, leftmargin=*, align=left, font=\fontsize{10pt}{11pt}\selectfont}
\setlist[RFTransacciones,2]{label*=\textbf{\arabic*.},font=\fontsize{10pt}{11pt}\selectfont}
\setlist[RFTransacciones,3]{label*=\textbf{\arabic*.},font=\fontsize{9pt}{11pt}\selectfont}
\setlist[RFTransacciones,4]{label*=\textbf{\arabic*.},font=\fontsize{8pt}{11pt}\selectfont}
\setlist[RFTransacciones,5]{label*=\textbf{\arabic*.},font=\fontsize{8pt}{11pt}\selectfont}


\subsubsubsection{Gestión de transacciones}
\begin{RFTransacciones}
	\item El sistema deberá de registrar todos los movimientos de monedas realizados por los usuarios.
	\item Cada transacción en el sistema deberá de poder identicarse de forma inequívoca. \hypertarget{RT-2}{}
	\begin{RFTransacciones}
		\item Deberá tener un identificador único.
		\item Deberá tener un identificador de usuario.
		\item Deberá tener un concepto explicativo.
		\item Deberá tener una fecha y hora de realización.
		\item Deberá tener un importe.
		\item Deberá de tener una referencia al activo afectado.
	\end{RFTransacciones}
	\item El sistema deberá de permitir a los usuarios autenticados consultar sus transacciones.
	\begin{RFTransacciones}
		\item El sistema mostrará los datos mencionados en el punto \coloredUnderline{\hyperlink{RT-2}{RT-2}}.
	\end{RFTransacciones}
	\item Un usuario con rol de administrador deberá de poder consultar las transacciones de todos los usuarios.	
\end{RFTransacciones}