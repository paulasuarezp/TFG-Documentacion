
\newlist{RFColeccionCartas}{enumerate}{5}
\setlist[RFColeccionCartas,1]{label=\textbf{RCC-\arabic*.}, leftmargin=*, align=left, font=\fontsize{10pt}{11pt}\selectfont}
\setlist[RFColeccionCartas,2]{label*=\textbf{\arabic*.},font=\fontsize{10pt}{11pt}\selectfont}
\setlist[RFColeccionCartas,3]{label*=\textbf{\arabic*.},font=\fontsize{9pt}{11pt}\selectfont}
\setlist[RFColeccionCartas,4]{label*=\textbf{\arabic*.},font=\fontsize{8pt}{11pt}\selectfont}
\setlist[RFColeccionCartas,5]{label*=\textbf{\arabic*.},font=\fontsize{8pt}{11pt}\selectfont}


\subsubsubsection{Colección de cartas}
\begin{RFColeccionCartas}
	\item El sistema tendrá una colección de cartas disponibles para los usuarios.
	\begin{RFColeccionCartas}
		\item Las cartas serán representaciones de pokémon.
		\item Cada carta cuenta con los siguientes campos:
		\begin{RFColeccionCartas}
			\item Un identificador único (UID), generado por la base de datos.
			\item Un identificador de colección (IDC) en formato \coloredUnderline{\hyperlink{confParam:cc-formatoIDCCarta}{CC\_FORMATO\_IDC\_CARTA}}, generado por el sistema.
			\item Nombre del pokémon.
			\item Rareza de la carta.
			\item Fecha de lanzamiento.
			\item Cantidad disponible.
			\item Identificadores de las cartas vendidas.
			\item Tipo del pokémon.
			\item Descripción del pokémon.
			\item Imagen del pokémon.
			\item HP del pokémon.
			\item Ataque del pokémon.
			\item Defensa del pokémon.
			\item Velocidad del pokémon.
			\item Peso del pokémon.
			\item Altura del pokémon.
			\item Valor que indica si el pokémon es legendario.
			\item Valor que indica si el pokémon es mítico.
			\item Gimnasio al que pertenece el pokémon, si es que pertenece a alguno.
			\item Número de áreas en las que se puede encontrar el pokémon.
			\item Número de encuentros.
			\item Media de posibilidad de captura.
		\end{RFColeccionCartas}
		\item La rareza de las cartas se calcula en función de:
		\begin{RFColeccionCartas}
			\item El pokémon que representan.
			\item Rareza del pokémon.
			\item Media de posibilidad de captura.
			\item Valor aleatorio.
		\end{RFColeccionCartas}
	\end{RFColeccionCartas}
	\item Los usuarios autenticados podrán coleccionar cartas.\label{req_coleccion_cartas} \hypertarget{req_coleccion_cartas}{}
	\begin{RFColeccionCartas}
		\item El usuario podrá visualizar las cartas que posee en su colección.
		\item El usuario podrá consultar la información de cada carta.
		\item El usuario podrá añadir cartas a su colección mediante:
		\begin{RFColeccionCartas}
			\item La compra de sobres (ver \coloredUnderline{\hyperlink{req_sobres}{Gestión de sobres}}).
			\item La compra de cartas individuales mediante subastas (ver \coloredUnderline{\hyperlink{req_subastas_pujas}{Gestión de subastas y pujas}}).
		\end{RFColeccionCartas}
		\item El usuario podrá consultar los movimientos de cartas, es decir, las transacciones de las cartas de su colección.
		\item El usuario podrá subastar cartas de su colección (ver \coloredUnderline{\hyperlink{req_subastas_pujas}{Gestión de subastas y pujas}})
	\end{RFColeccionCartas}
\end{RFColeccionCartas}