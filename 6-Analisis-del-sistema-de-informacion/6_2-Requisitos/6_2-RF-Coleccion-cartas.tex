
\newlist{RFColeccionCartas}{enumerate}{5}
\setlist[RFColeccionCartas,1]{label=\textbf{RCC-\arabic*.}, leftmargin=*, align=left, font=\fontsize{10pt}{11pt}\selectfont}
\setlist[RFColeccionCartas,2]{label*=\textbf{\arabic*.},font=\fontsize{10pt}{11pt}\selectfont}
\setlist[RFColeccionCartas,3]{label*=\textbf{\arabic*.},font=\fontsize{9pt}{11pt}\selectfont}
\setlist[RFColeccionCartas,4]{label*=\textbf{\arabic*.},font=\fontsize{8pt}{11pt}\selectfont}
\setlist[RFColeccionCartas,5]{label*=\textbf{\arabic*.},font=\fontsize{8pt}{11pt}\selectfont}


\subsubsubsection{Colección de cartas}
\begin{RFColeccionCartas}
	\item El sistema tendrá una colección de cartas disponibles para los usuarios.
	\begin{RFColeccionCartas}
		\item Las cartas serán representaciones de pokémon.
		\item Cada carta tendrá dos identificadores:
		\begin{RFColeccionCartas}
			\item Un identificador único (UID).
			\item Un identificador de colección (IDC) en formato \coloredUnderline{\hyperlink{confParam:cc-formatoIDCCarta}{CC\_FORMATO\_IDC\_CARTA}}..
		\end{RFColeccionCartas}
		\item El valor de cada carta será determinado por:
		\begin{RFColeccionCartas}
			\item El pokémon que representan.
			\item Rareza del pokémon.
			\item Rareza de la carta.
		\end{RFColeccionCartas}
	\end{RFColeccionCartas}
	\item El sistema deberá permitir a los usuarios autenticados coleccionar cartas.\label{req_coleccion_cartas}
	\begin{RFColeccionCartas}
		\item Un usuario deberá poder visualizar las cartas que posee en su colección.
		\item El sistema deberá de mostrar si una carta está repetida en la colección del usuario.
		\item Un usuario deberá poder añadir cartas a su colección mediante:
		\begin{RFColeccionCartas}
			\item La compra de sobres (ver Compra de Sobres).
			\item La compra de cartas individuales mediante subastas (ver Gestión de Subastas).
		\end{RFColeccionCartas}
	\end{RFColeccionCartas}
\end{RFColeccionCartas}