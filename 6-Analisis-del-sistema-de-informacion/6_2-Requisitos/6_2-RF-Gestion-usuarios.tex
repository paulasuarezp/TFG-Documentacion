
\newlist{RFGestionUsuarios}{enumerate}{5}
\setlist[RFGestionUsuarios,1]{label=\textbf{RGU-\arabic*.}, leftmargin=*, align=left, font=\fontsize{10pt}{11pt}\selectfont}
\setlist[RFGestionUsuarios,2]{label*=\textbf{\arabic*.},font=\fontsize{10pt}{11pt}\selectfont}
\setlist[RFGestionUsuarios,3]{label*=\textbf{\arabic*.},font=\fontsize{9pt}{11pt}\selectfont}
\setlist[RFGestionUsuarios,4]{label*=\textbf{\arabic*.},font=\fontsize{8pt}{11pt}\selectfont}
\setlist[RFGestionUsuarios,5]{label*=\textbf{\arabic*.},font=\fontsize{8pt}{11pt}\selectfont}


\subsubsection*{Registro e inicio de sesión}
\begin{RFGestionUsuarios}
	\item Un usuario deberá poder registrarse en el sistema.\label{req_registro}
	\begin{RFGestionUsuarios}
		\item El usuario deberá proporcionar el nombre de usuario que desea utilizar.
			\begin{RFGestionUsuarios}
				\item El sistema verificará que el nombre de usuario está disponible.
				\item El sistema verificará que el nombre de usuario cumpla con las restricciones de formato especificadas en \coloredUnderline{\hyperlink{confParam:gu-nombreUsuario}{GU\_NOMBRE\_USUARIO}}.
			\end{RFGestionUsuarios} 
		\item El usuario deberá proporcionar una contraseña.
			\begin{RFGestionUsuarios}
				\item El sistema asegurará que la contraseña cumpla con las políticas de seguridad especificadas en \coloredUnderline{\hyperlink{confParam:gu-contrasena}{GU\_CONTRASEÑA}}.
			\end{RFGestionUsuarios}
		\item Un usuario deberá de confirmar la contraseña
			\begin{RFGestionUsuarios}
				\item El sistema verificará que la contraseña y su confirmación coincidan.
			\end{RFGestionUsuarios}
		\item Un usuario deberá de especificar la fecha de nacimiento.
			\begin{RFGestionUsuarios}
               \item El sistema verificará que la fecha de nacimiento está en el formato \coloredUnderline{\hyperlink{confParam:gu-fechaNacimiento}{GU\_FECHA\_NACIMIENTO}}.
				\item El sistema verificará que el usuario cumple con \coloredUnderline{\hyperlink{confParam:gu-minEdad}{GU\_MIN\_EDAD}}.
                
			\end{RFGestionUsuarios}
		\item Una vez finalizado el registro, el sistema escribirá los datos en la base de datos:
			\begin{RFGestionUsuarios}
				\item Nombre de usuario
				\item Contraseña cifrada
				\item Fecha de nacimiento
				\item Fecha de creación
				\item \coloredUnderline{\hyperlink{confParam:gu-rolDefecto}{GU\_ROL\_DEFECTO}}
				\item \coloredUnderline{\hyperlink{confParam:gu-imgPerfilDefecto}{GU\_IMG\_PERFIL\_DEFECTO}}
				\item \coloredUnderline{\hyperlink{confParam:gu-saldoDefecto}{GU\_SALDO\_DEFECTO}}
			\end{RFGestionUsuarios}

		\item El usuario será redirigido a la pantalla de inicio de sesión.
	\end{RFGestionUsuarios}

	\item Un usuario deberá poder iniciar sesión en el sistema.\label{req_inicio_sesion}
    \begin{RFGestionUsuarios}%RF2.1
      \item Un usuario deberá iniciar sesión las siguientes credenciales.
		\begin{RFGestionUsuarios}
		  \item Nombre de usuario.
		  \item Contraseña.
		\end{RFGestionUsuarios}
	  \item El sistema validará las credenciales introducidas.
		\begin{RFGestionUsuarios}
			\item El sistema comprobará que el nombre de usuario existe en la base de datos.
			\item El sistema comprobará que la contraseña se corresponde con la registrada para dicho nombre de usuario.
		\end{RFGestionUsuarios}

      \item Si las credenciales son correctas:
		\begin{RFGestionUsuarios}
			\item El sistema redirigirá al usuario a la pantalla principal.
			\item Se generará un token de sesión con las restricciones \coloredUnderline{\hyperlink{confParam:gu-tokenSesion}{GU\_TOKEN\_SESION}}.
		\end{RFGestionUsuarios}
	  \item Si las credenciales son incorrectas:
		\begin{RFGestionUsuarios}
			\item El sistema mostrará un mensaje de error.
			\item El usuario podrá intentar iniciar sesión de nuevo.
		\end{RFGestionUsuarios}
    \end{RFGestionUsuarios}

	\item Un usuario autenticado deberá poder cerrar sesión en el sistema.\label{req_cerrar_sesion}
	\begin{RFGestionUsuarios}
		\item El sistema invalidará el token de sesión.
		\item El sistema redirigirá al usuario a la pantalla de inicio de sesión.
	\end{RFGestionUsuarios}

\end{RFGestionUsuarios}