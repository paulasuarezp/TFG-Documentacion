\newlist{RFSubastas}{enumerate}{5}
\setlist[RFSubastas,1]{label=\textbf{RSP-\arabic*.}, leftmargin=*, align=left, font=\fontsize{10pt}{11pt}\selectfont}
\setlist[RFSubastas,2]{label*=\textbf{\arabic*.},font=\fontsize{10pt}{11pt}\selectfont}
\setlist[RFSubastas,3]{label*=\textbf{\arabic*.},font=\fontsize{9pt}{11pt}\selectfont}
\setlist[RFSubastas,4]{label*=\textbf{\arabic*.},font=\fontsize{8pt}{11pt}\selectfont}
\setlist[RFSubastas,5]{label*=\textbf{\arabic*.},font=\fontsize{8pt}{11pt}\selectfont}


\subsubsubsection{Gestión de subastas y pujas} \label{req_subastas_pujas}
\hypertarget{req_subastas_pujas}{}
\begin{RFSubastas}
	\item El sistema permitirá a los usuarios autenticados subastar cartas de su colección.
	\begin{RFSubastas}
		\item Un usuario podrá seleccionar una carta de su colección para subastar.
		\item El sistema verificará que la carta seleccionada no esté en una subasta activa.
		\item Un usuario podrá especificar el precio mínimo al que acepta vender la carta.
		\item Un usuario podrá especificar la duración de la subasta en horas.
		\item El sistema registrará en la base de datos una nueva subasta con los siguientes datos:
		\begin{RFSubastas}
			\item Identificador único (UID).
			\item Identificador único (UID) del usuario vendedor.
			\item Identificador único (UID) de la carta subastada.
			\item Identificador de colección (IDC) de la carta subastada.
			\item Nombre de usuario vendedor.
			\item Estado de la subasta 'activa'.
			\item Precio de salida.
			\item Fecha de inicio.
			\item Duración de la subasta.
			\item Fecha estimada de finalización.
		\end{RFSubastas}
		\item El sistema actualizará el estado de la carta a 'en subasta'.
		\item Se registrará una transacción en la base de datos con los siguientes datos:
		\begin{RFSubastas}
			\item Identificador único (UID), generado por la base de datos.
			\item Identificador único (UID) del usuario que pone en subasta la carta.
			\item Nombre de usuario.
			\item Identificador único (UID) de la carta subastada.
			\item Identificador de colección (IDC) de la carta subastada.
			\item Identificador único (UID) de la subasta.
			\item Concepto de la transacción, en este caso, la puesta en subasta de una carta.
			\item Precio de salida.
			\item Fecha.
		\end{RFSubastas}
		\item El sistema mostrará la subasta activa en la lista de subastas activas.
	\end{RFSubastas}

	\item El sistema permitirá a los usuarios autenticados visualizar las subastas activas.
	\begin{RFSubastas}
		\item Los usuarios podrán consultar la duración restante de la subasta.
		\item Si el usuario es el propietario de la subasta activa se mostrará la opción de retirar la subasta.\hypertarget{req_subastas_pujas:retirarSubasta}{}
		\begin{RFSubastas}   
		\item Un usuario podrá seleccionar una subasta activa para retirar.
		\item El sistema verificará que la subasta seleccionada pertenezca al usuario.
		\item El sistema actualizará el estado de la subasta a 'cancelada'.
		\item El sistema devolverá la carta subastada a la colección del usuario.
		\item El sistema finalizará las pujas activas:
		\begin{RFSubastas}
			\item El sistema actualizará el estado de las pujas a 'cancelada'.
			\item El sistema notificará a los usuarios que hayan pujado en la subasta de la cancelación.
		\end{RFSubastas}
		\item Se registrará una transacción en la base de datos por la cancelación de la subasta con los siguientes datos:
		\begin{RFSubastas}
			\item Identificador único (UID), generado por la base de datos.
			\item Identificador único (UID) del usuario que cancela la subasta.
			\item Nombre de usuario.
			\item Identificador único (UID) de la carta subastada.
			\item Identificador de colección (IDC) de la carta subastada.
			\item Identificador único (UID) de la subasta.
			\item Concepto de la transacción, en este caso, la cancelación de una subasta.
			\item Precio base de la subasta.
			\item Fecha.
		\end{RFSubastas}
		\item Se registrar una transacción por cada puja cancelada con los siguientes datos:
		\begin{RFSubastas}
			\item Identificador único (UID), generado por la base de datos.
			\item Identificador único (UID) del usuario que realizó la puja.
			\item Nombre de usuario.
			\item Identificador único (UID) de la carta subastada.
			\item Identificador de colección (IDC) de la carta subastada.
			\item Identificador único (UID) de la subasta.
			\item Identificador único (UID) de la puja.
			\item Concepto de la transacción, en este caso, la cancelación de una subasta.
			\item Precio de la puja.
			\item Fecha.
		\end{RFSubastas}

	\end{RFSubastas}
		\item El sistema permitirá a los usuarios autenticados pujar en subastas activas.
		\begin{RFSubastas}
			\item El sistema verificará que no se trata de una subasta propia del usuario.
			\item Un usuario deberá especificar el precio de la puja.
			\item El sistema verificará que el usuario no tenga una puja activa en la subasta.
			\item El sistema registrará en la base de datos una nueva puja con los siguientes datos:
			\begin{RFSubastas}
				\item Identificador único (UID).
				\item Identificador único (UID) del usuario que realizó la puja.
				\item Nombre de usuario.
				\item Identificador único (UID) de la carta subastada.
				\item Identificador de colección (IDC) de la carta subastada.
				\item Identificador único (UID) de la subasta.
				\item Estado de la puja 'activa'.
				\item Precio de la puja.
				\item Fecha de la puja.
				\item Fecha estimada de finalización.
			\end{RFSubastas}
			\item El sistema mostrará la puja activa en la lista de pujas activas.
			\item Se registrará una transacción en la base de datos por la puja realizada con los siguientes datos:
			\begin{RFSubastas}
				\item Identificador único (UID), generado por la base de datos.
				\item Identificador único (UID) del usuario que realizó la puja.
				\item Nombre de usuario.
				\item Identificador único (UID) de la carta subastada.
				\item Identificador de colección (IDC) de la carta subastada.
				\item Identificador único (UID) de la subasta.
				\item Identificador único (UID) de la puja.
				\item Concepto de la transacción, en este caso, la realización de una puja.
				\item Precio de la puja.
				\item Fecha.
			\end{RFSubastas}
		\end{RFSubastas}
	\end{RFSubastas}

	\item El sistema permitirá a los usuarios autenticados visualizar las pujas realizadas y que están pendientes de resolución.
		\begin{RFSubastas}
			\item Se mostrará el precio de la puja realizada.
			\item El sistema permitirá al usuario retirar la puja.
			\begin{RFSubastas}
				\item El sistema verificará que la puja seleccionada pertenezca al usuario.
				\item El sistema actualizará el estado de la puja a 'cancelada'.
				\item La puja no se tendrá en cuenta en el proceso de selección de la puja ganadora.
				\item El sistema no mostrará la puja en la lista de pujas activas.
				\item Se registrará una transacción en la base de datos por la cancelación de la puja con los siguientes datos:
				\begin{RFSubastas}
					\item Identificador único (UID), generado por la base de datos.
					\item Identificador único (UID) del usuario que realizó la puja.
					\item Nombre de usuario.
					\item Identificador único (UID) de la carta subastada.
					\item Identificador de colección (IDC) de la carta subastada.
					\item Identificador único (UID) de la subasta.
					\item Identificador único (UID) de la puja.
					\item Concepto de la transacción, en este caso, la cancelación de una puja.
					\item Precio de la puja.
					\item Fecha.
				\end{RFSubastas}
			\end{RFSubastas}
		\end{RFSubastas}
	
	\item El administrador del sistema podrá visualizar todas las subastas activas.
	\begin{RFSubastas}
		\item El administrador puede consultar el precio base que el vendedor ha establecido.
		\item Si considera que la subasta no cumple con las normas del sistema, podrá cancelarla.
		\begin{RFSubastas}
			\item Se notificará a los usuarios implicados en la subasta de la cancelación.
			\item Se le enviará una notificación en tiempo real al usuario vendedor.
			\item Se registrará una transacción en la base de datos por la cancelación de la subasta como se describe en \coloredUnderline{\hyperlink{req_subastas_pujas:retirarSubasta}{Retirar subasta}}.
		\end{RFSubastas}
	\end{RFSubastas}

	\item El administrador del sistema podrá finalizar las subastas activas cuya duración haya expirado.
	\begin{RFSubastas}
		\item El administrador confirmará la finalización de la subastas.
		\item Si hay pujas activas, el sistema seleccionará la puja ganadora.
		\begin{RFSubastas}
			\item El sistema verificará que el usuario tenga saldo suficiente para realizar la compra.
			\item Si el usuario no tiene saldo suficiente se seleccionará la siguiente puja más alta, repitiendo el proceso hasta encontrar un usuario con saldo suficiente.
			\item Si el usuario tiene saldo suficiente se procederá a la compra de la carta.
			\item En el caso de que no haya ninguna puja válida, la carta se devolverá a la colección del usuario vendedor.
			\item En el caso de que haya una puja válida:
			\begin{RFSubastas}
				\item Se registrará una transacción en la base de datos con los siguientes datos:
				\begin{RFSubastas}
					\item Identificador único (UID), generado por la base de datos.
					\item Identificador único (UID) del usuario que realizó la puja.
					\item Nombre de usuario comprador.
					\item Identificador único (UID) de la carta subastada.
					\item Identificador de colección (IDC) de la carta subastada.
					\item Identificador único (UID) de la subasta.
					\item Identificador único (UID) de la puja.
					\item Concepto de la transacción, en este caso, la compra de una carta al ganar una subasta.
					\item Precio de la puja.
					\item Fecha.
				\end{RFSubastas}
				\item El sistema actualizará el saldo del usuario vendedor.
				\item El sistema actualizará el saldo del usuario comprador.
				\item El sistema transferirá la carta al usuario comprador.
				\item El sistema actualizará el estado de la puja a 'ganadora'.
				\item El sistema actualizará la fecha de finalización de la puja a la fecha en la que el administrador confirmó la finalización.
				\item El sistema enviará una notificación al usuario vendedor en tiempo real.
				\item El sistema enviará una notificación al usuario comprador en tiempo real.
			\end{RFSubastas}

			\item Si no hay ninguna puja válida, el sistema devolverá la carta subastada a la colección del usuario.
		\end{RFSubastas}
		\item El sistema notificará a los usuarios implicados en la subasta del resultado.
	
		\item Se registrará la transacción de finalización de la subasta con los siguientes datos:
			\begin{RFSubastas}
				\item Identificador único (UID), generado por la base de datos.
				\item Identificador único (UID) del usuario que realizó la puja.
				\item Nombre de usuario.
				\item Identificador único (UID) de la carta subastada.
				\item Identificador de colección (IDC) de la carta subastada.
				\item Identificador único (UID) de la subasta.
				\item Identificador único (UID) de la puja.
				\item Concepto de la transacción.
				\begin{RFSubastas}
					\item Si hay una puja válida, el concepto será 'venta de carta'.
					\item Si no hay ninguna puja válida, el concepto será 'carta no vendida'.
				\end{RFSubastas}
				\item Precio de la puja, si hay una puja válida.
				\item Fecha.
			\end{RFSubastas}
		\item El sistema actualizará el estado de la subasta a 'finalizada'.
		\item El sistema actualizará la fecha de finalización de la subasta a la fecha en la que el administrador confirmó la finalización.
	
	\end{RFSubastas}
	
	
\end{RFSubastas}