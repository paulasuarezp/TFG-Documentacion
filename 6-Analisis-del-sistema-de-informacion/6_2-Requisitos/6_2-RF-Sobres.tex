\newlist{RFSobres}{enumerate}{5}
\setlist[RFSobres,1]{label=\textbf{RS-\arabic*.}, leftmargin=*, align=left, font=\fontsize{10pt}{11pt}\selectfont}
\setlist[RFSobres,2]{label*=\textbf{\arabic*.},font=\fontsize{10pt}{11pt}\selectfont}
\setlist[RFSobres,3]{label*=\textbf{\arabic*.},font=\fontsize{9pt}{11pt}\selectfont}
\setlist[RFSobres,4]{label*=\textbf{\arabic*.},font=\fontsize{8pt}{11pt}\selectfont}
\setlist[RFSobres,5]{label*=\textbf{\arabic*.},font=\fontsize{8pt}{11pt}\selectfont}


\subsubsubsection{Gestión de sobres}\hypertarget{req_sobres}{}

\begin{RFSobres}
	\item El sistema tendrá una colección de sobres de cartas disponibles para los usuarios.
	\begin{RFSobres}
		\item Cada sobre de cartas tendrá los siguientes campos:
		\begin{RFSobres}
			\item Un identificador único (UID), generado por la base de datos.
			\item Un identificador de colección (IDC) en formato \coloredUnderline{\hyperlink{confParam:cc-formatoIDCSobre}{CC\_FORMATO\_IDC\_SOBRE}}, generado por el sistema.
			\item Nombre del sobre.
			\item Cantidad disponible.
			\item Precio del sobre.
			\item Número de cartas que contiene.
			\item Cantidad de cartas de cada mazo.
			\item Fecha de lanzamiento.
			\item Campo que indica si el sobre está disponible.
		\end{RFSobres}
		\item El valor de cada sobre de cartas será determinado por:
		\begin{RFSobres}
			\item La cantidad de cartas que contiene.
			\item Los mazos a los que pertenecen las cartas.
		\end{RFSobres}
	\end{RFSobres}
	\item Los usuarios autenticados podrán visualizar los sobres disponibles para su compra.
	\item Los usuarios autenticados tendrán la posibilidad de comprar sobres de cartas.\label{req_compra_sobres}
	\begin{RFSobres}
		\item El sistema verificará que el usuario tenga saldo suficiente antes de permitir la compra de un sobre.
		\item Al confirmar la compra, el sistema deducirá el precio del sobre del saldo del usuario.
		\item Se decrementará la cantidad disponible de ese tipo de sobre en el inventario.
		\item El sistema deberá de seleccionar un mazo del que extraer las cartas en base al tipo de sobre comprado.
		\item El sistema generará N cartas aleatorias.
		\item Las cartas generadas serán añadidas a la colección del usuario, identificadas como \textit{USERCARD}.
		\item Cada \textit{USERCARD} se vinculará al tipo de carta correspondiente.
		\item El sistema registrará en la base de datos una nueva transacción por cada carta obtenida, con los siguientes campos:
		\begin{RFSobres}
			\item Identificador único (UID), generado por la base de datos.
			\item Identificador único (UID) del usuario que realizó la compra.
			\item Nombre de usuario.
			\item Identificador único (UID) de la carta obtenida.
			\item Identificador de colección (IDC) de la carta obtenida.
			\item Concepto de compra, en este caso, la compra de un sobre.
			\item Identificador del sobre.
			\item Fecha de compra.
			\item Precio del sobre.
		\end{RFSobres}
		\item Finalmente, el sistema mostrará las cartas adquiridas \textit{USERCARD} al usuario.
	\end{RFSobres}
\end{RFSobres}