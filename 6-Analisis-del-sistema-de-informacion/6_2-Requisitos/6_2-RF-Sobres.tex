\newlist{RFSobres}{enumerate}{5}
\setlist[RFSobres,1]{label=\textbf{RS-\arabic*.}, leftmargin=*, align=left, font=\fontsize{10pt}{11pt}\selectfont}
\setlist[RFSobres,2]{label*=\textbf{\arabic*.},font=\fontsize{10pt}{11pt}\selectfont}
\setlist[RFSobres,3]{label*=\textbf{\arabic*.},font=\fontsize{9pt}{11pt}\selectfont}
\setlist[RFSobres,4]{label*=\textbf{\arabic*.},font=\fontsize{8pt}{11pt}\selectfont}
\setlist[RFSobres,5]{label*=\textbf{\arabic*.},font=\fontsize{8pt}{11pt}\selectfont}


\subsubsubsection{Gestión de sobres}\hypertarget{req_sobres}{}

\begin{RFSobres}
	\item El sistema tendrá una colección de sobres de cartas disponibles para los usuarios.
	\begin{RFSobres}
		\item Cada sobre de cartas tendrá un identificador único (UID).
		\item El valor de cada sobre de cartas será determinado por:
		\begin{RFSobres}
			\item La cantidad de cartas que contiene.
			\item La rareza del sobre.
		\end{RFSobres}
		\item El sistema limitará la cantidad disponible de cada tipo de sobre.
	\end{RFSobres}
	\item Los usuarios autenticados podrán visualizar los sobres disponibles para la compra.
	\item Los usuarios autenticados tendrán la posibilidad de comprar sobres de cartas.\label{req_compra_sobres}
	\begin{RFSobres}
		\item El sistema verificará que el usuario tenga saldo suficiente antes de permitir la compra de un sobre.
		\item Al confirmar la compra, el sistema deducirá el precio del sobre del saldo del usuario.
		\item Se decrementará la cantidad disponible de ese tipo de sobre en el inventario.
		\item El sistema deberá de seleccionar un mazo del que extraer las cartas en base a la rareza del sobre.
		\item El sistema generará N cartas aleatorias.
		\item Las cartas generadas serán añadidas a la colección del usuario, identificadas como \textit{USERCARD}.
		\item Cada \textit{USERCARD} se vinculará al tipo de carta correspondiente.
		\item El sistema registrará en la base de datos una nueva transacción con los siguientes datos:
		\begin{RFSobres}
			\item Identificador único (UID).
			\item Usuario que realiza la compra.
			\item Concepto de compra.
			\item Identificador del sobre.
			\item Fecha de compra.
			\item Precio del sobre.
			\item Cartas obtenidas.
		\end{RFSobres}
		\item Finalmente, el sistema mostrará las cartas \textit{USERCARD} al usuario.
	\end{RFSobres}
\end{RFSobres}