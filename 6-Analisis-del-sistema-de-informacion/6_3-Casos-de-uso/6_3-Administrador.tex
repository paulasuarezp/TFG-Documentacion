\subsubsection{Caso de uso. Histórico de transacciones del sistema} \label{sec:cu_transacciones-sistema}
\begin{longtable}{
   >{\columncolor{lightgreen!20}}p{4cm} % Primera columna con color de fondo verde claro
    >{\columncolor{white}}p{12cm}        % Segunda columna con color de fondo blanco (explícito)
    }
    \caption{Caso de uso. Histórico de transacciones del sistema} \label{table:cu_transacciones-sistema} \\
    \toprule
    \rowcolor{darkgreen!50} % Aplicando color de fondo verde oscuro a toda la fila
    \textbf{Caso de uso} & \centering\arraybackslash \textbf{HISTÓRICO DE TRANSACCIONES DEL SISTEMA} \\
    \endfirsthead
    
    \multicolumn{2}{c}%
    {\tablename\ \thetable{} -- continuación de la página anterior} \\
    \toprule
    \rowcolor{darkgreen!50}
    \textbf{Caso de uso} & \centering\arraybackslash \textbf{HISTÓRICO DE TRANSACCIONES DEL SISTEMA} \\
    \midrule
    \endhead
    
    \midrule
    \multicolumn{2}{r}{Continúa en la siguiente página...} \\ 
    \endfoot
    
    \bottomrule
    \endlastfoot
    
    \midrule
    Descripción & Un usuario administrador puede consultar el histórico de transacciones del sistema. \\
    \midrule
    Actores principales & Usuario administrador \\
    \midrule
    Actores secundarios &  \\
    \midrule
    Precondiciones & \begin{itemize}[nosep,leftmargin=*]
        \item El usuario ha iniciado sesión en el sistema.
    \end{itemize} \\
    \midrule
    Postcondiciones &  \\
    \midrule
    Disparadores & El usuario selecciona la opción de consultar el histórico de transacciones. \\
    \midrule
    Escenario principal & \begin{enumerate}[nosep,leftmargin=*]
        \item El sistema muestra el listado de transacciones realizadas en el sistema por los diferentes usuarios.
    \end{enumerate} \\
    \midrule
    Escenarios alternativos & 
    \begin{itemize}[nosep,leftmargin=*]
        \item \textbf{Escenario alternativo 1. No hay transacciones registradas.}
        \begin{enumerate}[nosep,leftmargin=*]
            \item El sistema muestra un mensaje indicando que no hay transacciones registradas.
        \end{enumerate}
    \end{itemize} \\
    \midrule
    Situaciones de error & 
    \begin{itemize}[nosep,leftmargin=*]
        \item \textbf{Error de conexión a la base de datos.}
        \begin{enumerate}[nosep,leftmargin=*]
            \item El sistema muestra un mensaje de error.
        \end{enumerate}
    \end{itemize} \\
\end{longtable}



\subsubsection{Caso de uso. Cancelar subasta activa} \label{sec:cu_cancelar-subasta}
\begin{longtable}{
    >{\columncolor{lightgreen!20}}p{4cm}
    p{12cm}
    }
    \caption{Caso de uso. Cancelar subasta activa} \label{table:cu_cancelar-subasta} \\
    \toprule
    \rowcolor{darkgreen!50}
    \textbf{Caso de uso} & \multicolumn{1}{>{\columncolor{darkgreen!50}\centering\arraybackslash}p{12cm}}{\textbf{CANCELAR SUBASTA ACTIVA}} \\
    \endfirsthead
    
    \multicolumn{2}{c}%
    {{ \tablename\ \thetable{} Caso de uso. Cancelar subasta activa -- continuación de la página anterior}} \\
    \toprule
    \rowcolor{darkgreen!50}
    \textbf{Caso de uso} & \multicolumn{1}{>{\columncolor{darkgreen!50}\centering\arraybackslash}p{12cm}}{\textbf{CANCELAR SUBASTA ACTIVA}} \\
    \midrule
    \endhead
    
    \midrule
    \multicolumn{2}{r}{{Continúa en la siguiente página...}} \\ 
    \endfoot
    
    \bottomrule
    \endlastfoot
    
    \midrule
    Descripción & Un usuario administrador puede cancelar una subasta activa en el sistema. \\
    \midrule
    Actores principales & Usuario administrador \\
    \midrule
    Actores secundarios &  \\
    \midrule
    Precondiciones & \begin{itemize}[nosep,leftmargin=*]
        \item El usuario ha iniciado sesión en el sistema.
        \item Existe una subasta activa en el sistema.
    \end{itemize} \\
    \midrule
    Postcondiciones & \begin{itemize}[nosep,leftmargin=*]
        \item Se actualiza el estado de la subasta a cancelada en la base de datos.
        \item Se actualizan las pujas realizadas en la subasta cancelada en la base de datos, marcándolas como 'canceladas'.
        \item Se notifica a los usuarios participantes en la subasta de la cancelación.
    \end{itemize} \\
    \midrule
    Disparadores & El usuario selecciona la opción de cancelar una subasta activa. \\
    \midrule
    Escenario principal & \begin{enumerate}[nosep,leftmargin=*]
        \item El sistema muestra el listado de subastas activas en el sistema.
        \item El usuario selecciona la subasta que desea cancelar.
        \item El sistema muestra un mensaje de confirmación de la cancelación.
        \item El usuario confirma la cancelación de la subasta.
        \item El sistema actualiza el estado de la subasta a cancelada en la base de datos.
        \item El sistema actualiza las pujas realizadas en la subasta cancelada en la base de datos, marcándolas como 'canceladas'.
        \item El sistema notifica a los usuarios participantes en la subasta de la cancelación.
        \item El sistema redirige al usuario al listado de subastas activas.
    \end{enumerate} \\
    \midrule
    Escenarios alternativos & \\
    \midrule
    Situaciones de error & 
    \begin{itemize}[nosep,leftmargin=*]
        \item \textbf{Error de conexión a la base de datos.}
        \begin{enumerate}[nosep,leftmargin=*]
            \item El sistema muestra un mensaje de error.
        \end{enumerate}
    \end{itemize} \\
\end{longtable}