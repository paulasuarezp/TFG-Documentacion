
\subsubsection{Caso de uso. Comprar sobre} \label{sec:cu_comprar-sobre}
\begin{longtable}{
    >{\columncolor{lightgreen!20}}p{4cm}
    p{12cm}
    }
    \caption{Caso de uso. Comprar sobre} \label{table:cu_comprar-sobre} \\
    \toprule
    \rowcolor{darkgreen!50}
    \textbf{Caso de uso} & \multicolumn{1}{>{\columncolor{darkgreen!50}\centering\arraybackslash}p{12cm}}{\textbf{COMPRAR SOBRE}} \\
    \endfirsthead
    
    \multicolumn{2}{c}%
    {{ \tablename\ \thetable{} Caso de uso. Comprar sobre -- continuación de la página anterior}} \\
    \toprule
    \rowcolor{darkgreen!50}
    \textbf{Caso de uso} & \multicolumn{1}{>{\columncolor{darkgreen!50}\centering\arraybackslash}p{12cm}}{\textbf{COMPRAR SOBRE}} \\
    \midrule
    \endhead
    
    \midrule
    \multicolumn{2}{r}{{Continúa en la siguiente página...}} \\ 
    \endfoot
    
    \bottomrule
    \endlastfoot
    
    \midrule
    Descripción & Un usuario autenticado puede comprar un sobre de cartas. \\
    \midrule
    Actores principales & Usuario autenticado \\
    \midrule
    Actores secundarios &  \\
    \midrule
    Precondiciones & \begin{itemize}[nosep,leftmargin=*]
        \item El usuario ha iniciado sesión en el sistema.
        \item El usuario dispone de saldo suficiente para comprar el sobre.
    \end{itemize} \\
    \midrule
    Postcondiciones & \begin{itemize}[nosep,leftmargin=*]
        \item Se descuenta el precio del sobre del saldo del usuario.
        \item Se añaden las cartas del sobre a la colección del usuario.
        \item Se decrementa en una unidad la cantidad de sobres disponibles en el inventario.
        \item Se registra la transacción en el historial de compras del usuario.
    \end{itemize} \\
    \midrule
    Disparadores & El usuario hace clic en el botón de comprar sobre. \\
    \midrule
    Escenario principal & \begin{enumerate}[nosep,leftmargin=*]
        \item El sistema muestra el inventario de sobres disponibles.
        \item El usuario selecciona el sobre que desea comprar.
        \item El usuario hace clic en el botón de comprar sobre.
        \item El sistema valida que el usuario dispone de saldo suficiente.
        \item El sistema descuenta el precio del sobre del saldo del usuario.
        \item El sistema genera las cartas del sobre.
        \item El sistema añade las cartas del sobre a la colección del usuario.
    \end{enumerate} \\
    \midrule
    Escenarios alternativos & 
    \begin{itemize}[nosep,leftmargin=*]
        \item \textbf{Escenario alternativo 1. El usuario intenta comprar un sobre sin saldo suficiente.}
        \begin{enumerate}[nosep,leftmargin=*]
            \item El usuario intenta comprar un sobre sin saldo suficiente.
            \item El sistema muestra un mensaje de error.
            \item El sistema le ofrece al usuario la posibilidad de recargar saldo.
        \end{enumerate}
    \end{itemize} \\
    \midrule
    Situaciones de error & 
    \begin{itemize}[nosep,leftmargin=*]
        \item \textbf{Error de conexión a la base de datos.}
        \begin{enumerate}[nosep,leftmargin=*]
            \item El sistema muestra un mensaje de error.
            \item El sistema no descuenta el precio del sobre del saldo del usuario.
            \item El sistema no añade las cartas del sobre a la colección del usuario.
            \item El sistema no decrementa en una unidad la cantidad de sobres disponibles en el inventario.
        \end{enumerate}
    \end{itemize} \\
\end{longtable}



\subsubsection{Caso de uso. Ver colección de cartas} \label{sec:cu_coleccion-cartas}
\begin{longtable}{
    >{\columncolor{lightgreen!20}}p{4cm}
    p{12cm}
    }
    \caption{Caso de uso. Ver colección de cartas} \label{table:cu_coleccion-cartas} \\
    \toprule
    \rowcolor{darkgreen!50}
    \textbf{Caso de uso} & \multicolumn{1}{>{\columncolor{darkgreen!50}\centering\arraybackslash}p{12cm}}{\textbf{VER COLECCIÓN DE CARTAS}} \\
    \endfirsthead
    
    \multicolumn{2}{c}%
    {{ \tablename\ \thetable{} Caso de uso. Ver colección de cartas -- continuación de la página anterior}} \\
    \toprule
    \rowcolor{darkgreen!50}
    \textbf{Caso de uso} & \multicolumn{1}{>{\columncolor{darkgreen!50}\centering\arraybackslash}p{12cm}}{\textbf{VER COLECCIÓN DE CARTAS}} \\
    \midrule
    \endhead
    
    \midrule
    \multicolumn{2}{r}{{Continúa en la siguiente página...}} \\ 
    \endfoot
    
    \bottomrule
    \endlastfoot
    
    \midrule
    Descripción & Un usuario autenticado puede ver la colección de cartas que posee. \\
    \midrule
    Actores principales & Usuario autenticado \\
    \midrule
    Actores secundarios &  \\
    \midrule
    Precondiciones & \begin{itemize}[nosep,leftmargin=*]
        \item El usuario debe haber iniciado sesión en el sistema.
    \end{itemize} \\
    \midrule
    Postcondiciones & \\
    \midrule
    Disparadores & El usuario accede a la sección de colección de cartas. \\
    \midrule
    Escenario principal & \begin{enumerate}[nosep,leftmargin=*]
        \item El sistema muestra la colección de cartas del usuario.
        \item El usuario puede seleccionar una carta para ver su información detallada.
    \end{enumerate} \\
    \midrule
    Escenarios alternativos & 
    \begin{itemize}[nosep,leftmargin=*]
        \item \textbf{Escenario alternativo 1. El usuario no tiene cartas en su colección.}
        \begin{enumerate}[nosep,leftmargin=*]
            \item Se mostrará un mensaje indicando que el usuario no tiene cartas en su colección.
            \item Se mostrará la opción de comprar un sobre de cartas o ir a la sección de subastas.
        \end{enumerate}
    \end{itemize} \\
    \midrule
    Situaciones de error & 
    \begin{itemize}[nosep,leftmargin=*]
        \item \textbf{Error de conexión a la base de datos.}
        \begin{enumerate}[nosep,leftmargin=*]
            \item El sistema mostrará un mensaje de error.
            \item El sistema le dará al usuario la opción de intentar cargar de nuevo la colección o volver a la página principal.
        \end{enumerate}
    \end{itemize} \\
\end{longtable}


\subsubsection{Caso de uso. Marcar carta como destacada} \label{sec:cu_carta-destacada}
\begin{longtable}{
    >{\columncolor{lightgreen!20}}p{4cm}
    p{12cm}
    }
    \caption{Caso de uso. Marcar carta como destacada} \label{table:cu_carta-destacada} \\
    \toprule
    \rowcolor{darkgreen!50}
    \textbf{Caso de uso} & \multicolumn{1}{>{\columncolor{darkgreen!50}\centering\arraybackslash}p{12cm}}{\textbf{MARCAR CARTA COMO DESTACADA}} \\
    \endfirsthead
    
    \multicolumn{2}{c}%
    {{ \tablename\ \thetable{} Caso de uso. Marcar carta como destacada -- continuación de la página anterior}} \\
    \toprule
    \rowcolor{darkgreen!50}
    \textbf{Caso de uso} & \multicolumn{1}{>{\columncolor{darkgreen!50}\centering\arraybackslash}p{12cm}}{\textbf{MARCAR CARTA COMO DESTACADA}} \\
    \midrule
    \endhead
    
    \midrule
    \multicolumn{2}{r}{{Continúa en la siguiente página...}} \\ 
    \endfoot
    
    \bottomrule
    \endlastfoot
    
    \midrule
    Descripción & Un usuario autenticado puede marcar una carta de su colección como destacada. \\
    \midrule
    Actores principales & Usuario autenticado \\
    \midrule
    Actores secundarios &  \\
    \midrule
    Precondiciones & \begin{itemize}[nosep,leftmargin=*]
        \item El usuario debe haber iniciado sesión en el sistema.
        \item El usuario debe tener al menos una carta en su colección.
    \end{itemize} \\
    \midrule
    Postcondiciones & \begin{itemize}[nosep,leftmargin=*]
        \item Se marca la carta como destacada en la base de datos.
    \end{itemize} \\
    \midrule
    Disparadores & El usuario accede a la sección de colección de cartas, selecciona una carta y la marca como destacada. \\
    \midrule
    Escenario principal & \begin{enumerate}[nosep,leftmargin=*]
        \item El sistema muestra la colección de cartas del usuario.
        \item El usuario puede seleccionar una carta para marcarla como destacada.
        \item El usuario hace clic en el botón de marcar como destacada.
        \item El sistema valida que el usuario no tenga ya una carta marcada como destacada.
        \item El sistema marca la carta como destacada en la base de datos.
        \item El sistema muestra un mensaje de éxito.
        \item El sistema redirige al usuario a la página que muestra su colección de cartas.
    \end{enumerate} \\
    \midrule
    Escenarios alternativos & 
    \begin{itemize}[nosep,leftmargin=*]
        \item \textbf{Escenario alternativo 1. El usuario ya tiene una carta marcada como destacada.}
        \begin{enumerate}[nosep,leftmargin=*]
            \item Se mostrará un mensaje indicando que el usuario ya tiene una carta marcada como destacada.
            \item Se mostrará la opción de desmarcar la carta actualmente destacada.
            \item Si el usuario confirma, se desmarcará la carta actualmente destacada y se marcará la nueva carta.
        \end{enumerate}
    \end{itemize} \\
    \midrule
    Situaciones de error & 
    \begin{itemize}[nosep,leftmargin=*]
        \item \textbf{Error de conexión a la base de datos.}
        \begin{enumerate}[nosep,leftmargin=*]
            \item El sistema mostrará un mensaje de error.
            \item El sistema no actualizará la carta como destacada en la base de datos.
            \item El sistema redirigirá al usuario a la página que muestra su colección de cartas.
        \end{enumerate}
    \end{itemize} \\
\end{longtable}




\subsubsection{Caso de uso. Modificar perfil} \label{sec:cu_modificar-perfil}
\begin{longtable}{
    >{\columncolor{lightgreen!20}}p{4cm}
    p{12cm}
    }
    \caption{Caso de uso. Modificar perfil} \label{table:cu_modificar-perfil} \\
    \toprule
    \rowcolor{darkgreen!50}
    \textbf{Caso de uso} & \multicolumn{1}{>{\columncolor{darkgreen!50}\centering\arraybackslash}p{12cm}}{\textbf{MODIFICAR PERFIL}} \\
    \endfirsthead
    
    \multicolumn{2}{c}%
    {{ \tablename\ \thetable{} Caso de uso. Modificar perfil -- continuación de la página anterior}} \\
    \toprule
    \rowcolor{darkgreen!50}
    \textbf{Caso de uso} & \multicolumn{1}{>{\columncolor{darkgreen!50}\centering\arraybackslash}p{12cm}}{\textbf{MODIFICAR PERFIL}} \\
    \midrule
    \endhead
    
    \midrule
    \multicolumn{2}{r}{{Continúa en la siguiente página...}} \\ 
    \endfoot
    
    \bottomrule
    \endlastfoot
    
    \midrule
    Descripción & Un usuario autenticado puede modificar su perfil de usuario. \\
    \midrule
    Actores principales & Usuario autenticado \\
    \midrule
    Actores secundarios &  \\
    \midrule
    Precondiciones & \begin{itemize}[nosep,leftmargin=*]
        \item El usuario debe haber iniciado sesión en el sistema.
    \end{itemize} \\
    \midrule
    Postcondiciones & \begin{itemize}[nosep,leftmargin=*]
        \item Se modifican los datos del perfil del usuario en la base de datos.
    \end{itemize} \\
    \midrule
    Disparadores & El usuario accede a la sección de modificar perfil. \\
    \midrule
    Escenario principal & \begin{enumerate}[nosep,leftmargin=*]
        \item El sistema muestra el formulario de modificación de perfil.
        \item El usuario modifica los campos que desee.
        \item El usuario hace clic en el botón de guardar cambios.
        \item El sistema valida los campos del formulario.
        \item El sistema actualiza los datos del perfil del usuario.
        \item El sistema muestra un mensaje de éxito.
        \item El sistema redirige al usuario a la página que muestra su perfil.
    \end{enumerate} \\
    \midrule
    Escenarios alternativos & 
    \begin{itemize}[nosep,leftmargin=*]
        \item \textbf{Escenario alternativo 1. El usuario cancela la modificación de perfil.}
        \begin{enumerate}[nosep,leftmargin=*]
            \item El usuario hace clic en el botón de cancelar.
            \item El sistema no modifica los datos del perfil del usuario.
            \item El sistema redirige al usuario a la página que muestra su perfil.
        \end{enumerate}
        \item \textbf{Escenario alternativo 2. El usuario introduce datos inválidos.}
        \begin{enumerate}[nosep,leftmargin=*]
            \item El sistema muestra un mensaje de error.
            \item El sistema no modifica los datos del perfil del usuario.
            \item El sistema muestra los campos con errores.
            \item El sistema permite al usuario corregir los errores.
        \end{enumerate}
    \end{itemize} \\
    \midrule
    Situaciones de error & 
    \begin{itemize}[nosep,leftmargin=*]
        \item \textbf{Error de conexión a la base de datos.}
        \begin{enumerate}[nosep,leftmargin=*]
            \item El sistema mostrará un mensaje de error.
            \item El sistema no modificará los datos del perfil del usuario.
            \item El sistema le dará al usuario la opción de intentar guardar de nuevo los cambios o volver a la página que muestra su perfil.
        \end{enumerate}
    \end{itemize} \\
\end{longtable}




\subsubsection{Caso de uso. Consultar histórico de transacciones realizadas} \label{sec:cu_transacciones-realizadas}
\begin{longtable}{
    >{\columncolor{lightgreen!20}}p{4cm}
    p{12cm}
    }
    \caption{Caso de uso. Consultar histórico de transacciones realizadas} \label{table:cu_transacciones-realizadas} \\
    \toprule
    \rowcolor{darkgreen!50}
    \textbf{Caso de uso} & \multicolumn{1}{>{\columncolor{darkgreen!50}\centering\arraybackslash}p{12cm}}{\textbf{CONSULTAR HISTÓRICO DE TRANSACCIONES REALIZADAS}} \\
    \endfirsthead
    
    \multicolumn{2}{c}%
    {{ \tablename\ \thetable{} Caso de uso. Consultar histórico de transacciones realizadas -- continuación de la página anterior}} \\
    \toprule
    \rowcolor{darkgreen!50}
    \textbf{Caso de uso} & \multicolumn{1}{>{\columncolor{darkgreen!50}\centering\arraybackslash}p{12cm}}{\textbf{CONSULTAR HISTÓRICO DE TRANSACCIONES REALIZADAS}} \\
    \midrule
    \endhead
    
    \midrule
    \multicolumn{2}{r}{{Continúa en la siguiente página...}} \\ 
    \endfoot
    
    \bottomrule
    \endlastfoot
    
    \midrule
    Descripción & Un usuario autenticado puede consultar el histórico de transacciones realizadas en el sistema. \\
    \midrule
    Actores principales & Usuario autenticado \\
    \midrule
    Actores secundarios &  \\
    \midrule
    Precondiciones & \begin{itemize}[nosep,leftmargin=*]
        \item El usuario debe haber iniciado sesión en el sistema.
    \end{itemize} \\
    \midrule
    Postcondiciones & \\
    \midrule
    Disparadores & El usuario accede a la sección de historial de transacciones. \\
    \midrule
    Escenario principal & \begin{enumerate}[nosep,leftmargin=*]
        \item El sistema muestra el historial de transacciones del usuario.
    \end{enumerate} \\
    \midrule
    Escenarios alternativos & 
    \begin{itemize}[nosep,leftmargin=*]
        \item \textbf{Escenario alternativo 1. El usuario no tiene transacciones realizadas.}
        \begin{enumerate}[nosep,leftmargin=*]
            \item El sistema mostrará un mensaje indicando que el usuario no tiene transacciones realizadas.
        \end{enumerate}
    \end{itemize} \\
    \midrule
    Situaciones de error & 
    \begin{itemize}[nosep,leftmargin=*]
        \item \textbf{Error de conexión a la base de datos.}
        \begin{enumerate}[nosep,leftmargin=*]
            \item El sistema mostrará un mensaje de error.
            \item El sistema le dará al usuario la opción de intentar cargar de nuevo el historial de transacciones o volver a la página principal.
        \end{enumerate}
    \end{itemize} \\
\end{longtable}




\subsubsection{Caso de uso. Recargar saldo} \label{sec:cu_recarga-saldo}
\begin{longtable}{
    >{\columncolor{lightgreen!20}}p{4cm}
    p{12cm}
    }
    \caption{Caso de uso. Recargar saldo} \label{table:cu_recarga-saldo} \\
    \toprule
    \rowcolor{darkgreen!50}
    \textbf{Caso de uso} & \multicolumn{1}{>{\columncolor{darkgreen!50}\centering\arraybackslash}p{12cm}}{\textbf{RECRAGAR SALDO}} \\
    \endfirsthead
    
    \multicolumn{2}{c}%
    {{ \tablename\ \thetable{} Caso de uso. Recargar saldo -- continuación de la página anterior}} \\
    \toprule
    \rowcolor{darkgreen!50}
    \textbf{Caso de uso} & \multicolumn{1}{>{\columncolor{darkgreen!50}\centering\arraybackslash}p{12cm}}{\textbf{RECARGAR SALDO}} \\
    \midrule
    \endhead
    
    \midrule
    \multicolumn{2}{r}{{Continúa en la siguiente página...}} \\ 
    \endfoot
    
    \bottomrule
    \endlastfoot
    
    \midrule
    Descripción & Un usuario autenticado puede recargar saldo en su cuenta. \\
    \midrule
    Actores principales & Usuario autenticado \\
    \midrule
    Actores secundarios &  Pasarela de pago \\
    \midrule
    Precondiciones & \begin{itemize}[nosep,leftmargin=*]
        \item El usuario debe haber iniciado sesión en el sistema.
    \end{itemize} \\
    \midrule
    Postcondiciones & \begin{itemize}[nosep,leftmargin=*]
        \item Se incrementa el saldo del usuario en la base de datos.
    \end{itemize} \\
    \midrule
    Disparadores & El usuario accede a la sección de recarga de saldo. \\
    \midrule
    Escenario principal & \begin{enumerate}[nosep,leftmargin=*]
        \item El sistema muestra el formulario de recarga de saldo.
        \item El usuario introduce la cantidad de saldo que desea recargar.
        \item El usuario hace clic en el botón de recargar saldo.
        \item El sistema valida la cantidad de saldo introducida.
        \item El sistema redirige al usuario a la pasarela de pago.
        \item El usuario completa el pago.
        \item El sistema incrementa el saldo del usuario en la base de datos.
        \item El sistema muestra un mensaje de éxito.
    \end{enumerate} \\
    \midrule
    Escenarios alternativos & 
    \begin{itemize}[nosep,leftmargin=*]
        \item \textbf{Escenario alternativo 1. El usuario cancela la recarga de saldo en la pasarela de pago.}
        \begin{enumerate}[nosep,leftmargin=*]
            \item El usuario cancela el pago.
            \item El sistema no incrementa el saldo del usuario en la base de datos.
            \item El sistema redirige al usuario a la página que muestra su saldo actual.
            \item El sistema muestra un mensaje de cancelación.
        \end{enumerate}
        \item \textbf{Escenario alternativo 2. El usuario introduce una cantidad de saldo inválida.}
        \begin{enumerate}[nosep,leftmargin=*]
            \item El sistema muestra un mensaje de error.
            \item El sistema muestra los campos con errores.
            \item El sistema permite al usuario corregir los errores.
        \end{enumerate}
    \end{itemize} \\
    \midrule
    Situaciones de error & 
    \begin{itemize}[nosep,leftmargin=*]
        \item \textbf{Error de conexión a la base de datos.}
        \begin{enumerate}[nosep,leftmargin=*]
            \item El sistema mostrará un mensaje de error.
            \item El sistema no incrementará el saldo del usuario en la base de datos.
            \item El sistema le dará al usuario la opción de intentar recargar de nuevo el saldo o volver a la página de inicio.
        \end{enumerate}
        \item \textbf{Error en la pasarela de pago.}
        \begin{enumerate}[nosep,leftmargin=*]
            \item El sistema mostrará un mensaje de error.
            \item El sistema no incrementará el saldo del usuario en la base de datos.
            \item El sistema redirigirá al usuario a la página de recarga de saldo.
        \end{enumerate}
    \end{itemize} \\
\end{longtable}



\subsubsection{Caso de uso. Crear una subasta} \label{sec:cu_crear-subasta}
\begin{longtable}{
    >{\columncolor{lightgreen!20}}p{4cm}
    p{12cm}
    }
    \caption{Caso de uso. Crear una subasta} \label{table:cu_crear-subasta} \\
    \toprule
    \rowcolor{darkgreen!50}
    \textbf{Caso de uso} & \multicolumn{1}{>{\columncolor{darkgreen!50}\centering\arraybackslash}p{12cm}}{\textbf{CREAR UNA SUBASTA}} \\
    \endfirsthead
    
    \multicolumn{2}{c}%
    {{ \tablename\ \thetable{} Caso de uso. Crear una subasta -- continuación de la página anterior}} \\
    \toprule
    \rowcolor{darkgreen!50}
    \textbf{Caso de uso} & \multicolumn{1}{>{\columncolor{darkgreen!50}\centering\arraybackslash}p{12cm}}{\textbf{CREAR UNA SUBASTA}} \\
    \midrule
    \endhead
    
    \midrule
    \multicolumn{2}{r}{{Continúa en la siguiente página...}} \\ 
    \endfoot
    
    \bottomrule
    \endlastfoot
    
    \midrule
    Descripción & Un usuario autenticado puede crear una subasta de una carta de su colección. \\
    \midrule
    Actores principales & Usuario autenticado \\
    \midrule
    Actores secundarios &  \\
    \midrule
    Precondiciones & \begin{itemize}[nosep,leftmargin=*]
        \item El usuario debe haber iniciado sesión en el sistema.
        \item El usuario debe tener al menos una carta en su colección.
        \item El usuario no debe tener ya una subasta activa para la carta que desea subastar.
    \end{itemize} \\
    \midrule
    Postcondiciones & \begin{itemize}[nosep,leftmargin=*]
        \item Se crea una nueva subasta en la base de datos.
        \item Se marca la carta como 'en subasta' en la base de datos.
    \end{itemize} \\
    \midrule
    Disparadores & El usuario accede a la sección de subastas y hace clic en el botón de crear subasta. \\
    \midrule
    Escenario principal & \begin{enumerate}[nosep,leftmargin=*]
        \item El sistema muestra el formulario de creación de subasta.
        \item El usuario selecciona la carta que desea subastar.
        \item El usuario introduce el precio de salida y la duración de la subasta.
        \item El usuario hace clic en el botón de crear subasta.
        \item El sistema valida los campos del formulario.
        \item El sistema crea una nueva subasta en la base de datos.
        \item El sistema marca la carta como 'en subasta' en la base de datos.
        \item El sistema muestra un mensaje de éxito.
        \item El sistema redirige al usuario a la página de subastas.
        \item El sistema muestra la subasta creada en la lista de subastas.
    \end{enumerate} \\
    \midrule
    Escenarios alternativos & 
    \begin{itemize}[nosep,leftmargin=*]
        \item \textbf{Escenario alternativo 1. El usuario ya tiene una subasta activa para la carta que desea subastar.}
        \begin{enumerate}[nosep,leftmargin=*]
            \item El usuario intenta crear una subasta para una carta que ya está en subasta.
            \item El sistema muestra un mensaje de error.
            \item El sistema redirige al usuario a la página de subastas.
        \end{enumerate}
        \item \textbf{Escenario alternativo 2. El usuario introduce datos inválidos.}
        \begin{enumerate}[nosep,leftmargin=*]
            \item El sistema muestra un mensaje de error.
            \item El sistema no crea la subasta en la base de datos.
            \item El sistema no marcará la carta como 'en subasta' en la base de datos.
            \item El sistema le mostrará al usuario los campos con errores.
            \item El sistema permitirá al usuario corregir los errores.
        \end{enumerate}
    \end{itemize} \\
    \midrule
    Situaciones de error & 
    \begin{itemize}[nosep,leftmargin=*]
        \item \textbf{Error de conexión a la base de datos.}
        \begin{enumerate}[nosep,leftmargin=*]
            \item El sistema mostrará un mensaje de error.
            \item El sistema no creará la subasta en la base de datos.
            \item El sistema no marcará la carta como 'en subasta' en la base de datos.
            \item El sistema redirigirá al usuario a la página de subastas.
        \end{enumerate}
    \end{itemize} \\
\end{longtable}




\subsubsection{Caso de uso. Retirar una subasta} \label{sec:cu_retirar-subasta}
\begin{longtable}{
    >{\columncolor{lightgreen!20}}p{4cm}
    p{12cm}
    }
    \caption{Caso de uso. Retirar una subasta} \label{table:cu_retirar-subasta} \\
    \toprule
    \rowcolor{darkgreen!50}
    \textbf{Caso de uso} & \multicolumn{1}{>{\columncolor{darkgreen!50}\centering\arraybackslash}p{12cm}}{\textbf{RETIRAR UNA SUBASTA}} \\
    \endfirsthead
    
    \multicolumn{2}{c}%
    {{ \tablename\ \thetable{} Caso de uso. Retirar una subasta -- continuación de la página anterior}} \\
    \toprule
    \rowcolor{darkgreen!50}
    \textbf{Caso de uso} & \multicolumn{1}{>{\columncolor{darkgreen!50}\centering\arraybackslash}p{12cm}}{\textbf{RETIRAR UNA SUBASTA}} \\
    \midrule
    \endhead
    
    \midrule
    \multicolumn{2}{r}{{Continúa en la siguiente página...}} \\ 
    \endfoot
    
    \bottomrule
    \endlastfoot
    
    \midrule
    Descripción & Un usuario  \\
    \midrule
    Actores principales & Usuario autenticado \\
    \midrule
    Actores secundarios &  \\
    \midrule
    Precondiciones & \begin{itemize}[nosep,leftmargin=*]
        \item El usuario debe haber iniciado sesión en el sistema.
        \item El usuario debe tener al menos una subasta activa.
    \end{itemize} \\
    \midrule
    Postcondiciones & \begin{itemize}[nosep,leftmargin=*]
        \item Se marca la subasta como 'cancelada' en la base de datos.
        \item La carta subastada vuelve a estar disponible en la colección del usuario.
        \item Se actualizan las pujas de la subasta en la base de datos, marcando las pujas como 'canceladas'.
        \item Se informa a los usuarios que hayan pujado en la subasta de que ha sido cancelada.
    \end{itemize} \\
    \midrule
    Disparadores & El usuario accede a la sección de subastas y hace clic en el botón de retirar subasta. \\
    \midrule
    Escenario principal & \begin{enumerate}[nosep,leftmargin=*]
        \item El sistema muestra la lista de subastas activas del usuario.
        \item El usuario selecciona la subasta que desea retirar.
        \item El usuario hace clic en el botón de retirar subasta.
        \item El sistema muestra un mensaje de confirmación.
        \item El usuario confirma la retirada de la subasta.
        \item El sistema marca la subasta como 'cancelada' en la base de datos.
        \item La carta subastada vuelve a estar disponible en la colección del usuario.
        \item El sistema muestra un mensaje de éxito.
        \item El sistema redirige al usuario a la página de subastas.
    \end{enumerate} \\
    \midrule
    Escenarios alternativos & 
    \begin{itemize}[nosep,leftmargin=*]
        \item \textbf{Escenario alternativo 1. El usuario cancela la retirada de la subasta.}
        \begin{enumerate}[nosep,leftmargin=*]
            \item El usuario hace clic en el botón de cancelar en el mensaje de confirmación.
            \item El sistema no marcará la subasta como 'cancelada' en la base de datos.
            \item El sistema redirigirá al usuario a la página de subastas.
        \end{enumerate}
    \end{itemize} \\
    \midrule
    Situaciones de error & 
    \begin{itemize}[nosep,leftmargin=*]
        \item \textbf{Error de conexión a la base de datos.}
        \begin{enumerate}[nosep,leftmargin=*]
            \item El sistema mostrará un mensaje de error.
            \item El sistema no marcará la subasta como 'cancelada' en la base de datos.
            \item El sistema no devolverá la carta subastada a la colección del usuario.
            \item El sistema redirigirá al usuario a la página de subastas.
        \end{enumerate}
    \end{itemize} \\
\end{longtable}




\subsubsection{Caso de uso. Consultar subastas activas} \label{sec:cu_consultar-subastas}
\begin{longtable}{
    >{\columncolor{lightgreen!20}}p{4cm}
    p{12cm}
    }
    \caption{Caso de uso. Consultar subastas activas} \label{table:cu_consultar-subastas} \\
    \toprule
    \rowcolor{darkgreen!50}
    \textbf{Caso de uso} & \multicolumn{1}{>{\columncolor{darkgreen!50}\centering\arraybackslash}p{12cm}}{\textbf{CONSULTAR SUBASTAS ACTIVAS}} \\
    \endfirsthead
    
    \multicolumn{2}{c}%
    {{ \tablename\ \thetable{} Caso de uso. Consultar subastas activas -- continuación de la página anterior}} \\
    \toprule
    \rowcolor{darkgreen!50}
    \textbf{Caso de uso} & \multicolumn{1}{>{\columncolor{darkgreen!50}\centering\arraybackslash}p{12cm}}{\textbf{CONSULTAR SUBASTAS ACTIVAS}} \\
    \midrule
    \endhead
    
    \midrule
    \multicolumn{2}{r}{{Continúa en la siguiente página...}} \\ 
    \endfoot
    
    \bottomrule
    \endlastfoot
    
    \midrule
    Descripción & Un usuario  \\
    \midrule
    Actores principales & Usuario autenticado o usuario administrador \\
    \midrule
    Actores secundarios &  \\
    \midrule
    Precondiciones & \begin{itemize}[nosep,leftmargin=*]
        \item El usuario debe haber iniciado sesión en el sistema.
    \end{itemize} \\
    \midrule
    Postcondiciones &  \\
    \midrule
    Disparadores & El usuario accede a la sección de subastas. \\
    \midrule
    Escenario principal & \begin{enumerate}[nosep,leftmargin=*]
        \item El sistema muestra la lista de subastas activas.
        \item El usuario puede seleccionar una subasta para ver su información detallada.
    \end{enumerate} \\
    \midrule
    Escenarios alternativos & 
    \begin{itemize}[nosep,leftmargin=*]
        \item \textbf{Escenario alternativo 1. No hay subastas activas.}
        \begin{enumerate}[nosep,leftmargin=*]
            \item El sistema mostrará un mensaje indicando que no hay subastas activas.
        \end{enumerate}
        \item \textbf{Escenario alternativo 2. El usuario consulta sus propias subastas.} 
        \begin{enumerate}[nosep,leftmargin=*]
            \item El usuario accede a la sección de subastas.
            \item El sistema muestra la opción de ver las subastas activas del usuario.
            \item El usuario selecciona la opción de ver sus propias subastas.
            \item El sistema muestra la lista de subastas activas del usuario.
        \end{enumerate}
        \item \textbf{Escenario alternativo 3. Un usuario administrador consulta las subastas activas.}
        \begin{enumerate}[nosep,leftmargin=*]
            \item El usuario administrador accede a la sección de subastas.
            \item El sistema muestra la lista de subastas activas de todos los usuarios.
            \item El usuario administrador puede seleccionar una subasta para ver su información detallada.
            \item El usuario administrador puede seleccionar una subasta para cancelarla.
        \end{enumerate}
    \end{itemize} \\
    \midrule
    Situaciones de error & 
    \begin{itemize}[nosep,leftmargin=*]
        \item \textbf{Error de conexión a la base de datos.}
        \begin{enumerate}[nosep,leftmargin=*]
            \item El sistema mostrará un mensaje de error.
        \end{enumerate}
    \end{itemize} \\
\end{longtable}



\subsubsection{Caso de uso. Realizar puja} \label{sec:cu_realizar-puja}
\begin{longtable}{
    >{\columncolor{lightgreen!20}}p{4cm}
    p{12cm}
    }
    \caption{Caso de uso. Realizar puja} \label{table:cu_realizar-puja} \\
    \toprule
    \rowcolor{darkgreen!50}
    \textbf{Caso de uso} & \multicolumn{1}{>{\columncolor{darkgreen!50}\centering\arraybackslash}p{12cm}}{\textbf{REALIZAR PUJA}} \\
    \endfirsthead
    
    \multicolumn{2}{c}%
    {{ \tablename\ \thetable{} Caso de uso. Realizar puja -- continuación de la página anterior}} \\
    \toprule
    \rowcolor{darkgreen!50}
    \textbf{Caso de uso} & \multicolumn{1}{>{\columncolor{darkgreen!50}\centering\arraybackslash}p{12cm}}{\textbf{REALIZAR PUJA}} \\
    \midrule
    \endhead
    
    \midrule
    \multicolumn{2}{r}{{Continúa en la siguiente página...}} \\ 
    \endfoot
    
    \bottomrule
    \endlastfoot
    
    \midrule
    Descripción & Un usuario autenticado puede realizar una puja en una subasta activa. \\
    \midrule
    Actores principales & Usuario autenticado \\
    \midrule
    Actores secundarios &  \\
    \midrule
    Precondiciones & \begin{itemize}[nosep,leftmargin=*]
        \item El usuario debe haber iniciado sesión en el sistema.
        \item El usuario debe tener saldo suficiente para realizar la puja.
    \end{itemize} \\
    \midrule
    Postcondiciones & \begin{itemize}[nosep,leftmargin=*]
        \item Se registra la puja en la base de datos.
    \end{itemize} \\
    \midrule
    Disparadores & El usuario selecciona una subasta activa y hace clic en el botón de realizar puja. \\
    \midrule
    Escenario principal & \begin{enumerate}[nosep,leftmargin=*]
        \item El sistema muestra la información detallada de la subasta.
        \item El usuario introduce la cantidad de la puja.
        \item El usuario hace clic en el botón de realizar puja.
        \item El sistema valida la cantidad de la puja.
        \item El sistema registra la puja en la base de datos.
        \item El sistema muestra un mensaje de éxito.
        \item El sistema muestra un mensaje informativo, indicando que si en el momento de finalización del tiempo de la subasta no cuenta con el saldo suficiente, la puja no será válida.
        \item El sistema redirige al usuario a la página de subastas.
    \end{enumerate} \\
    \midrule
    Escenarios alternativos & 
    \begin{itemize}[nosep,leftmargin=*]
        \item \textbf{Escenario alternativo 1. El usuario no tiene saldo suficiente para realizar la puja.}
        \begin{enumerate}[nosep,leftmargin=*]
            \item El usuario intenta realizar una puja sin tener saldo suficiente.
            \item El sistema mostrará un mensaje de error.
            \item El sistema no registrará la puja en la base de datos.
            \item El sistema redirigirá al usuario a la página de subastas.
        \end{enumerate}
        \item \textbf{Escenario alternativo 2. El usuario introduce una cantidad de puja inválida.}
        \begin{enumerate}[nosep,leftmargin=*]
            \item El sistema mostrará un mensaje de error.
            \item El sistema no registrará la puja en la base de datos.
            \item El sistema le mostrará al usuario los campos con errores.
            \item El sistema permitirá al usuario corregir los errores.
        \end{enumerate}
    \end{itemize} \\
    \midrule
    Situaciones de error & 
    \begin{itemize}[nosep,leftmargin=*]
        \item \textbf{Error de conexión a la base de datos.}
        \begin{enumerate}[nosep,leftmargin=*]
            \item El sistema mostrará un mensaje de error.
            \item El sistema no registrará la puja en la base de datos.
            \item El sistema redirigirá al usuario a la página de subastas.
        \end{enumerate}
    \end{itemize} \\
\end{longtable}



\subsubsection{Caso de uso. Retirar puja} \label{sec:cu_retirar-puja}
\begin{longtable}{
    >{\columncolor{lightgreen!20}}p{4cm}
    p{12cm}
    }
    \caption{Caso de uso. Retirar puja} \label{table:cu_retirar-puja} \\
    \toprule
    \rowcolor{darkgreen!50}
    \textbf{Caso de uso} & \multicolumn{1}{>{\columncolor{darkgreen!50}\centering\arraybackslash}p{12cm}}{\textbf{RETIRAR PUJA}} \\
    \endfirsthead
    
    \multicolumn{2}{c}%
    {{ \tablename\ \thetable{} Caso de uso. Retirar puja -- continuación de la página anterior}} \\
    \toprule
    \rowcolor{darkgreen!50}
    \textbf{Caso de uso} & \multicolumn{1}{>{\columncolor{darkgreen!50}\centering\arraybackslash}p{12cm}}{\textbf{RETIRAR PUJA}} \\
    \midrule
    \endhead
    
    \midrule
    \multicolumn{2}{r}{{Continúa en la siguiente página...}} \\ 
    \endfoot
    
    \bottomrule
    \endlastfoot
    
    \midrule
    Descripción & Un usuario podrá retirar una puja realizada en una subasta activa. \\
    \midrule
    Actores principales & Usuario autenticado \\
    \midrule
    Actores secundarios &  \\
    \midrule
    Precondiciones & \begin{itemize}[nosep,leftmargin=*]
        \item El usuario debe haber iniciado sesión en el sistema.
        \item El usuario debe haber realizado una puja en una subasta activa.
        \item La subasta no debe haber finalizado.
    \end{itemize} \\
    \midrule
    Postcondiciones & \begin{itemize}[nosep,leftmargin=*]
        \item Se actualiza la puja en la base de datos, marcándola como 'retirada'.
        \item La puja no se muestra en la lista de pujas activas del usuario.
        \item La puja no se tendr´a en cuenta en el cálculo de la puja más alta.
    \end{itemize} \\
    \midrule
    Disparadores & El usuario accede a la sección de pujas activas, selecciona una puja y hace clic en el botón de retirar puja. \\
    \midrule
    Escenario principal & \begin{enumerate}[nosep,leftmargin=*]
        \item El sistema muestra la lista de pujas activas del usuario.
        \item El usuario selecciona la puja que desea retirar.
        \item El usuario hace clic en el botón de retirar puja.
        \item El sistema muestra un mensaje de confirmación.
        \item El usuario confirma la retirada de la puja.
        \item El sistema actualiza la puja en la base de datos, marcándola como 'retirada'.
        \item El sistema muestra un mensaje de éxito.
        \item El sistema redirige al usuario a la página de subastas.
    \end{enumerate} \\
    \midrule
    Escenarios alternativos & 
    \begin{itemize}[nosep,leftmargin=*]
        \item \textbf{Escenario alternativo 1. El usuario cancela la retirada de la puja.}
        \begin{enumerate}[nosep,leftmargin=*]
            \item El usuario hace clic en el botón de cancelar en el mensaje de confirmación.
            \item El sistema no marcará la puja como 'retirada' en la base de datos.
            \item El sistema redirigirá al usuario a la página de subastas.
        \end{enumerate}
    \end{itemize} \\
    \midrule
    Situaciones de error & 
    \begin{itemize}[nosep,leftmargin=*]
        \item \textbf{Error de conexión a la base de datos.}
        \begin{enumerate}[nosep,leftmargin=*]
            \item El sistema mostrará un mensaje de error.
            \item El sistema redirigirá al usuario a la página de subastas.
        \end{enumerate}
    \end{itemize} \\
\end{longtable}



\subsubsection{Caso de uso. Consultar pujas activas} \label{sec:cu_consultar-pujas}
\begin{longtable}{
    >{\columncolor{lightgreen!20}}p{4cm}
    p{12cm}
    }
    \caption{Caso de uso. Consultar pujas activas} \label{table:cu_consultar-puja} \\
    \toprule
    \rowcolor{darkgreen!50}
    \textbf{Caso de uso} & \multicolumn{1}{>{\columncolor{darkgreen!50}\centering\arraybackslash}p{12cm}}{\textbf{CONSULTAR PUJAS ACTIVAS}} \\
    \endfirsthead
    
    \multicolumn{2}{c}%
    {{ \tablename\ \thetable{} Caso de uso. Consultar pujas activas -- continuación de la página anterior}} \\
    \toprule
    \rowcolor{darkgreen!50}
    \textbf{Caso de uso} & \multicolumn{1}{>{\columncolor{darkgreen!50}\centering\arraybackslash}p{12cm}}{\textbf{CONSULTAR PUJAS ACTIVAS}} \\
    \midrule
    \endhead
    
    \midrule
    \multicolumn{2}{r}{{Continúa en la siguiente página...}} \\ 
    \endfoot
    
    \bottomrule
    \endlastfoot
    
    \midrule
    Descripción & Un usuario  \\
    \midrule
    Actores principales & Usuario autenticado \\
    \midrule
    Actores secundarios &  \\
    \midrule
    Precondiciones & \begin{itemize}[nosep,leftmargin=*]
        \item El usuario debe haber iniciado sesión en el sistema.
    \end{itemize} \\
    \midrule
    Postcondiciones & \\
    \midrule
    Disparadores & El usuario accede a la sección de subastas y hace clic en el botón de ver pujas activas. \\
    \midrule
    Escenario principal & \begin{enumerate}[nosep,leftmargin=*]
        \item El sistema muestra la lista de pujas activas del usuario.
    \end{enumerate} \\
    \midrule
    Escenarios alternativos & 
    \begin{itemize}[nosep,leftmargin=*]
        \item \textbf{Escenario alternativo 1. No hay pujas activas.}
        \begin{enumerate}[nosep,leftmargin=*]
            \item El sistema mostrará un mensaje indicando que no hay pujas activas.
        \end{enumerate}
    \end{itemize} \\
    \midrule
    Situaciones de error & 
    \begin{itemize}[nosep,leftmargin=*]
        \item \textbf{Error de conexión a la base de datos.}
        \begin{enumerate}[nosep,leftmargin=*]
            \item El sistema mostrará un mensaje de error.
        \end{enumerate}
    \end{itemize} \\
\end{longtable}



\subsubsection{Caso de uso. Cerrar sesión} \label{sec:cu_cerrar-sesion}
\begin{longtable}{
    >{\columncolor{lightgreen!20}}p{4cm}
    p{12cm}
    }
    \caption{Caso de uso. Cerrar sesión} \label{table:cu_cerrar-sesion} \\
    \toprule
    \rowcolor{darkgreen!50}
    \textbf{Caso de uso} & \multicolumn{1}{>{\columncolor{darkgreen!50}\centering\arraybackslash}p{12cm}}{\textbf{CERRAR SESIÓN}} \\
    \endfirsthead
    
    \multicolumn{2}{c}%
    {{ \tablename\ \thetable{} Caso de uso. Cerrar sesión -- continuación de la página anterior}} \\
    \toprule
    \rowcolor{darkgreen!50}
    \textbf{Caso de uso} & \multicolumn{1}{>{\columncolor{darkgreen!50}\centering\arraybackslash}p{12cm}}{\textbf{CERRAR SESIÓN}} \\
    \midrule
    \endhead
    
    \midrule
    \multicolumn{2}{r}{{Continúa en la siguiente página...}} \\ 
    \endfoot
    
    \bottomrule
    \endlastfoot
    
    \midrule
    Descripción & Un usuario puede cerrar sesión en el sistema si previamente ha iniciado sesión. \\
    \midrule
    Actores principales & Usuario autenticado o usuario administrador \\
    \midrule
    Actores secundarios &  \\
    \midrule
    Precondiciones & \begin{itemize}[nosep,leftmargin=*]
        \item El usuario debe haber iniciado sesión en el sistema.
    \end{itemize} \\
    \midrule
    Postcondiciones & \begin{itemize}[nosep,leftmargin=*]
        \item Se elimina la sesión del usuario.
    \end{itemize} \\
    \midrule
    Disparadores & El usuario hace clic en el botón de cerrar sesión. \\
    \midrule
    Escenario principal & \begin{enumerate}[nosep,leftmargin=*]
        \item El usuario hace clic en el botón de cerrar sesión.
        \item El sistema elimina la sesión del usuario.
        \item El sistema redirige al usuario a la página de inicio.
    \end{enumerate} \\
    \midrule
    Escenarios alternativos &  \\
    \midrule
    Situaciones de error &  \\
\end{longtable}