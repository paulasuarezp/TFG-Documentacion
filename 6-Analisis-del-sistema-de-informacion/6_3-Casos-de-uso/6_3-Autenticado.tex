
\subsubsection{Caso de uso. Comprar sobre} \label{sec:cu_comprar-sobre}
\begin{longtable}{
    >{\columncolor{lightgreen!20}}p{4cm}
    p{12cm}
    }
    \caption{Caso de uso. Comprar sobre} \label{table:cu_comprar-sobre} \\
    \toprule
    \rowcolor{darkgreen!50}
    \textbf{Caso de uso} & \multicolumn{1}{>{\columncolor{darkgreen!50}\centering\arraybackslash}p{12cm}}{\textbf{COMPRAR SOBRE}} \\
    \endfirsthead
    
    \multicolumn{2}{c}%
    {{ \tablename\ \thetable{} Caso de uso. Comprar sobre -- continuación de la página anterior}} \\
    \toprule
    \rowcolor{darkgreen!50}
    \textbf{Caso de uso} & \multicolumn{1}{>{\columncolor{darkgreen!50}\centering\arraybackslash}p{12cm}}{\textbf{COMPRAR SOBRE}} \\
    \midrule
    \endhead
    
    \midrule
    \multicolumn{2}{r}{{Continúa en la siguiente página...}} \\ 
    \endfoot
    
    \bottomrule
    \endlastfoot
    
    \midrule
    Descripción & Un usuario autenticado puede comprar un sobre de cartas. \\
    \midrule
    Actores principales & Usuario autenticado \\
    \midrule
    Actores secundarios &  \\
    \midrule
    Precondiciones & \begin{itemize}[nosep,leftmargin=*]
        \item El usuario ha iniciado sesión en el sistema.
        \item El usuario dispone de saldo suficiente para comprar el sobre.
    \end{itemize} \\
    \midrule
    Postcondiciones & \begin{itemize}[nosep,leftmargin=*]
        \item Se descuenta el precio del sobre del saldo del usuario.
        \item Se añaden las cartas del sobre a la colección del usuario.
        \item Se decrementa en una unidad la cantidad de sobres disponibles en el inventario.
        \item Se registra la transacción en el historial de compras del usuario.
    \end{itemize} \\
    \midrule
    Disparadores & El usuario hace clic en el botón de comprar sobre. \\
    \midrule
    Escenario principal & \begin{enumerate}[nosep,leftmargin=*]
        \item El sistema muestra el inventario de sobres disponibles.
        \item El usuario selecciona el sobre que desea comprar.
        \item El usuario hace clic en el botón de comprar sobre.
        \item El sistema valida que el usuario dispone de saldo suficiente.
        \item El sistema descuenta el precio del sobre del saldo del usuario.
        \item El sistema genera las cartas del sobre.
        \item El sistema añade las cartas del sobre a la colección del usuario.
    \end{enumerate} \\
    \midrule
    Escenarios alternativos & 
    \begin{itemize}[nosep,leftmargin=*]
        \item \textbf{Escenario alternativo 1. El usuario intenta comprar un sobre sin saldo suficiente.}
        \begin{enumerate}[nosep,leftmargin=*]
            \item El usuario intenta comprar un sobre sin saldo suficiente.
            \item El sistema muestra un mensaje de error.
            \item El sistema le ofrece al usuario la posibilidad de recargar saldo.
        \end{enumerate}
    \end{itemize} \\
    \midrule
    Situaciones de error & 
    \begin{itemize}[nosep,leftmargin=*]
        \item \textbf{Error de conexión a la base de datos.}
        \begin{enumerate}[nosep,leftmargin=*]
            \item El sistema muestra un mensaje de error.
            \item El sistema no descuenta el precio del sobre del saldo del usuario.
            \item El sistema no añade las cartas del sobre a la colección del usuario.
            \item El sistema no decrementa en una unidad la cantidad de sobres disponibles en el inventario.
        \end{enumerate}
    \end{itemize} \\
\end{longtable}



\subsubsection{Caso de uso. Ver colección de cartas} \label{sec:cu_coleccion-cartas}
\begin{longtable}{
    >{\columncolor{lightgreen!20}}p{4cm}
    p{12cm}
    }
    \caption{Caso de uso. Ver colección de cartas} \label{table:cu_coleccion-cartas} \\
    \toprule
    \rowcolor{darkgreen!50}
    \textbf{Caso de uso} & \multicolumn{1}{>{\columncolor{darkgreen!50}\centering\arraybackslash}p{12cm}}{\textbf{VER COLECCIÓN DE CARTAS}} \\
    \endfirsthead
    
    \multicolumn{2}{c}%
    {{ \tablename\ \thetable{} Caso de uso. Ver colección de cartas -- continuación de la página anterior}} \\
    \toprule
    \rowcolor{darkgreen!50}
    \textbf{Caso de uso} & \multicolumn{1}{>{\columncolor{darkgreen!50}\centering\arraybackslash}p{12cm}}{\textbf{VER COLECCIÓN DE CARTAS}} \\
    \midrule
    \endhead
    
    \midrule
    \multicolumn{2}{r}{{Continúa en la siguiente página...}} \\ 
    \endfoot
    
    \bottomrule
    \endlastfoot
    
    \midrule
    Descripción & Un usuario autenticado puede ver la colección de cartas que posee. \\
    \midrule
    Actores principales & Usuario autenticado \\
    \midrule
    Actores secundarios &  \\
    \midrule
    Precondiciones & \begin{itemize}[nosep,leftmargin=*]
        \item El usuario debe haber iniciado sesión en el sistema.
    \end{itemize} \\
    \midrule
    Postcondiciones & \\
    \midrule
    Disparadores & El usuario accede a la sección de colección de cartas. \\
    \midrule
    Escenario principal & \begin{enumerate}[nosep,leftmargin=*]
        \item El sistema muestra la colección de cartas del usuario.
        \item El usuario puede seleccionar una carta para ver su información detallada.
    \end{enumerate} \\
    \midrule
    Escenarios alternativos & 
    \begin{itemize}[nosep,leftmargin=*]
        \item \textbf{Escenario alternativo 1. El usuario no tiene cartas en su colección.}
        \begin{enumerate}[nosep,leftmargin=*]
            \item Se mostrará un mensaje indicando que el usuario no tiene cartas en su colección.
            \item Se mostrará la opción de comprar un sobre de cartas o ir a la sección de subastas.
        \end{enumerate}
    \end{itemize} \\
    \midrule
    Situaciones de error & 
    \begin{itemize}[nosep,leftmargin=*]
        \item \textbf{Error de conexión a la base de datos.}
        \begin{enumerate}[nosep,leftmargin=*]
            \item El sistema mostrará un mensaje de error.
            \item El sistema no mostrará la colección de cartas del usuario.
            \item El sistema le dará al usuario la opción de intentar cargar de nuevo la colección o volver a la página principal.
        \end{enumerate}
    \end{itemize} \\
\end{longtable}


\subsubsection{Caso de uso. Marcar carta como destacada} \label{sec:cu_carta-destacada}
\begin{longtable}{
    >{\columncolor{lightgreen!20}}p{4cm}
    p{12cm}
    }
    \caption{Caso de uso. Marcar carta como destacada} \label{table:cu_carta-destacada} \\
    \toprule
    \rowcolor{darkgreen!50}
    \textbf{Caso de uso} & \multicolumn{1}{>{\columncolor{darkgreen!50}\centering\arraybackslash}p{12cm}}{\textbf{MARCAR CARTA COMO DESTACADA}} \\
    \endfirsthead
    
    \multicolumn{2}{c}%
    {{ \tablename\ \thetable{} Caso de uso. Marcar carta como destacada -- continuación de la página anterior}} \\
    \toprule
    \rowcolor{darkgreen!50}
    \textbf{Caso de uso} & \multicolumn{1}{>{\columncolor{darkgreen!50}\centering\arraybackslash}p{12cm}}{\textbf{MARCAR CARTA COMO DESTACADA}} \\
    \midrule
    \endhead
    
    \midrule
    \multicolumn{2}{r}{{Continúa en la siguiente página...}} \\ 
    \endfoot
    
    \bottomrule
    \endlastfoot
    
    \midrule
    Descripción & Un usuario autenticado puede marcar una carta de su colección como destacada. \\
    \midrule
    Actores principales & Usuario autenticado \\
    \midrule
    Actores secundarios &  \\
    \midrule
    Precondiciones & \begin{itemize}[nosep,leftmargin=*]
        \item El usuario debe haber iniciado sesión en el sistema.
        \item El usuario debe tener al menos una carta en su colección.
    \end{itemize} \\
    \midrule
    Postcondiciones & \begin{itemize}[nosep,leftmargin=*]
        \item Se marca la carta como destacada en la base de datos.
    \end{itemize} \\
    \midrule
    Disparadores & El usuario accede a la sección de colección de cartas, selecciona una carta y la marca como destacada. \\
    \midrule
    Escenario principal & \begin{enumerate}[nosep,leftmargin=*]
        \item El sistema muestra la colección de cartas del usuario.
        \item El usuario puede seleccionar una carta para marcarla como destacada.
        \item El usuario hace clic en el botón de marcar como destacada.
        \item El sistema valida que el usuario no tenga ya una carta marcada como destacada.
        \item El sistema marca la carta como destacada en la base de datos.
        \item El sistema muestra un mensaje de éxito.
        \item El sistema redirige al usuario a la página que muestra su colección de cartas.
    \end{enumerate} \\
    \midrule
    Escenarios alternativos & 
    \begin{itemize}[nosep,leftmargin=*]
        \item \textbf{Escenario alternativo 1. El usuario ya tiene una carta marcada como destacada.}
        \begin{enumerate}[nosep,leftmargin=*]
            \item Se mostrará un mensaje indicando que el usuario ya tiene una carta marcada como destacada.
            \item Se mostrará la opción de desmarcar la carta actualmente destacada.
            \item Si el usuario confirma, se desmarcará la carta actualmente destacada y se marcará la nueva carta.
        \end{enumerate}
    \end{itemize} \\
    \midrule
    Situaciones de error & 
    \begin{itemize}[nosep,leftmargin=*]
        \item \textbf{Error de conexión a la base de datos.}
        \begin{enumerate}[nosep,leftmargin=*]
            \item El sistema mostrará un mensaje de error.
            \item El sistema no actualizará la carta como destacada en la base de datos.
            \item El sistema redirigirá al usuario a la página que muestra su colección de cartas.
        \end{enumerate}
    \end{itemize} \\
\end{longtable}




\subsubsection{Caso de uso. Modificar perfil} \label{sec:cu_modificar-perfil}
\begin{longtable}{
    >{\columncolor{lightgreen!20}}p{4cm}
    p{12cm}
    }
    \caption{Caso de uso. Modificar perfil} \label{table:cu_modificar-perfil} \\
    \toprule
    \rowcolor{darkgreen!50}
    \textbf{Caso de uso} & \multicolumn{1}{>{\columncolor{darkgreen!50}\centering\arraybackslash}p{12cm}}{\textbf{MODIFICAR PERFIL}} \\
    \endfirsthead
    
    \multicolumn{2}{c}%
    {{ \tablename\ \thetable{} Caso de uso. Modificar perfil -- continuación de la página anterior}} \\
    \toprule
    \rowcolor{darkgreen!50}
    \textbf{Caso de uso} & \multicolumn{1}{>{\columncolor{darkgreen!50}\centering\arraybackslash}p{12cm}}{\textbf{MODIFICAR PERFIL}} \\
    \midrule
    \endhead
    
    \midrule
    \multicolumn{2}{r}{{Continúa en la siguiente página...}} \\ 
    \endfoot
    
    \bottomrule
    \endlastfoot
    
    \midrule
    Descripción & Un usuario autenticado puede modificar su perfil de usuario. \\
    \midrule
    Actores principales & Usuario autenticado \\
    \midrule
    Actores secundarios &  \\
    \midrule
    Precondiciones & \begin{itemize}[nosep,leftmargin=*]
        \item El usuario debe haber iniciado sesión en el sistema.
    \end{itemize} \\
    \midrule
    Postcondiciones & \begin{itemize}[nosep,leftmargin=*]
        \item Se modifican los datos del perfil del usuario en la base de datos.
    \end{itemize} \\
    \midrule
    Disparadores & El usuario accede a la sección de modificar perfil. \\
    \midrule
    Escenario principal & \begin{enumerate}[nosep,leftmargin=*]
        \item El sistema muestra el formulario de modificación de perfil.
        \item El usuario modifica los campos que desee.
        \item El usuario hace clic en el botón de guardar cambios.
        \item El sistema valida los campos del formulario.
        \item El sistema actualiza los datos del perfil del usuario.
        \item El sistema muestra un mensaje de éxito.
        \item El sistema redirige al usuario a la página que muestra su perfil.
    \end{enumerate} \\
    \midrule
    Escenarios alternativos & 
    \begin{itemize}[nosep,leftmargin=*]
        \item \textbf{Escenario alternativo 1. El usuario cancela la modificación de perfil.}
        \begin{enumerate}[nosep,leftmargin=*]
            \item El usuario hace clic en el botón de cancelar.
            \item El sistema no modifica los datos del perfil del usuario.
            \item El sistema redirige al usuario a la página que muestra su perfil.
        \end{enumerate}
        \item \textbf{Escenario alternativo 2. El usuario introduce datos inválidos.}
        \begin{enumerate}[nosep,leftmargin=*]
            \item El sistema muestra un mensaje de error.
            \item El sistema no modifica los datos del perfil del usuario.
            \item El sistema muestra los campos con errores.
            \item El sistema permite al usuario corregir los errores.
        \end{enumerate}
    \end{itemize} \\
    \midrule
    Situaciones de error & 
    \begin{itemize}[nosep,leftmargin=*]
        \item \textbf{Error de conexión a la base de datos.}
        \begin{enumerate}[nosep,leftmargin=*]
            \item El sistema mostrará un mensaje de error.
            \item El sistema no modificará los datos del perfil del usuario.
            \item El sistema le dará al usuario la opción de intentar guardar de nuevo los cambios o volver a la página que muestra su perfil.
        \end{enumerate}
    \end{itemize} \\
\end{longtable}




\subsubsection{Caso de uso. Consultar histórico de transacciones realizadas} \label{sec:cu_transacciones-realizadas}
\begin{longtable}{
    >{\columncolor{lightgreen!20}}p{4cm}
    p{12cm}
    }
    \caption{Caso de uso. Consultar histórico de transacciones realizadas} \label{table:cu_transacciones-realizadas} \\
    \toprule
    \rowcolor{darkgreen!50}
    \textbf{Caso de uso} & \multicolumn{1}{>{\columncolor{darkgreen!50}\centering\arraybackslash}p{12cm}}{\textbf{CONSULTAR HISTÓRICO DE TRANSACCIONES REALIZADAS}} \\
    \endfirsthead
    
    \multicolumn{2}{c}%
    {{ \tablename\ \thetable{} Caso de uso. Consultar histórico de transacciones realizadas -- continuación de la página anterior}} \\
    \toprule
    \rowcolor{darkgreen!50}
    \textbf{Caso de uso} & \multicolumn{1}{>{\columncolor{darkgreen!50}\centering\arraybackslash}p{12cm}}{\textbf{CONSULTAR HISTÓRICO DE TRANSACCIONES REALIZADAS}} \\
    \midrule
    \endhead
    
    \midrule
    \multicolumn{2}{r}{{Continúa en la siguiente página...}} \\ 
    \endfoot
    
    \bottomrule
    \endlastfoot
    
    \midrule
    Descripción & Un usuario autenticado puede consultar el histórico de transacciones realizadas en el sistema. \\
    \midrule
    Actores principales & Usuario autenticado \\
    \midrule
    Actores secundarios &  \\
    \midrule
    Precondiciones & \begin{itemize}[nosep,leftmargin=*]
        \item El usuario debe haber iniciado sesión en el sistema.
    \end{itemize} \\
    \midrule
    Postcondiciones & \\
    \midrule
    Disparadores & El usuario accede a la sección de historial de transacciones. \\
    \midrule
    Escenario principal & \begin{enumerate}[nosep,leftmargin=*]
        \item El sistema muestra el historial de transacciones del usuario.
    \end{enumerate} \\
    \midrule
    Escenarios alternativos & 
    \begin{itemize}[nosep,leftmargin=*]
        \item \textbf{Escenario alternativo 1. El usuario no tiene transacciones realizadas.}
        \begin{enumerate}[nosep,leftmargin=*]
            \item El sistema mostrará un mensaje indicando que el usuario no tiene transacciones realizadas.
        \end{enumerate}
    \end{itemize} \\
    \midrule
    Situaciones de error & 
    \begin{itemize}[nosep,leftmargin=*]
        \item \textbf{Error de conexión a la base de datos.}
        \begin{enumerate}[nosep,leftmargin=*]
            \item El sistema mostrará un mensaje de error.
            \item El sistema le dará al usuario la opción de intentar cargar de nuevo el historial de transacciones o volver a la página principal.
        \end{enumerate}
    \end{itemize} \\
\end{longtable}