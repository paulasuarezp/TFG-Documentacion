En este apartado se describirá brevemente el sistema a desarrollar indicando el alcance del mismo.
\subsection{Determinación del Alcance del Sistema}
BidMon Universe tiene como objetivo principal permitir al usuario coleccionar activos digitales, en este caso, cartas. Estas se pueden obtener mediante una compra directa a la aplicación o por medio de subastas. Para obtener más detalles sobre las funcionalidades de alto nivel, se puede consultar el apartado 
\coloredUnderline{\hyperlink{sec:2_identificacion_alcance_PSI}{2.1.2. Identificación del alcance del Plan de Sistemas de Información}}, y para una descripción más exhaustiva, ver el apartado \coloredUnderline{\hyperlink{sec:6_2-Requisitos}{6.2. Requisitos}}.
Es esencial establecer un sistema de trazabilidad robusto que registre cada transacción de compra y venta dentro de la aplicación de manera clara e inequívoca, asegurando la integridad y la transparencia de todas las operaciones.
En esta versión del sistema, la carga inicial de datos se llevará a cabo una única vez. Los administradores del sistema no podrán añadir nuevas cartas a través de la interfaz de usuario, cualquier adición de nuevas cartas se realizará directamente en la base de datos. Además, la plataforma no incluirá una opción que permita a los usuarios exportar la colección de cartas que posean.
La aplicación no incluye un sistema de mensajería interna ni permite la formación de relaciones sociales directas entre los usuarios, tales como "amistades" o conexiones similares. La plataforma se centra exclusivamente en las transacciones y la colección de cartas.



\subsection{Identificación de Actores del Sistema} 
El sistema está diseñado para soportar tres categorías principales de usuarios, cada uno con niveles de acceso y funcionalidades específicas adecuadas a su rol dentro de la plataforma:
\begin{enumerate}
    \item \textbf{Usuarios Invitados}
    \begin{itemize}
        \item Tienen acceso limitado exclusivamente a la página de bienvenida y a otras páginas informativas de acceso público dentro de la plataforma.
        \item El objetivo es proporcionar a los usuarios no registrados información general sobre la plataforma y sus características sin comprometer la seguridad o la privacidad de las transacciones y actividades de los usuarios registrados.
    \end{itemize}
    
    \item \textbf{Usuarios Autenticados}
    \begin{itemize}
        \item Tienen acceso a todas las funcionalidades operativas de la plataforma, incluyendo la colección de cartas, compra y subasta de estas.
    \end{itemize}
    
    \item \textbf{Usuarios Administradores}
    \begin{itemize}
        \item El acceso estaría limitado a la consulta de subastas activas, capacidad de intervenir en una subasta que considere fraudulenta y acceso al histórico de transacciones.
    \end{itemize}
\end{enumerate}