\subsection{Determinación del Alcance del Sistema}
BidMon Universe tiene como objetivo principal permitir al usuario coleccionar activos digitales, en este caso, cartas. Estas se pueden obtener mediante una compra directa a la aplicación o 
por medio de subastas ciegas, donde las ofertas de otros participantes permanecen ocultas hasta el cierre de la subasta, añadiendo un elemento emocionante 
y asegurando la imparcialidad en el proceso de compra y venta. 

Es esencial establecer un sistema de trazabilidad robusto que registre cada transacción de compra y venta dentro de la aplicación de manera clara e inequívoca, asegurando la integridad y 
la transparencia de todas las operaciones. Un sistema integral de notificaciones mantendrá a los usuarios constantemente informados sobre las actualizaciones de las subastas en las que están involucrados, 
ya sea como vendedores o como postores.
En esta versión del sistema, la carga inicial de datos se llevará a cabo una única vez. Los administradores del sistema no podrán añadir nuevas cartas a través de la interfaz de usuario, 
cualquier adición de nuevas cartas se realizará directamente en la base de datos. 

Además, la plataforma no incluirá una opción que permita a los usuarios exportar la colección de cartas que posean.
La aplicación no incluye un sistema de mensajería interna ni permite la formación de relaciones sociales directas entre los usuarios. 
La plataforma se centra exclusivamente en las transacciones y la colección de cartas. Los usuarios podrán administrar sus colecciones de manera privada, 
destacando cartas específicas en sus perfiles para exhibir sus activos más valorados.

En \coloredUnderline{\hyperlink{sec:2_2_1-Especificacion_ambito_alcance}{\ref*{sec:2_2_1-Especificacion_ambito_alcance} \nameref*{sec:2_2_1-Especificacion_ambito_alcance}}} se detalla en profundidad el alcance del sistema y sus limitaciones.
En el apartado \coloredUnderline{\hyperlink{sec:2_identificacion_alcance_PSI}{\ref*{sec:2_identificacion_alcance_PSI} \nameref*{sec:2_identificacion_alcance_PSI}}} se listan las 
funcionalidades de alto nivel que se espera que tenga la aplicación, y para una descripción más exhaustiva, 
ver el apartado \coloredUnderline{\hyperlink{sec:6_2-Requisitos}{\ref*{sec:6_2-Requisitos} \nameref*{sec:6_2-Requisitos}}}.


\subsection{Identificación de Actores del Sistema} \label{sec:6_1-Identificacion_actores}
\hypertarget{sec:6_1-Identificacion_actores}{}

El sistema está diseñado para soportar tres categorías principales de usuarios, cada uno con niveles de acceso y funcionalidades específicas adecuadas a su rol dentro de la plataforma:
\begin{enumerate}
    \item \textbf{Usuarios Invitados}. 
    Los usuarios invitados son aquellos que aún no han completado el proceso de registro ni han iniciado sesión en la plataforma. 
    \begin{itemize}
        \item Tienen acceso exclusivamente a la página de bienvenida y a otras páginas informativas de acceso público dentro de la plataforma.
        \item El objetivo es proporcionar a los usuarios no registrados información general sobre la plataforma y sus características, sin comprometer la seguridad o la privacidad de las transacciones y actividades de los usuarios registrados.
   \end{itemize}
    
    \item \textbf{Usuarios Autenticados}.
    Un usuario autenticado es aquel que ha completado el proceso de registro y ha iniciado sesión en la plataforma.
    \begin{itemize}
        \item Tienen acceso a todas las funcionalidades operativas de la plataforma, incluyendo la colección de cartas, compra y subasta de estas.
    \end{itemize}
    
    \item \textbf{Usuarios Administradores}.
    Un usuario administrador es un usuario que ha iniciado sesión en la plataforma y tiene permisos especiales para administrar y supervisar las operaciones de la plataforma.
    \begin{itemize}
        \item Tienen acceso a todas las funcionalidades operativas de la plataforma, sin poder realizar compras o subastas.
        \item Tiene la capacidad de intervenir en una subasta que considere fraudulenta.
        \item Tiene acceso a todo el historial de transacciones y subastas realizadas en la plataforma.
    \end{itemize}
\end{enumerate}