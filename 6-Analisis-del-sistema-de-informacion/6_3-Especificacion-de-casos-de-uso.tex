
\textcolor[rgb]{0.65,0.16,0}{Ejemplo de tabla para especificación de casos de uso}

\begin{table}[htbp]
  \centering
  \caption{Especificación Caso de Uso 1}
    \begin{tabular}{p{20.855em}r}
\cmidrule{1-1}    \rowcolor[rgb]{ .949,  .949,  .949} \multicolumn{1}{p{20.855em}}{\textbf{Nombre del caso de uso}} & \multicolumn{1}{r}{\cellcolor[rgb]{ 1,  1,  1}} \\
\cmidrule{1-1}    \multicolumn{1}{p{20.855em}}{Registro} & \multicolumn{1}{r}{} \\
    \midrule
    \rowcolor[rgb]{ .949,  .949,  .949} \multicolumn{2}{p{31.64em}}{\textbf{Descripción}} \\
    \midrule
    \multicolumn{2}{p{31.64em}}{Un usuario no registrado debe poder registrarse en el sistema mediante su cuenta de Google, lo que hará que automáticamente se inicie sesión en la aplicación.} \\
    \bottomrule
    \end{tabular}%
  \label{espec_caso_uso_1}%
  \vspace{-4mm}
\end{table}%
