
\section{PREPARACIÓN DEL ENTORNO DE GENERACIÓN Y CONSTRUCCIÓN}

\subsection{Herramientas y programas usados para el desarrollo}
Para el desarrollo de la aplicación se han utilizado diversas herramientas y programas que han facilitado la creación del sistema.  

\subsubsection{Lenguaje de programación}
El sistema ha sido desarrollado en su totalidad en TypeScript\cite{typescript}, tanto el \textit{frontend} como el \textit{backend}. 

TypeScript es un superconjunto de JavaScript que añade tipado estático al lenguaje.
Esto permite detectar errores en tiempo de compilación y facilita el mantenimiento del código ya que es mucho más descriptivo.

\begin{figure}[H]
    \centering
    \includegraphics[width=0.2\textwidth]{figures/7-Construccion/Typescript.png}
    \caption{Logo de TypeScript}
\end{figure}

\subsubsection{Entorno de ejecución}
Se ha utilizado Node.js\cite{nodejs} como entorno de ejecución para JavaScript.
Node.js permite ejecutar código JavaScript en el servidor y es muy popular en el desarrollo de aplicaciones web.
A través de npm, el gestor de paquetes de Node.js, se han instalado las dependencias necesarias para el desarrollo de la aplicación.
Este gestor de paquetes permite instalar y gestionar las dependencias de un proyecto de forma sencilla y cuenta con un amplio repositorio de paquetes.

\begin{figure}[H]
    \centering
    \includegraphics[width=0.2\textwidth]{figures/7-Construccion/Nodejs.jpeg}
    \caption{Logo de Node.js}
\end{figure}


\subsubsection{Base de datos}
La base de datos utilizada en el sistema es MongoDB\cite{mongodb}, una base de datos NoSQL orientada a documentos.
Se ha utilizado MongoDB Atlas como servicio de base de datos en la nube, lo que ha permitido desplegar la base de datos de forma sencilla y segura.

\begin{figure}[H]
    \centering
    \includegraphics[width=0.2\textwidth]{figures/7-Construccion/mongodb.png}
    \caption{Logo de MongoDB}
\end{figure}


\subsubsection{Proveedor de servicios en la nube}
Para el despliegue de la aplicación se ha utilizado Azure\cite{azure}, la plataforma en la nube de Microsoft.
Azure ofrece una amplia gama de servicios en la nube, como servidores virtuales, bases de datos, almacenamiento y servicios de inteligencia artificial.

\begin{figure}[H]
    \centering
    \includegraphics[width=0.2\textwidth]{figures/7-Construccion/MicrosoftAzure.png}
    \caption{Logo de Microsoft Azure}
\end{figure}


\subsubsection{Librerías y frameworks}
Se han utilizado diversas librerías y frameworks para el desarrollo de la aplicación, tanto en el \textit{frontend} como en el \textit{backend}.
Este tipo de herramientas permiten acelerar el desarrollo y facilitan la creación de interfaces de usuario y la gestión de la lógica de negocio.
\subsubsection{Frontend}
El \textit{frontend} de la aplicación ha sido desarrollado en React\cite{react}, una librería de JavaScript para la creación de interfaces de usuario,
en combinación con Material-UI\cite{materialui}, una librería de componentes de React, se ha creado una interfaz de usuario moderna y atractiva.

\begin{figure}[H]
    \centering
    \begin{minipage}{0.3\textwidth}
        \centering
        \includegraphics[width=\textwidth]{figures/7-Construccion/React.png}
        \caption{Logo de React}
    \end{minipage}
    \hfill
    \begin{minipage}{0.3\textwidth}
        \centering
        \includegraphics[width=\textwidth]{figures/7-Construccion/MaterialUI.png}
        \caption{Logo de Material-UI}
    \end{minipage}
\end{figure}


El resto de dependencias que cabe destacar utilizadas en el \textit{frontend} son:
\begin{itemize}
        \item \texttt{redux}: Biblioteca para el manejo del estado de la aplicación.
        \item \texttt{framer-motion}: Librería para animaciones en React.
        \item \texttt{socket.io-client}: Cliente para la comunicación en tiempo real con WebSockets.
        \item \texttt{styled-components}: Librería para escribir CSS en JavaScript.
        \item \texttt{yup}: Librería para la validación de esquemas de datos.
        \item \texttt{sweetalert2}: Librería para mostrar alertas personalizadas.
        \item \texttt{react-router-dom}: Librería para la navegación en React.
        \item \texttt{react-hook-form}: Librería para la gestión de formularios en React.
        \item \texttt{react-paypal-js}: Librería para la integración de pagos con PayPal.
        \item \texttt{react-slick}: Librería para la creación de carruseles en React.
        \item \texttt{Jest}: Framework de pruebas para JavaScript.
        \item \texttt{Puppeteer}: Librería para realizar pruebas de extremo a extremo.
        \item \texttt{Cucumber}: Herramienta para pruebas.
\end{itemize}



\subsubsection{Backend}
El \textit{backend} de la aplicación ha sido desarrollado con Express\cite{express}, un framework de Node.js para la creación de aplicaciones web y APIs.
Express es muy popular en el desarrollo de aplicaciones web y permite crear servidores de forma sencilla y rápida.

\begin{figure}[H]
    \centering
    \includegraphics[width=0.2\textwidth]{figures/7-Construccion/express.png}
    \caption{Logo de Express}
\end{figure}

Las dependencias más destacadas utilizadas en el \textit{backend} son:
\begin{itemize}
    \item \texttt{axios}: Cliente HTTP basado en promesas para Node.js y el navegador.
    \item \texttt{bcrypt}: Librería para el hashing de contraseñas.
    \item \texttt{body-parser}: Middleware para analizar cuerpos de solicitudes HTTP en Node.js.
    \item \texttt{cors}: Middleware para habilitar CORS (Cross-Origin Resource Sharing).
    \item \texttt{jsonwebtoken}: Implementación de JSON Web Tokens.
    \item \texttt{mongoose}: Librería de modelado de datos de MongoDB para Node.js.
    \item \texttt{socket.io}: Biblioteca para aplicaciones web en tiempo real.
    \item \texttt{yup}: Librería para la validación de esquemas de datos.
    \item \texttt{csv-parser}: Librería para analizar archivos CSV.
    \item \texttt{csv-writer}: Librería para escribir archivos CSV.
    \item \texttt{paypal-rest-sdk}: SDK para la integración con PayPal.
    \item \texttt{Jest}: Framework de pruebas para JavaScript.
    \item \texttt{supertest}: Biblioteca para pruebas HTTP.
\end{itemize}



\subsubsection{Entorno de desarrollo}
\subsubsubsection{Gestión de versiones}
Para la gestión de versiones se ha utilizado Git\cite{git}, un sistema de control de versiones distribuido que permite llevar un control de los cambios en el código fuente.
Como plataforma de alojamiento de repositorios se ha utilizado GitHub\cite{github}, que permite alojar proyectos de software y facilita la colaboración entre desarrolladores.
Además, se ha utilizado GitHub Actions para la integración continua y GitKraken\cite{gitkraken} como cliente de escritorio para la gestión de ramas y \textit{pull requests}.


\begin{figure}[H]
    \centering
    \begin{minipage}{0.2\textwidth}
        \centering
        \includegraphics[width=\textwidth]{figures/7-Construccion/Git.png}
        \caption{Logo de Git}
    \end{minipage}
    \hfill
    \begin{minipage}{0.2\textwidth}
        \centering
        \includegraphics[width=\textwidth]{figures/7-Construccion/Github.png}
        \caption{Logo de GitHub}
    \end{minipage}
    \hfill
    \begin{minipage}{0.3\textwidth}
        \centering
        \includegraphics[width=\textwidth]{figures/7-Construccion/GitKraken.jpeg}
        \caption{Logo de GitKraken}
    \end{minipage}
\end{figure}

\subsubsubsection{Editor de código}
Para el desarrollo del código fuente se ha utilizado Visual Studio Code\cite{vscode}, 
un editor de código fuente desarrollado por Microsoft que incluye soporte para depuración, control de versiones y resaltado de sintaxis.
Admite extensiones que permiten ampliar sus funcionalidades y es muy popular entre los desarrolladores.

\begin{figure}[H]
    \centering
    \includegraphics[width=0.2\textwidth]{figures/7-Construccion/VSCode.png}
    \caption{Logo de Visual Studio Code}
\end{figure}

Además, se ha utilizado Notepad++\cite{notepad++} como editor de texto para la edición de archivos de configuración y otros archivos de texto
necesarios para desplegar la aplicación en un servidor.

\begin{figure}[H]
    \centering
    \includegraphics[width=0.2\textwidth]{figures/7-Construccion/Notepad.png}
    \caption{Logo de Notepad++}
\end{figure}


\subsubsection{Cliente SSH}
Para la conexión remota con el servidor se ha utilizado MobaXterm\cite{mobaxterm}, un cliente SSH que permite conectarse a servidores remotos de forma segura.
Tiene una interfaz gráfica intuitiva y permite la transferencia de archivos a través de SCP y SFTP.
Se ha utilizado principalmente para la conexión con el servidor de producción y para la transferencia de archivos.

\begin{figure}[H]
    \centering
    \includegraphics[width=0.2\textwidth]{figures/7-Construccion/MobaXterm.png}
    \caption{Logo de MobaXterm}
\end{figure}

\subsubsection{Postman}
Para probar las rutas de la API se ha utilizado Postman\cite{postman}, una herramienta que permite realizar peticiones HTTP a servidores y analizar las respuestas.
De esta manera se ha comprobado el funcionamiento de la API sin necesidad de utilizar el \textit{frontend}, lo que ha facilitado el desarrollo y la depuración de la aplicación.

\begin{figure}[H]
    \centering
    \includegraphics[width=0.2\textwidth]{figures/7-Construccion/Postman.png}
    \caption{Logo de Postman}
\end{figure}


\subsubsection{Documentación}

\subsubsubsection{Memoria del proyecto}
La documentación del proyecto se ha realizado con LaTeX\cite{latex}, un sistema de composición de textos que permite crear documentos de alta calidad tipográfica.
Se ha utilizado la herramienta Overleaf\cite{overleaf} para la visualización y edición de los documentos, ya que permite la colaboración en tiempo real y la exportación a PDF.


\begin{figure}[H]
    \centering
    \begin{minipage}{0.2\textwidth}
        \centering
        \includegraphics[width=\textwidth]{figures/7-Construccion/LaTeX.png}
        \caption{Logo de LaTeX}
    \end{minipage}
    \hfill
    \begin{minipage}{0.2\textwidth}
        \centering
        \includegraphics[width=\textwidth]{figures/7-Construccion/Overleaf.png}
        \caption{Logo de Overleaf}
    \end{minipage}
\end{figure}


\subsubsubsection{Diagramas}
Para la creación de diagramas se han utilizado diversas herramientas, como Lucidchart\cite{lucidchart} y Draw.io\cite{drawio}, que permiten crear diagramas de forma sencilla y exportarlos en varios formatos.
Se ha utilizado PlantUML\cite{plantuml} para la creación de diagramas de forma textual, lo que facilita la creación y modificación de los diagramas. 
Por último, se ha utilizado Excalidraw\cite{excalidraw} para la creación de las interfaces de usuario de la aplicación.

\begin{figure}[H]
    \centering
    \begin{minipage}{0.2\textwidth}
        \centering
        \includegraphics[width=\textwidth]{figures/7-Construccion/Lucidchart.png}
        \caption{Logo de Lucidchart}
    \end{minipage}
    \hfill
    \begin{minipage}{0.2\textwidth}
        \centering
        \includegraphics[width=\textwidth]{figures/7-Construccion/Drawio.png}
        \caption{Logo de Draw.io}
    \end{minipage}
\end{figure}

\begin{figure}[H]
    \centering
    \begin{minipage}{0.2\textwidth}
        \centering
        \includegraphics[width=\textwidth]{figures/7-Construccion/PlantUML.png}
        \caption{Logo de PlantUML}
    \end{minipage}
    \hfill
    \begin{minipage}{0.2\textwidth}
        \centering
        \includegraphics[width=\textwidth]{figures/7-Construccion/Excalidraw.png}
        \caption{Logo de Excalidraw}
    \end{minipage}
\end{figure}


\subsubsubsection{Hojas de cálculo}
Para la creación de hojas de cálculo se ha utilizado Microsoft Excel\cite{excel}, una herramienta de Microsoft Office que permite crear y editar hojas de cálculo.
Se ha utilizado principalemente para la elaboración de presupuestos y cálculos de costes.

\begin{figure}[H]
    \centering
    \includegraphics[width=0.2\textwidth]{figures/7-Construccion/Excel.png}
    \caption{Logo de Microsoft Excel}
\end{figure}

\subsubsubsection{Planificación}
Para la planificación del proyecto se ha utilizado Microsoft Project\cite{project}, una herramienta de Microsoft Office que permite crear diagramas de Gantt y planificar tareas y recursos.
Se ha utilizado para la planificación del proyecto y el seguimiento de las tareas.

\begin{figure}[H]
    \centering
    \includegraphics[width=0.2\textwidth]{figures/7-Construccion/Project.png}
    \caption{Logo de Microsoft Project}
\end{figure}



\newpage
\section{ELABORACIÓN DE LOS MANUALES DE USUARIO}

\subsection{Manual de Instalación} 

\subsection{Manual de Ejecución} 

\subsection{Manual de Usuario} 

\subsection{Manual del Programador}


\newpage
\section{CONSTRUCCIÓN DE LOS COMPONENTES Y PROCEDIMIENTOS DE MIGRACIÓN Y CARGA INICIAL DE DATOS}

