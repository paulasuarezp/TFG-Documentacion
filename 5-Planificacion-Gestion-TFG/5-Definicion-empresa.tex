En esta sección se presenta la definición del modelo empresarial que ejecutará el proyecto. 
Se especificarán los costos asociados al personal que integra la empresa, así como los costos de los materiales, licencias, infraestructura y 
otros gastos necesarios para llevar a cabo las diversas actividades empresariales. 
Adicionalmente, se proporcionará un análisis de la facturación anual y los beneficios obtenidos por la empresa.

\subsubsubsection{Personal de la empresa y sueldos}
La empresa cuenta con un total de 6 personas, en el apartado \coloredUnderline{\hyperlink{sec:5-OBS}{\ref*{sec:5-OBS} \nameref*{sec:5-OBS}}}
se detalla la estructura organizativa de la empresa.
A continuación, se especifica el sueldo bruto anual de cada uno de los perfiles de la empresa, así como el coste total anual que supone el personal de la empresa.


\begin{longtable}{
    >{\raggedright\arraybackslash}p{4cm}
    >{\centering\arraybackslash}p{1.5cm}
    >{\centering\arraybackslash}p{3cm}
    >{\centering\arraybackslash}p{3cm}
    >{\centering\arraybackslash}p{3cm} }
    \caption{Costes salariales del proyecto} \label{table:costes-salariales} 
    \hypertarget{table:costes-salariales}{}
    \\

    \toprule
    \rowcolor{darkgreen!50}
    \textbf{Personal} & \textbf{Número} & \textbf{Sueldo Bruto Anual} & \textbf{Coste Salarial Anual} & \textbf{TOTAL} \\
    \midrule
    \endfirsthead

    \toprule
    \rowcolor{darkgreen!50}
    \textbf{Personal} & \textbf{Número} & \textbf{Sueldo Bruto Anual} & \textbf{Coste Salarial Anual} & \textbf{TOTAL} \\
    \midrule
    \endhead

    \midrule
    \multicolumn{5}{r}{{Continúa en la siguiente página\ldots}} \\
    \endfoot

    \bottomrule
    \endlastfoot

    \rowcolor{lightgreen!20}
    Jefe de proyecto & 1 & 30.000,00€ & 38.400,00€ & 38.400,00€ \\
    \midrule
    Analista junior & 1 & 23.000,00€ & 29.440,00€ & 29.440,00€ \\
    \midrule
    \rowcolor{lightgreen!20}
    Diseñador UX/UI junior & 1 & 21.000,00€ & 26.880,00€ & 26.880,00€ \\
    \midrule
    Desarrollador Software junior & 1 & 22.000,00€ & 28.160,00€ & 28.160,00€ \\
    \midrule
    \rowcolor{lightgreen!20}
    Tester junior & 1 & 21.000,00€ & 26.880,00€ & 26.880,00€ \\
    \midrule
    Documentador técnico & 1 & 24.500,00€ & 31.360,00€ & 31.360,00€ \\
    \midrule
    \rowcolor{lightgreen!30}
    \textbf{TOTAL} & \textbf{6} &  &  & \textbf{181.120,00€} \\
\end{longtable}


\subsubsubsection{Costes debidos a trabajos no productivos}

En la tabla \coloredUnderline{\hyperlink{table:costes-directos-indirectos}{\ref*{table:costes-directos-indirectos} \nameref*{table:costes-directos-indirectos}}} se detalla la asignación de los costes 
salariales a costes directos e indirectos. Esta asignación se debe a que no todo el tiempo de trabajo de los empleados se dedica a tareas productivas, 
también se dedica tiempo a tareas no productivas, como reuniones, formación, etc. 

La productividad se define como el porcentaje de tiempo que se dedica a tareas productivas y se calcula mediante la fórmula:

\[
\text{Productividad (R)} = \frac{\text{Cantidad de producto obtenido}}{\text{Cantidad de recurso utilizado (R)}}
\]

De esta manera, se pueden distinguir claramente los costes directos, asociados al tiempo efectivamente empleado en la producción, y los costes indirectos, correspondientes al tiempo dedicado a actividades no productivas.



\begin{longtable}{
    >{\raggedright\arraybackslash}p{4cm}
    >{\centering\arraybackslash}p{3cm}
    >{\centering\arraybackslash}p{2cm}
    >{\centering\arraybackslash}p{3cm}
    >{\centering\arraybackslash}p{2cm}
    >{\centering\arraybackslash}p{3cm} }
    \caption{Asignación de los costes salariales a costes directos e indirectos} \label{table:costes-directos-indirectos} 
    \hypertarget{table:costes-directos-indirectos}{}
    \\

    \toprule
    \rowcolor{darkgreen!50}
    \textbf{Personal} & \textbf{TOTAL} & \textbf{Prod (\%)} & \textbf{Coste Directo} & \textbf{CI (\%)} & \textbf{Coste Indirecto} \\
    \midrule
    \endfirsthead

    \toprule
    \rowcolor{darkgreen!50}
    \textbf{Personal} & \textbf{TOTAL} & \textbf{Prod (\%)} & \textbf{Coste Directo} & \textbf{CI (\%)} & \textbf{Coste Indirecto} \\
    \midrule
    \endhead

    \midrule
    \multicolumn{6}{r}{{Continúa en la siguiente página\ldots}} \\
    \endfoot

    \bottomrule
    \endlastfoot

    \rowcolor{lightgreen!20}
    Jefe de proyecto & 38.400,00€ & 0\% & 0,00€ & 100\% & 38.400,00€ \\
    \midrule
    Analista junior & 29.440,00€ & 85\% & 25.024,00€ & 15\% & 4.416,00€ \\
    \midrule
    \rowcolor{lightgreen!20}
    Diseñador UX/UI junior & 26.880,00€ & 85\% & 22.848,00€ & 15\% & 4.032,00€ \\
    \midrule
    Desarrollador Software junior & 28.160,00€ & 85\% & 23.936,00€ & 15\% & 4.224,00€ \\
    \midrule
    \rowcolor{lightgreen!20}
    Tester junior & 26.880,00€ & 85\% & 22.848,00€ & 15\% & 4.032,00€ \\
    \midrule
    Documentador técnico & 31.360,00€ & 90\% & 28.224,00€ & 10\% & 3.136,00€ \\
    \midrule
    \rowcolor{lightgreen!30}
    \textbf{TOTAL} & \textbf{181.120,00€} &  & \textbf{122.880,00€} &  & \textbf{58.240,00€} \\
\end{longtable}


Calculando las horas productivas por perfil se obtiene la tabla \coloredUnderline{\hyperlink{table:horas-productivas}{\ref*{table:horas-productivas} \nameref*{table:horas-productivas}}},
donde se obtiene el número de horas productivas por perfil y el total de horas productivas de la empresa.


\begin{longtable}{
    >{\raggedright\arraybackslash}p{4cm}
    >{\centering\arraybackslash}p{1cm}
    >{\centering\arraybackslash}p{2cm}
    >{\centering\arraybackslash}p{2cm}
    >{\centering\arraybackslash}p{3cm}
    >{\centering\arraybackslash}p{3cm} }
    \caption{Número de horas productivas por perfil y en total} \label{table:horas-productivas} 
    \hypertarget{table:horas-productivas}{}
    \\

    \toprule
    \rowcolor{darkgreen!50}
    \textbf{Personal} & \textbf{Número} & \textbf{Productividad (\%)} & \textbf{Horas / año} & \textbf{Horas productivas / año (Por persona)} & \textbf{Horas productivas (Total empresa)} \\
    \midrule
    \endfirsthead

    \toprule
    \rowcolor{darkgreen!50}
    \textbf{Personal} & \textbf{Número} & \textbf{Productividad (\%)} & \textbf{Horas / año} & \textbf{Horas productivas / año (Por persona)} & \textbf{Horas productivas (Total empresa)} \\
    \midrule
    \endhead

    \midrule
    \multicolumn{6}{r}{{Continúa en la siguiente página\ldots}} \\
    \endfoot

    \bottomrule
    \endlastfoot

    \rowcolor{lightgreen!20}
    Jefe de proyecto & 1 & 0\% & 2032 & 0 & 0 \\
    \midrule
    Analista junior & 1 & 80\% & 2032 & 1625,6 & 1625,6 \\
    \midrule
    \rowcolor{lightgreen!20}
    Diseñador UX/UI junior & 1 & 85\% & 2032 & 1727,2 & 1727,2 \\
    \midrule
    Desarrollador Software junior & 1 & 85\% & 2032 & 1727,2 & 1727,2 \\
    \midrule
    \rowcolor{lightgreen!20}
    Tester junior & 1 & 80\% & 2032 & 1625,6 & 1625,6 \\
    \midrule
    Documentador técnico & 1 & 90\% & 2032 & 1828,8 & 1828,8 \\
    \midrule
    \rowcolor{lightgreen!30}
    \textbf{TOTAL} & \textbf{6} &  &  &  & \textbf{8534,4} \\
\end{longtable}


\subsubsubsection{Costes de servicios}
Es necesario calcular los costes de los servicios inherentes a la empresa, estos costes se detallan en la tabla \coloredUnderline{\hyperlink{table:costes-servicios}{\ref*{table:costes-servicios} \nameref*{table:costes-servicios}}}.
Estos costes son aquellos que no están directamente relacionados con la producción, pero que son necesarios para el funcionamiento de la empresa.

\begin{longtable}{
    >{\raggedright\arraybackslash}p{7cm}
    >{\centering\arraybackslash}p{3cm}
    >{\centering\arraybackslash}p{3cm} }
    \caption{Costes de servicios} \label{table:costes-servicios} 
    \hypertarget{table:costes-servicios}{}
    \\

    \toprule
    \rowcolor{darkgreen!50}
    \textbf{Servicio} & \textbf{Coste mes} & \textbf{Coste anual} \\
    \midrule
    \endfirsthead

    \toprule
    \rowcolor{darkgreen!50}
    \textbf{Servicio} & \textbf{Coste mes} & \textbf{Coste anual} \\
    \midrule
    \endhead

    \midrule
    \multicolumn{3}{r}{{Continúa en la siguiente página\ldots}} \\
    \endfoot

    \bottomrule
    \endlastfoot

    \rowcolor{lightgreen!20}
    Alquiler de inmueble & 800,00€ & 9.600,00€ \\
    \midrule
    Tributos y tasas diversas & 650,00€ & 7.800,00€ \\
    \midrule
    \rowcolor{lightgreen!20}
    Limpieza & 500,00€ & 6.000,00€ \\
    \midrule
    Consumos de agua & 80,00€ & 960,00€ \\
    \midrule
    \rowcolor{lightgreen!20}
    Consumos de electricidad (excepto consumos para producción) & 130,00€ & 1.560,00€ \\
    \midrule
    Honorarios de asesorías, auditorías y otros profesionales & 150,00€ & 1.800,00€ \\
    \midrule
    \rowcolor{lightgreen!20}
    Primas de seguros & 600,00€ & 7.200,00€ \\
    \midrule
    Gastos en material de oficina & 50,00€ & 600,00€ \\
    \midrule
    \rowcolor{lightgreen!20}
    Gastos financieros & 200,00€ & 2.400,00€ \\
    \midrule
    \rowcolor{lightgreen!30}
    \textbf{TOTAL} & \textbf{3.160,00€} & \textbf{37.920,00€} \\
\end{longtable}


\subsubsubsection{Costes de los medios de producción}
Por otro lado, es necesario calcular los costes de los medios de producción, estos costes se detallan en la tabla \coloredUnderline{\hyperlink{table:costes-medios-produccion}{\ref*{table:costes-medios-produccion} \nameref*{table:costes-medios-produccion}}}.
Estos costes son aquellos que están directamente relacionados con la producción, como los costes de los equipos informáticos, licencias de software, etc.

Estos costes se han clasificado en dos tipos: costes de amortización y costes de alquiler. Los costes de amortización son aquellos que se pagan de una sola vez y se amortizan a lo largo de varios años, 
mientras que los costes de alquiler son aquellos que se pagan de forma periódica. Para los costes de amortización se ha calculado el coste anual de amortización y el plazo de amortización en años.

\begin{longtable}{
    >{\raggedright\arraybackslash}p{3cm}
    >{\raggedright\arraybackslash}p{4cm}
    >{\centering\arraybackslash}p{1cm}
    >{\centering\arraybackslash}p{2cm}
    >{\centering\arraybackslash}p{3cm}
    >{\centering\arraybackslash}p{2cm}
    >{\centering\arraybackslash}p{2cm}
    >{\centering\arraybackslash}p{2cm} }
    \caption{Costes de los medios de producción} \label{table:costes-medios-produccion} 
    \hypertarget{table:costes-medios-produccion}{}
    \\

    \toprule
    \rowcolor{darkgreen!50}
    \textbf{Equipo / Licencia} & \textbf{Características} & \textbf{ud.} & \textbf{Precio} & \textbf{Coste Total} & \textbf{Coste Año} & \textbf{Tipo} & \textbf{Plazo} \\
    \midrule
    \endfirsthead

    \toprule
    \rowcolor{darkgreen!50}
    \textbf{Equipo / Licencia} & \textbf{Características} & \textbf{ud.} & \textbf{Precio} & \textbf{Coste Total} & \textbf{Coste Año} & \textbf{Tipo} & \textbf{Plazo} \\
    \midrule
    \endhead

    \midrule
    \multicolumn{8}{r}{{Continúa en la siguiente página\ldots}} \\
    \endfoot

    \bottomrule
    \endlastfoot

    \rowcolor{lightgreen!20}
    Portátiles & MacBook Pro 16 pulgadas Chip M3 Max de Apple con CPU de 14 núcleos, GPU de 30 núcleos y Neural Engine de 16 núcleos, 36GB memoria unificada, 1TB de almacenamiento SSD & 1 & 4.299,00€ & 4.299,00€ & 1.074,75€ & Amortización & 4 \\
    \midrule
    Portátiles & Portátil HP Pavilion Plus 16-ab0004ns, Windows 11 Home, Intel\textregistered{} Core\texttrademark{} i7 13700H (13.ª generación), 16 GB RAM, 1 TB SSD, 16 pulgadas, WQXGA (2560 x 1600), 120 Hz, NVIDIA\textregistered{} GeForce RTX\texttrademark{} 3050 (6 GB) & 5 & 1.499,00€ & 7.495,00€ & 1.873,75€ & Amortización & 4 \\
    \midrule
    \rowcolor{lightgreen!20}
    Licencia Overleaf & Licencia \textit{Group Standard} para 10 usuarios & 1 & 1.160,00€ & 1.160,00€ & 1.160,00€ & Alquiler & - \\
    \midrule
    Licencia Excalidraw & Licencia Anual & 2 & 72,00€ & 144,00€ & 144,00€ & Alquiler & - \\
    \midrule
    \rowcolor{lightgreen!20}
    Licencia Office 365 & Microsoft 365 Plan Empresa Premium & 6 & 247,20€ & 1.483,20€ & 1.483,20€ & Alquiler & - \\
    \midrule
    Licencia MS Project & Compra de pago único Project Profesional 2021 & 1 & 1.659,00€ & 1.659,00€ & 331,80€ & Amortización & 5 \\
    \midrule
    \rowcolor{lightgreen!20}
    Conexión a internet & Fibra 600 MB con centralita virtual & 1 & 1.013,76€ & 1.013,76€ & 1.013,76€ & Alquiler & - \\
    \midrule
    \rowcolor{lightgreen!30}
    \textbf{TOTAL} &  &  &  & \textbf{7.081,26€} &  &  &  \\
\end{longtable}



\subsubsubsection{Precio por hora de trabajo y facturación total de la empresa}
En la tabla \coloredUnderline{\hyperlink{table:precios-facturacion}{\ref*{table:precios-facturacion} \nameref*{table:precios-facturacion}}} se detalla el precio por hora de trabajo de cada uno de los perfiles de la empresa,
así como el total de horas productivas de la empresa y la facturación total de la empresa.

Este precio por hora permite cubrir los costes salariales, los costes de servicios y los costes de los medios de producción, así como obtener un beneficio para la empresa.

La facturación total de la empresa se calcula multiplicando el precio por hora de trabajo de cada perfil por el número de horas productivas de cada perfil y sumando el total de horas productivas de la empresa,
con lo que se obtiene que la empresa facturará un total de 230.530,40€ al año.

\begin{longtable}{
    >{\raggedright\arraybackslash}p{5cm}
    >{\centering\arraybackslash}p{3cm}
    >{\centering\arraybackslash}p{4cm}
    >{\centering\arraybackslash}p{4cm} }
    \caption{Precios por hora de trabajo y facturación total de la empresa} \label{table:precios-facturacion} 
    \hypertarget{table:precios-facturacion}{}
    \\

    \toprule
    \rowcolor{darkgreen!50}
    \textbf{Personal} & \textbf{Precio/hora} & \textbf{Horas productivas (Total empresa)} & \textbf{Facturación} \\
    \midrule
    \endfirsthead

    \toprule
    \rowcolor{darkgreen!50}
    \textbf{Personal} & \textbf{Precio/hora} & \textbf{Horas productivas (Total empresa)} & \textbf{Facturación} \\
    \midrule
    \endhead

    \midrule
    \multicolumn{4}{r}{{Continúa en la siguiente página\ldots}} \\
    \endfoot

    \bottomrule
    \endlastfoot

    \rowcolor{lightgreen!20}
    Jefe de proyecto & 40,00€ & 0,00 & 0,00€ \\
    \midrule
    Analista junior & 30,00€ & 1625,60 & 48.768,00€ \\
    \midrule
    \rowcolor{lightgreen!20}
    Diseñador UX/UI junior & 26,00€ & 1727,20 & 44.907,20€ \\
    \midrule
    Desarrollador Software junior & 25,00€ & 1727,20 & 43.180,00€ \\
    \midrule
    \rowcolor{lightgreen!20}
    Tester junior & 25,00€ & 1625,60 & 40.640,00€ \\
    \midrule
    Documentador técnico & 29,00€ & 1828,80 & 53.035,20€ \\
    \midrule
    \rowcolor{lightgreen!30}
    \textbf{TOTAL} &  & \textbf{8534,40} & \textbf{230.530,40€} \\
\end{longtable}



\subsubsubsection{Precio por hora a usar para el cálculo de los proyectos de costes}
En la tabla \coloredUnderline{\hyperlink{table:precio-hora}{\ref*{table:precio-hora} \nameref*{table:precio-hora}}} se detalla el precio por hora a usar para el cálculo de los proyectos de costes.
Este es el precio real que la empresa tiene que pagar por hora de trabajo de cada uno de los perfiles de la empresa.


\begin{longtable}{
    >{\raggedright\arraybackslash}p{5cm}
    >{\centering\arraybackslash}p{3cm} }
    \caption{Precio por hora a usar para el cálculo de los proyectos de costes} \label{table:precio-hora} 
    \hypertarget{table:precio-hora}{}
    \\

    \toprule
    \rowcolor{darkgreen!50}
    \textbf{Personal} & \textbf{Precio/hora} \\
    \midrule
    \endfirsthead

    \toprule
    \rowcolor{darkgreen!50}
    \textbf{Personal} & \textbf{Precio/hora} \\
    \midrule
    \endhead

    \midrule
    \multicolumn{2}{r}{{Continúa en la siguiente página\ldots}} \\
    \endfoot

    \bottomrule
    \endlastfoot

    \rowcolor{lightgreen!20}
    Jefe de proyecto & 18,90€ \\
    \midrule
    Analista junior & 14,49€ \\
    \midrule
    \rowcolor{lightgreen!20}
    Diseñador UX/UI junior & 13,23€ \\
    \midrule
    Desarrollador Software junior & 13,86€ \\
    \midrule
    \rowcolor{lightgreen!20}
    Tester junior & 13,23€ \\
    \midrule
    Documentador técnico & 15,43€ \\
\end{longtable}



\subsubsubsection{Resumen del modelo de empresa}
Tras los cálculos realizados, se obtiene el resumen del modelo de empresa, que se detalla en la tabla \coloredUnderline{\hyperlink{table:resumen-modelo-empresa}{\ref*{table:resumen-modelo-empresa} \nameref*{table:resumen-modelo-empresa}}}.
En esta tabla se detalla el total de los costes directos, el total de los costes indirectos, la suma de los costes directos e indirectos, el beneficio deseado, el coste total, 
la facturación posible en función de las horas de producción y de los precios por hora calculados, y el margen entre el coste total y la facturación.

La empresa obtiene un margen del 2,034\% entre el coste total y la facturación, lo que indica que la empresa obtiene un beneficio del 2,034\% sobre el coste total.

\begin{longtable}{
    >{\centering\arraybackslash}p{1cm}
    >{\raggedright\arraybackslash}p{8cm}
    >{\centering\arraybackslash}p{4cm} }
    \caption{Resumen del modelo de empresa} \label{table:resumen-modelo-empresa} 
    \hypertarget{table:resumen-modelo-empresa}{}
    \\

    \toprule
    \rowcolor{darkgreen!50}
    \textbf{Nº} & \textbf{Concepto} & \textbf{Importe} \\
    \midrule
    \endfirsthead

    \toprule
    \rowcolor{darkgreen!50}
    \textbf{Nº} & \textbf{Concepto} & \textbf{Importe} \\
    \midrule
    \endhead

    \midrule
    \multicolumn{3}{r}{{Continúa en la siguiente página\ldots}} \\
    \endfoot

    \bottomrule
    \endlastfoot

    \rowcolor{lightgreen!20}
    1 & Total de los costes directos & 122.880,00€ \\
    \midrule
    2 & Total de los costes indirectos & 65.321,26€ \\
    \midrule
    \rowcolor{lightgreen!20}
    3 & Suma de los costes directos e indirectos & 188.201,26€ \\
    \midrule
    4 & Beneficio deseado (20\%) & 37.640,25€ \\
    \midrule
    \rowcolor{lightgreen!20}
    5 & Coste total (suma de los costes directos, indirectos y beneficios) & 225.841,51€ \\
    \midrule
    6 & Facturación posible en función de las horas de producción y de los precios por hora calculados & 230.530,40€ \\
    \midrule
    \rowcolor{lightgreen!20}
    7 & Margen entre el coste total y la facturación (relación entre 5 y 6) & 2,034\% \\
\end{longtable}
