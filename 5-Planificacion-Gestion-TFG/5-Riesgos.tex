En esta sección, se detallarán los riesgos identificados para el proyecto y cómo serán abordados. Según el PMBOK:
\begin{quote}
Un riesgo es un evento o condición incierta que, si sucede, tiene un efecto positivo o negativo en por lo menos uno de los objetivos del proyecto, tales como el alcance, el cronograma, el coste y la calidad. \cite{pmbok2013}
\end{quote}

\subsubsection{Plan de Gestión de Riesgos} 
El proyecto sigue una estrategia proactiva frente al riesgo, es decir, se han evaluado los riesgos inherentes al proyecto antes de que estos se produzcan con el objetivo de prevenirlos. El procedimiento realizado para identificar y evaluar los riesgos se detalla en el \refLink{plan_de_gestion_de_riesgos}.

\subsubsection{Identificación de Riesgos}


\subsubsection{Registro de Riesgos}

\begin{table}[htb]
    \centering
    \caption{Análisis de riesgo}
    \label{table:risk_analysis}
    \begin{tabular}{>{\columncolor{rowcolor}}l l l}
    \toprule
    \rowcolor{lightgreen}
    \textbf{Identificador} & \multicolumn{2}{l}{1} \\
    \midrule
    \textbf{Nombre} & \multicolumn{2}{l}{Problema de costos} \\
    \midrule
    \textbf{Descripción} & \multicolumn{2}{p{10cm}}{El costo de desarrollar y mantener un sitio web y un sistema de venta y distribución es alto, lo que puede afectar negativamente la rentabilidad de la empresa si no se generan suficientes ingresos para cubrir estos costos.} \\
    \midrule
    \textbf{Categoría} & \multicolumn{2}{l}{Riesgo de gestión} \\
    \midrule
    \textbf{Probabilidad} & \multicolumn{2}{l}{Media} \\
    \midrule
    \textbf{Impacto} & Presupuesto & Crítico \\
    \cmidrule(lr){2-3}
    & Planificación & Medio \\
    \cmidrule(lr){2-3}
    & Alcance & Alto \\
    \cmidrule(lr){2-3}
    & Calidad & Alto \\
    \cmidrule(lr){2-3}
    & Total & 0.45 \\
    \midrule
    \textbf{Respuesta} & \multicolumn{2}{p{10cm}}{Realizar un análisis financiero riguroso para determinar el costo total del proyecto, incluyendo los costos de desarrollo, mantenimiento, actualizaciones y otros gastos asociados. Establecer proyecciones financieras realistas para garantizar que la empresa pueda generar suficientes ingresos para cubrir estos costos y obtener una rentabilidad adecuada.} \\
    \midrule
    \textbf{Estrategia} & \multicolumn{2}{l}{Mitigar el riesgo} \\
    \bottomrule
    \end{tabular}
    \end{table}
