En esta sección, se detallarán los riesgos identificados para el proyecto y cómo serán abordados. Según el PMBOK:
\begin{quote}
Un riesgo es un evento o condición incierta que, si sucede, tiene un efecto positivo o negativo en por lo menos uno de los objetivos del proyecto, tales como el alcance, el cronograma, el coste y la calidad. \cite{pmbok2013}
\end{quote}

\subsubsection{Plan de Gestión de Riesgos} 
El proyecto sigue una estrategia proactiva frente al riesgo, es decir, se han evaluado los riesgos inherentes al proyecto antes de que estos se produzcan con el objetivo de prevenirlos. El procedimiento realizado para identificar y evaluar los riesgos se detalla en el \coloredUnderline{\hyperlink{anexo:plan_de_gestion_de_riesgos}{Plan de gestión de riesgos}}.


\subsubsection{Identificación de Riesgos}
Se han identificado un total de 12 riesgos que afectan al proyecto en distintos ámbitos. 
\begin{enumerate}
    \item Seguridad de la información.
    \item Problemas de accesibilidad.
    \item Problemas de usabilidad.
    \item Cambios de licencia en el software.
    \item Problemas de rendimiento.
    \item Intento de fraude por parte de los usuarios finales.
    \item Falta de comunicación con el tutor del TFG.
    \item Errores en las estimaciones de tareas.
    \item Problemas de compatibilidad horaria.
    \item Inexperiencia con Sockets.
    \item Averías del equipo de trabajo.
    \item Problemas con la Aceptación del Cliente.
\end{enumerate}

\subsubsection{Registro de Riesgos}
\begin{enumerate}
    \item Seguridad de la información.
    \begin{itemize}
        \item \textbf{Descripción:} La seguridad de la información se refiere a la protección de datos contra accesos no autorizados, 
        alteraciones, robos o eliminaciones, tanto accidentalmente como de manera intencional. 
        La falta de seguridad en el sistema puede resultar en la pérdida de datos, daños a la reputación de la empresa y sanciones legales. 
        \item \textbf{Probabilidad:} Media
        \item \textbf{Impacto:} Crítico
        \item \textbf{Estrategia:} Mitigar el riesgo
    \end{itemize}
    \item Problemas de accesibilidad.
    \begin{itemize}
        \item \textbf{Descripción:} Si un sitio web no está diseñado para ser accesible a personas con discapacidades, 
        se podría excluir a un segmento significativo de usuarios potenciales.
        \item \textbf{Probabilidad:} Baja
        \item \textbf{Impacto:} Medio
        \item \textbf{Estrategia:} Mitigar el riesgo
    \end{itemize}
    \item Problemas de usabilidad.
    \begin{itemize}
        \item \textbf{Descripción:} Una interfaz de usuario no intuitiva puede resultar en una curva de aprendizaje empinada, 
        frustración por parte de los usuarios y, en consecuencia, una menor adopción del producto.
        \item \textbf{Probabilidad:} Media
        \item \textbf{Impacto:} Alto
        \item \textbf{Estrategia:} Mitigar el riesgo
    \end{itemize}
    \item Cambios de licencia en el software.
    \begin{itemize}
        \item \textbf{Descripción:} Cambios en la licencia de software pueden afectar la disponibilidad, problemas legales, costos incrementados o la necesidad de buscar alternativas de software.
        \item \textbf{Probabilidad:} Baja
        \item \textbf{Impacto:} Medio
        \item \textbf{Estrategia:} Asumir el riesgo
    \end{itemize}
    \item Problemas de rendimiento.
    \begin{itemize}
        \item \textbf{Descripción:} Un sistema que no puede manejar la carga de trabajo esperada puede resultar en una experiencia de usuario deficiente y,
         en última instancia, en la pérdida de clientes.
        \item \textbf{Probabilidad:} Media
        \item \textbf{Impacto:} Alto
        \item \textbf{Estrategia:} Mitigar el riesgo
    \end{itemize}
    \item Intento de fraude por parte de los usuarios finales.
    \begin{itemize}
        \item \textbf{Descripción:} Los intentos de fraude, como la creación de cuentas falsas o transacciones fraudulentas, pueden tener un impacto financiero y de reputación significativo
        \item \textbf{Probabilidad:} Media
        \item \textbf{Impacto:} Alto
        \item \textbf{Estrategia:} Mitigar el riesgo
    \end{itemize}
    \item Falta de comunicación con el tutor del TFG.
    \begin{itemize}
        \item \textbf{Descripción:} En proyectos académicos como un TFG, una comunicación inadecuada con el tutor puede llevar a desviaciones del objetivo del proyecto, 
        retrasos y resultados que no cumplen con las expectativas académicas.
        \item \textbf{Probabilidad:} Baja
        \item \textbf{Impacto:} Alto
        \item \textbf{Estrategia:} Mitigar el riesgo
    \end{itemize}
    \item Errores en las estimaciones de tareas.
    \begin{itemize}
        \item \textbf{Descripción:} Las estimaciones incorrectas de la duración de las tareas pueden resultar en retrasos en el proyecto y
         en la necesidad de reajustar el cronograma.
        \item \textbf{Probabilidad:} Media
        \item \textbf{Impacto:} Medio
        \item \textbf{Estrategia:} Mitigar el riesgo
    \end{itemize}
    \item Conciliación entre responsabilidades académicas y laborales.
    \begin{itemize}
        \item \textbf{Descripción:} La conciliación entre las responsabilidades académicas y laborales puede reducir significativamente la disponibilidad de tiempo para dedicar al proyecto. 
        Esto, a su vez, puede disminuir la productividad y eventualmente provocar retrasos en la entrega del proyecto.
        \item \textbf{Probabilidad:} Media
        \item \textbf{Impacto:} Medio
        \item \textbf{Estrategia:} Mitigar el riesgo
    \end{itemize}
    \item Inexperiencia con Sockets.
    \begin{itemize}
        \item \textbf{Descripción:} La falta de experiencia con Sockets puede resultar en problemas de implementación y en la necesidad de dedicar más tiempo a la resolución de problemas.
        \item \textbf{Probabilidad:} Media
        \item \textbf{Impacto:} Medio
        \item \textbf{Estrategia:} Mitigar el riesgo
    \end{itemize}
    \item Fallos en el hardware o software.
    \begin{itemize}
        \item \textbf{Descripción:} Problemas con el hardware o software crítico para el proyecto pueden resultar en pérdida de datos importantes, retrasos en el proyecto y aumento en los costos.
        \item \textbf{Probabilidad:} Baja
        \item \textbf{Impacto:} Medio
        \item \textbf{Estrategia:} Mitigar el riesgo
    \end{itemize}
    \item Problemas con la Aceptación del Cliente.
    \begin{itemize}
        \item \textbf{Descripción:} Si el producto final no cumple con las expectativas o requisitos del cliente, en este caso el tribunal del TFG y el propio tutor, puede resultar en el rechazo del proyecto.
        \item \textbf{Probabilidad:} Media
        \item \textbf{Impacto:} Alto
        \item \textbf{Estrategia:} Mitigar el riesgo
    \end{itemize}
\end{enumerate}


\begin{table}[htb]
    \centering
    \caption{Análisis de riesgo}
    \label{table:risk_analysis}
    \begin{tabular}{>{\columncolor{rowcolor}}l l l}
    \toprule
    \rowcolor{lightgreen}
    \textbf{Identificador} & \multicolumn{2}{l}{1} \\
    \midrule
    \textbf{Nombre} & \multicolumn{2}{l}{Problema de costos} \\
    \midrule
    \textbf{Descripción} & \multicolumn{2}{p{10cm}}{El costo de desarrollar y mantener un sitio web y un sistema de venta y distribución es alto, lo que puede afectar negativamente la rentabilidad de la empresa si no se generan suficientes ingresos para cubrir estos costos.} \\
    \midrule
    \textbf{Categoría} & \multicolumn{2}{l}{Riesgo de gestión} \\
    \midrule
    \textbf{Probabilidad} & \multicolumn{2}{l}{Media} \\
    \midrule
    \textbf{Impacto} & Presupuesto & Crítico \\
    \cmidrule(lr){2-3}
    & Planificación & Medio \\
    \cmidrule(lr){2-3}
    & Alcance & Alto \\
    \cmidrule(lr){2-3}
    & Calidad & Alto \\
    \cmidrule(lr){2-3}
    & Total & 0.45 \\
    \midrule
    \textbf{Respuesta} & \multicolumn{2}{p{10cm}}{Realizar un análisis financiero riguroso para determinar el costo total del proyecto, incluyendo los costos de desarrollo, mantenimiento, actualizaciones y otros gastos asociados. Establecer proyecciones financieras realistas para garantizar que la empresa pueda generar suficientes ingresos para cubrir estos costos y obtener una rentabilidad adecuada.} \\
    \midrule
    \textbf{Estrategia} & \multicolumn{2}{l}{Mitigar el riesgo} \\
    \bottomrule
    \end{tabular}
    \end{table}
