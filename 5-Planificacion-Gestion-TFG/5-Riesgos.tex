En esta sección, se detallarán los riesgos identificados para el proyecto y cómo serán abordados. Según el PMBOK:
\begin{quote}
Un riesgo es un evento o condición incierta que, si sucede, tiene un efecto positivo o negativo en por lo menos uno de los objetivos del proyecto, tales como el alcance, el cronograma, el coste y la calidad. \cite{pmbok2013}
\end{quote}

\subsubsection{Plan de Gestión de Riesgos} 
El proyecto sigue una estrategia proactiva frente al riesgo, es decir, se han evaluado los riesgos inherentes al proyecto antes de que estos se produzcan con el objetivo de prevenirlos. El procedimiento realizado para identificar y evaluar los riesgos se detalla en el \coloredUnderline{\hyperlink{anexo:plan_de_gestion_de_riesgos}{Plan de gestión de riesgos}}.


\subsubsection{Identificación de Riesgos}\label{Risks:identificacion_riesgos}
\hypertarget{Risks:identificacion_riesgos}{}
Se han identificado un total de 12 riesgos que afectan al proyecto en distintos ámbitos. 
\begin{enumerate}
    \item Seguridad de la información.
    \item Problemas de accesibilidad.
    \item Problemas de usabilidad.
    \item Cambios de licencia en el software.
    \item Problemas de rendimiento.
    \item Intento de fraude por parte de los usuarios finales.
    \item Falta de comunicación con el tutor del TFG.
    \item Errores en las estimaciones de tareas.
    \item Problemas de compatibilidad horaria.
    \item Inexperiencia con Sockets.
    \item Averías del equipo de trabajo.
    \item Problemas con la Aceptación del Cliente.
\end{enumerate}

\subsubsection{Registro de Riesgos}
Se han clasificado los riesgos identificados en cuatro categorías:
\begin{itemize}
    \item \textbf{Riesgos Técnicos.} 
    \item \textbf{Riesgos Organizacionales.} Riesgos relacionados con la planificación, organización y control del proyecto. También incluye riesgos relacionados con el análisis de normativas y leyes aplicables al proyecto.
    \item \textbf{Riesgos de Recursos.} Riesgos relacionados con la disponibilidad y gestión de los recursos necesarios para el proyecto.
    \item \textbf{Riesgos de Proyecto.} Riesgos directamente relacionados con el éxito del proyecto, principalmente enfocados en el cliente y en la aceptación del producto final.
\end{itemize}

\begin{enumerate}
    \item Problemas de accesibilidad.
    \begin{itemize}
        \item \textbf{Descripción:} Si un sitio web no está diseñado para ser accesible a personas con discapacidades, 
        se podría excluir a un segmento significativo de usuarios potenciales.
        \item \textbf{Probabilidad:} Baja
        \item \textbf{Impacto:} Medio
        \item \textbf{Estrategia:} Mitigar el riesgo
    \end{itemize}
    \item Problemas de usabilidad.
    \begin{itemize}
        \item \textbf{Descripción:} Una interfaz de usuario no intuitiva puede resultar en una curva de aprendizaje empinada, 
        frustración por parte de los usuarios y, en consecuencia, una menor adopción del producto.
        \item \textbf{Probabilidad:} Media
        \item \textbf{Impacto:} Alto
        \item \textbf{Estrategia:} Mitigar el riesgo
    \end{itemize}
    
    \item Fallos en el hardware o software.
    \begin{itemize}
        \item \textbf{Descripción:} Problemas con el hardware o software crítico para el proyecto pueden resultar en pérdida de datos importantes, retrasos en el proyecto y aumento en los costos.
        \item \textbf{Probabilidad:} Baja
        \item \textbf{Impacto:} Medio
        \item \textbf{Estrategia:} Mitigar el riesgo
    \end{itemize}
  
    
\end{enumerate}


