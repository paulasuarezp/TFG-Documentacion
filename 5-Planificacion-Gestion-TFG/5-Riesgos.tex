En esta sección, se detallarán los riesgos identificados para el proyecto y cómo serán abordados. Según el PMBOK:
\begin{quote}
Un riesgo es un evento o condición incierta que, si sucede, tiene un efecto positivo o negativo en por lo menos uno de los objetivos del proyecto, tales como el alcance, el cronograma, el coste y la calidad. \cite{pmbok2013}
\end{quote}

\subsubsection{Plan de Gestión de Riesgos} 
El proyecto sigue una estrategia proactiva frente al riesgo, es decir, se han evaluado los riesgos inherentes al proyecto antes de que estos se produzcan con el objetivo de prevenirlos. El procedimiento realizado para identificar y evaluar los riesgos se detalla en el \coloredUnderline{\hyperlink{anexo:plan_de_gestion_de_riesgos}{Plan de gestión de riesgos}}.


\subsubsection{Identificación de Riesgos}\label{Risks:identificacion_riesgos}
\hypertarget{Risks:identificacion_riesgos}{}
Se han identificado un total de 12 riesgos que afectan al proyecto en distintos ámbitos. Estos riesgos, ordenados de mayor a menor prioridad, son los siguientes:
\begin{enumerate}
    \item Falta de comunicación con el tutor del TFG.
    \item Seguridad de la información.
    \item Intento de fraude por parte de los usuarios finales.
    \item Errores en las estimaciones de tareas.
    \item Conciliación entre las responsabilidades académicas y laborales.
    \item Problemas con la Aceptación del Cliente.
    \item Problemas de cumplimiento normativo y legal.
    \item Cambios de licencia en el software.
    \item Problemas de rendimiento.
    \item Fallos en el hardware o software.
    \item Problemas de usabilidad.
    \item Problemas de accesibilidad.
\end{enumerate}

\subsubsection{Registro de Riesgos}
Se ha llevado a cabo un análisis de los riesgos identificados en el proyecto, detallando la probabilidad de ocurrencia, el impacto en el proyecto, la prioridad y la estrategia de respuesta a cada riesgo. Este análisis se puede consultar en el anexo \coloredUnderline{\hyperlink{anexo:registro_de_riesgos}{Registro de riesgos}}.