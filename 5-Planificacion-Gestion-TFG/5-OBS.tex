En este proyecto, se simulará una pequeña empresa ficticia para su realización. 
Se han identificado los distintos perfiles que intervienen en el proyecto, así como las tareas que cada uno debe realizar. 
Aunque el proyecto se llevará a cabo de forma individual, se ha considerado necesario estructurar estos roles para poder asignar un presupuesto adecuado a cada función.

Los roles identificados se especifican en \coloredUnderline{\hyperlink{table:obs}{Tabla: \ref*{table:obs} \nameref*{table:obs}}}. 
Cabe destacar que el rol de Jefe de Proyecto (JP) es el encargado de supervisar y coordinar el trabajo de los demás roles y este rol ha sido considerado tanto para el alumno como para el tutor del TFG.


\begin{table}[H]
\centering
\hypertarget{table:obs}{}
\caption{OBS (Organizational Breakdown Structure)}
\label{table:obs}
\begin{tabular}{>{\columncolor{lightgreen!20}}p{7cm} p{10cm}}
\toprule
\rowcolor{darkgreen!50}
\textbf{Abreviatura} & \textbf{Rol} \\
\midrule
JP & Jefe de Proyecto \\
\midrule
AN & Analista junior\\
\midrule
DI & Diseñador junior \\
\midrule
DS & Desarrollador de Software junior\\
\midrule
TE & Tester junior \\
\midrule
DOC & Documentador técnico \\
\bottomrule
\end{tabular}
\end{table}
 
Se ha decidido realizar una matriz de asignación de responsabilidades RACI para relacionar las tareas identifcadas en \coloredUnderline{\hyperlink{sec:5-WBS}{\ref*{sec:5-WBS} \nameref*{sec:5-WBS}}} con cada rol.
En esta matriz, a cada tarea se le asigna un rol con una de las siguientes responsabilidades:
\begin{itemize}
    \item \textbf{R:} Responsable. Persona que realiza la tarea.
    \item \textbf{A:} Aprobador. Persona que aprueba la tarea.
    \item \textbf{C:} Consultado. Persona a la que se consulta sobre la tarea.
    \item \textbf{I:} Informado. Persona a la que se informa sobre el avance y los resultados de la tarea.
\end{itemize}
Un mismo recursos puede tener varias responsabilidades en una misma tarea, en tal caso se anotarán separadas por el caracter \textit{``/''}.

\subsubsubsection{Análisis del proyecto}
En la \coloredUnderline{\hyperlink{table:matriz-analisis}{Tabla: \ref*{table:matriz-analisis} \nameref*{table:matriz-analisis}}} se muestra la matriz de responsabilidades para la fase de análisis.
\begin{table}[H]
    \centering
    \caption{Matriz RACI. Análisis del proyecto}
    \label{table:matriz-analisis}
    \hypertarget{table:matriz-analisis}{}
    \begin{tabular}{
    >{\columncolor{lightgreen!20}}m{7cm} 
    >{\columncolor{white}}m{1cm} 
    >{\columncolor{white}}m{1cm} 
    >{\columncolor{white}}m{1cm} 
    >{\columncolor{white}}m{1cm} 
    >{\columncolor{white}}m{1cm} 
    >{\columncolor{white}}m{1cm}}
    \cmidrule(l){2-7}
    \rowcolor{darkgreen!50}
    \cellcolor{white} & \multicolumn{6}{c}{\textbf{Roles}} \\
    \midrule
    \rowcolor{lightgreen!20}
    \cellcolor{darkgreen!50}\textbf{Tarea} & \textbf{JP} & \textbf{AN} & \textbf{DI} & \textbf{DS} & \textbf{TE} & \textbf{DOC} \\
    \midrule
    Análisis del sistema & A & R &  & C &  &  \\
    \midrule
    Análisis de la arquitectura & A & R &  & C &  &  \\
    \midrule
    Análisis de la infraestructura & A & R &  & C &  &  \\
    \midrule
    Determinación del alcance de desarrollo & A & R & I & C & I & I \\
    \bottomrule
    \end{tabular}
\end{table}

\subsubsubsection{Seguimiento del proyecto}
En la \coloredUnderline{\hyperlink{table:matriz-seguimiento}{Tabla: \ref*{table:matriz-seguimiento} \nameref*{table:matriz-seguimiento}}} se muestra la matriz de responsabilidades para la fase de seguimiento.
\begin{table}[H]
    \centering
    \caption{Matriz RACI. Seguimiento del proyecto}
    \label{table:matriz-seguimiento}
    \hypertarget{table:matriz-seguimiento}{}
    \begin{tabular}{
    >{\columncolor{lightgreen!20}}m{7cm} 
    >{\columncolor{white}}m{1cm} 
    >{\columncolor{white}}m{1cm} 
    >{\columncolor{white}}m{1cm} 
    >{\columncolor{white}}m{1cm} 
    >{\columncolor{white}}m{1cm} 
    >{\columncolor{white}}m{1cm}}
    \cmidrule(l){2-7}
    \rowcolor{darkgreen!50}
    \cellcolor{white} & \multicolumn{6}{c}{\textbf{Roles}} \\
    \midrule
    \rowcolor{lightgreen!20}
    \cellcolor{darkgreen!50}\textbf{Tarea} & \textbf{JP} & \textbf{AN} & \textbf{DI} & \textbf{DS} & \textbf{TE} & \textbf{DOC} \\
    \midrule
    Reunión de arranque & A & I & I & I & I & R \\
    \midrule
    Reuniones periódicas & A & I & I & I & I & R\\
    \midrule
    Reunión de revisión & A & I & I & C & I & R  \\
    \midrule
    Reunión final & A & I & I & C & I & R  \\
    \bottomrule
    \end{tabular}
\end{table}

\subsubsubsection{Diseño del sistema}
En la \coloredUnderline{\hyperlink{table:matriz-diseno}{Tabla: \ref*{table:matriz-diseno} \nameref*{table:matriz-diseno}}} se presenta la matriz de responsabilidades correspondiente a la fase de diseño. 
Las tareas listadas en esta matriz son las tareas "hoja" de dicha fase, es decir, aquellas que no se descomponen en subtareas más pequeñas y, por lo tanto, representan las tareas finales 
de la fase de diseño.
\begin{table}[H]
    \centering
    \caption{Matriz RACI. Diseño del sistema}
    \label{table:matriz-diseno}
    \hypertarget{table:matriz-diseno}{}
    \begin{tabular}{
    >{\columncolor{lightgreen!20}}m{7cm} 
    >{\columncolor{white}}m{1cm} 
    >{\columncolor{white}}m{1cm} 
    >{\columncolor{white}}m{1cm} 
    >{\columncolor{white}}m{1cm} 
    >{\columncolor{white}}m{1cm} 
    >{\columncolor{white}}m{1cm}}
    \cmidrule(l){2-7}
    \rowcolor{darkgreen!50}
    \cellcolor{white} & \multicolumn{6}{c}{\textbf{Roles}} \\
    \midrule
    \rowcolor{lightgreen!20}
    \cellcolor{darkgreen!50}\textbf{Tarea} & \textbf{JP} & \textbf{AN} & \textbf{DI} & \textbf{DS} & \textbf{TE} & \textbf{DOC} \\
    \midrule
    \textit{Backend}. Diseño del módulo de usuarios & A &  & I & R &  &  I \\
    \midrule
    \textit{Backend}. Diseño del módulo de cartas & A &  & I & R &  & I \\
    \midrule
    \textit{Backend}. Diseño del módulo de sobres de cartas & A &  & I & R &  & I \\
    \midrule
    \textit{Backend}. Diseño del módulo de subastas & A &  & I & R &  & I \\
    \midrule
    \textit{Backend}. Diseño del módulo de transacciones & A &  & I & R &  & I \\
    \midrule
    \textit{Frontend}. Diseño de logo de la aplicación & A &  & R & I &  & I \\
    \midrule
    \textit{Frontend}. Diseño de la moneda de la aplicación & A &  & R & I &  & I \\
    \midrule
    \textit{Frontend}. Diseño de la temática & A &  & R & C/I &  & I \\
    \midrule
    \textit{Frontend}. Diseño del árbol de navegación & A &  & R & C/I &  & I \\
    \midrule
    \textit{Frontend}. Diseño de las páginas de información & A &  & R & C/I &  & I \\
    \midrule
    \textit{Frontend}. Diseño de las página \textit{Home} & A &  & R & C/I &  & I \\
    \midrule
    \textit{Frontend}. Diseño de las página de error & A &  & R & C/I &  & I \\
    \bottomrule
    \end{tabular}
\end{table}

\subsubsubsection{Implementación del sistema}
En la \coloredUnderline{\hyperlink{table:matriz-implementacion}{Tabla: \ref*{table:matriz-implementacion} \nameref*{table:matriz-implementacion}}} se muestra la matriz de responsabilidades para la fase de implementación.
\begin{table}[H]
    \centering
    \caption{Matriz RACI. Implementación del proyecto}
    \label{table:matriz-implementacion}
    \hypertarget{table:matriz-implementacion}{}
    \begin{tabular}{
    >{\columncolor{lightgreen!20}}m{7cm} 
    >{\columncolor{white}}m{1cm} 
    >{\columncolor{white}}m{1cm} 
    >{\columncolor{white}}m{1cm} 
    >{\columncolor{white}}m{1cm} 
    >{\columncolor{white}}m{1cm} 
    >{\columncolor{white}}m{1cm}}
    \cmidrule(l){2-7}
    \rowcolor{darkgreen!50}
    \cellcolor{white} & \multicolumn{6}{c}{\textbf{Roles}} \\
    \midrule
    \rowcolor{lightgreen!20}
    \cellcolor{darkgreen!50}\textbf{Tarea} & \textbf{JP} & \textbf{AN} & \textbf{DI} & \textbf{DS} & \textbf{TE} & \textbf{DOC} \\
    \midrule
    Reunión de arranque & A & I & I & I & I & R \\
    \midrule
    Reuniones periódicas & A & I & I & I & I & R\\
    \midrule
    Reunión de revisión & A & I & I & C & I & R  \\
    \midrule
    Reunión final & A & I & I & C & I & R  \\
    \bottomrule
    \end{tabular}
\end{table}