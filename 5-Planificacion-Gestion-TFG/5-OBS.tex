En este proyecto, se simulará una pequeña empresa ficticia para su realización. 
Se han identificado los distintos perfiles que intervienen en el proyecto, así como las tareas que cada uno debe realizar. 
Aunque el proyecto se llevará a cabo de forma individual, se ha considerado necesario estructurar estos roles para poder asignar un presupuesto adecuado a cada función.

Los roles identificados se especifican en \coloredUnderline{\hyperlink{table:obs}{Tabla: \ref*{table:obs} \nameref*{table:obs}}}. 
Cabe destacar que el rol de Jefe de Proyecto (JP) es el encargado de supervisar y coordinar el trabajo de los demás roles y este rol ha sido considerado tanto para el alumno como para el tutor del TFG.


\begin{table}[H]
\centering
\hypertarget{table:obs}{}
\caption{OBS (Organizational Breakdown Structure)}
\label{table:obs}
\begin{tabular}{>{\columncolor{lightgreen!20}}p{7cm} p{10cm}}
\toprule
\rowcolor{darkgreen!50}
\textbf{Abreviatura} & \textbf{Rol} \\
\midrule
JP & Jefe de Proyecto \\
\midrule
AN & Analista junior\\
\midrule
DI & Diseñador junior \\
\midrule
DS & Desarrollador de Software junior\\
\midrule
TE & Tester junior \\
\midrule
DOC & Documentador técnico \\
\bottomrule
\end{tabular}
\end{table}
 
Debido al tamaño del proyecto, se ha decidido implementar una matriz de asignación de roles en lugar de una matriz de responsabilidades, con el fin de evitar un aumento innecesario en la complejidad del proyecto. 
A continuación, se detallan las matrices de asignación de roles para cada fase identificada en \coloredUnderline{\hyperlink{sec:5-WBS}{\ref*{sec:5-WBS} \nameref*{sec:5-WBS}}}.

\subsubsubsection{Análisis del proyecto}
La matriz de asignación de roles para la fase de análisis del proyecto se muestra en la \coloredUnderline{\hyperlink{table:matriz-analisis}{Tabla: \ref*{table:matriz-analisis} \nameref*{table:matriz-analisis}}}.
\begin{table}[H]
    \centering
    \caption{Matriz de asignación de roles: Análisis del proyecto}
    \label{table:matriz-analisis}
    \begin{tabular}{
    >{\columncolor{lightgreen!20}}m{2cm} 
    >{\columncolor{white}}m{2cm} 
    >{\columncolor{white}}m{2cm} 
    >{\columncolor{white}}m{2cm} 
    >{\columncolor{white}}m{2cm} 
    >{\columncolor{white}}m{2cm} 
    >{\columncolor{white}}m{2cm}}
    \cmidrule(l){2-7}
    \rowcolor{darkgreen!50}
    \cellcolor{white} & \multicolumn{6}{c}{\textbf{Roles}} \\
    \midrule
    \rowcolor{lightgreen!20}
    \cellcolor{darkgreen!50}\textbf{Tarea} & \textbf{JP} & \textbf{AN} & \textbf{DI} & \textbf{DS} & \textbf{TE} & \textbf{DOC} \\
    \midrule
    Análisis del sistema & X & X &  & X &  &  \\
    \midrule
    Análisis de la arquitectura & X & X &  & X &  &  \\
    \midrule
    Análisis de la infraestructura & X & X &  & X &  &  \\
    \midrule
    Determinación del alcance de desarrollo & X & X &  &  & X &  \\
    \bottomrule
    \end{tabular}
\end{table}

\subsubsubsection{Seguimiento del proyecto}
La matriz de asignación de roles para la fase de seguimiento se muestra en la \coloredUnderline{\hyperlink{table:matriz-seguimiento}{Tabla: \ref*{table:matriz-seguimiento} \nameref*{table:matriz-seguimiento}}}.
\begin{table}[H]
    \centering
    \caption{Matriz de asignación de roles: Seguimiento del proyecto}
    \label{table:matriz-analisis}
    \begin{tabular}{
    >{\columncolor{lightgreen!20}}m{2cm} 
    >{\columncolor{white}}m{2cm} 
    >{\columncolor{white}}m{2cm} 
    >{\columncolor{white}}m{2cm} 
    >{\columncolor{white}}m{2cm} 
    >{\columncolor{white}}m{2cm} 
    >{\columncolor{white}}m{2cm}}
    \cmidrule(l){2-7}
    \rowcolor{darkgreen!50}
    \cellcolor{white} & \multicolumn{6}{c}{\textbf{Roles}} \\
    \midrule
    \rowcolor{lightgreen!20}
    \cellcolor{darkgreen!50}\textbf{Tarea} & \textbf{JP} & \textbf{AN} & \textbf{DI} & \textbf{DS} & \textbf{TE} & \textbf{DOC} \\
    \midrule
    Reunión de arranque & X & X & X & X & X & X \\
    \midrule
    Reuniones periódicas & X &  & X &  &  & X \\
    \midrule
    Análisis de la infraestructura & X & X &  & X &  &  \\
    \midrule
    Determinación del alcance de desarrollo & X & X &  &  & X &  \\
    \bottomrule
    \end{tabular}
\end{table}
