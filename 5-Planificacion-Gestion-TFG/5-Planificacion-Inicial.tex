\subsubsection{Planificación inicial} \label{sec:5-Planificacion-inicial}
\hypertarget{sec:5-Planificacion-inicial}{}
A partir de las tareas definidas en la sección \coloredUnderline{\hyperlink{sec:5-WBS}{\ref*{sec:5-WBS} \nameref*{sec:5-WBS}}} se ha realizado una planificación inicial del proyecto, en la que se han establecido las fechas de inicio y fin de cada tarea, así como su duración y las dependencias entre ellas. 
La planificación inicial se ha realizado en la herramienta \coloredUnderline{\href{hhttps://www.microsoft.com/es-es/microsoft-365/p/project-profesional-2019/cfq7ttc0k7cj}{Microsoft Project}}.
Al igual que en la sección anterior, se ha dividido la planificación en las fases en las que se divide el proyecto para mejorar la legibilidad mostrando primero una visión general y después entrando en detalle en cada una de las fases.
Se incluye una tabla con las tareas, su diuración y sus fechas de inicio y fin, así como una gráfica de Gantt con la planificación temporal de las tareas.

\subsubsubsection{Planificación inicial. Visión general}
En la \coloredUnderline{\hyperlink{table:5_PI-Vision-General}{\ref*{table:5_PI-Vision-General}: \nameref*{table:5_PI-Vision-General}}} se muestra la planificación inicial del proyecto de alto nivel, es decir, las tareas generales o fases que se deben realizar para cumplir con los objetivos del proyecto.
El total de días estimados para la realización del proyecto es de 56,5 días (462 horas), desde el 01/04/2024 hasta el 18/06/2024.


\begin{table}[H]
    \centering
    \caption{Planificación inicial. Visión general}
    \label{table:5_PI-Vision-General}
    \hypertarget{table:5_PI-Vision-General}{}
    \begin{tabular}{
       >{\columncolor{lightgreen!20}\raggedright\arraybackslash}p{1.5cm}
       >{\raggedright\arraybackslash}p{4.5cm}
       >{\raggedright\arraybackslash}p{2cm}
       >{\raggedright\arraybackslash}p{3cm}
       >{\raggedright\arraybackslash}p{3cm} }
    \rowcolor{darkgreen!50}
    \toprule
    \textbf{EDT} & \textbf{Nombre tarea} & \textbf{Duración} & \textbf{Fecha inicio} & \textbf{Fecha fin} \\
    \midrule
    1 & Proyecto BidMon Universe & 461,5 horas & 01/04/2024 & 19/06/2024 \\
    \midrule
    1.1 & Análisis del proyecto & 12 horas & 01/04/2024 & 06/04/2024 \\
    \midrule
    1.2 & Seguimiento del proyecto & 30,5 horas & 01/04/2024 & 18/06/2024 \\
    \midrule
    1.3 & Diseño del sistema & 76 horas & 06/04/2024 & 17/05/2024 \\
    \midrule
    1.4 & Implementación del sistema &  156,5 horas & 10/04/2024 & 23/05/2024 \\
    \midrule
    1.5 & Fase de pruebas & 9 horas & 24/05/2024 & 25/05/2024 \\
    \midrule
    1.6 & Despliegue del sistema & 8 horas & 13/06/2024 & 14/06/2024 \\
    \midrule
    1.7 & Documentación del proyecto & 170 horas & 06/04/2024 & 18/06/2024 \\
    \bottomrule
    \end{tabular}
\end{table}


\subsubsubsection{Planificación inicial. Análisis del proyecto}
En la \coloredUnderline{\hyperlink{table:5_PI-Analisis}{\ref*{table:5_PI-Analisis}: \nameref*{table:5_PI-Analisis}}}, se detallan la planificación de las tareas que se deben realizar en la fase de análisis del sistema.
El total de días estimados para la realización de esta fase es de 12 horas, desde el 01/04/2024 hasta el 06/04/2024.
\begin{table}[H]
    \centering
    \caption{Planificación inicial. Visión general}
    \label{table:5_PI-Vision-General}
    \hypertarget{table:5_PI-Vision-General}{}
    \begin{tabular}{
       >{\columncolor{lightgreen!20}\raggedright\arraybackslash}p{1.5cm}
       >{\raggedright\arraybackslash}p{4.5cm}
       >{\raggedright\arraybackslash}p{2cm}
       >{\raggedright\arraybackslash}p{3cm}
       >{\raggedright\arraybackslash}p{3cm} }
    \rowcolor{darkgreen!50}
    \toprule
    \textbf{EDT} & \textbf{Nombre tarea} & \textbf{Duración} & \textbf{Fecha inicio} & \textbf{Fecha fin} \\
    \midrule
    1.1 & Análisis del proyecto & 12 horas & 01/04/2024 & 06/04/2024 \\
    \midrule
    1.1.1 & Análisis del sistema & 5 horas & 01/04/2024 & 04/04/2024 \\
    \midrule
    1.1.2 & Análisis de la arquitectura & 3 horas & 04/04/2024 & 06/04/2024 \\
    \bottomrule
    \end{tabular}
\end{table}

