\subsection{Glosario de términos}
\begin{table}[htb]
    \centering
    \caption{Descripción de términos}
    \begin{tabular}{>{\columncolor{lightgreen!20}}p{7cm} p{10cm}}
    \toprule
    \rowcolor{darkgreen!50}
    \textbf{Término} & \textbf{Descripción} \\
    \midrule
    Cifrado de contraseña & Proceso de encriptación de una contraseña. \\
    \midrule
    Rol & Permisos y funcionalidades que tiene un usuario en la aplicación. \\
    \midrule
    Rareza & Valor que indica la probabilidad de obtener un activo en la aplicación. Se utiliza este valor para medir la exclusividad de un activo. Se aplica tanto a cartas, sobres como Pokemon. \\
    \bottomrule
    \end{tabular}
\end{table}

\subsection{Glosario de acrónimos}
\begin{table}[htb]
    \centering
    \caption{Descripción de acrónimos}
    \begin{tabular}{>{\columncolor{lightgreen!20}}p{7cm} p{10cm}}
    \toprule
    \rowcolor{darkgreen!50}
    \textbf{Acrónimo} & \textbf{Descripción} \\
    \midrule
    UID & Identificador único. \\
    \midrule
    API & Interfaz de programación de aplicaciones. \\
    \midrule
    RBS & \textit{Risk Breakdown Structure}, Estructura de desglose de riesgos. \\
    \bottomrule
    \end{tabular}
\end{table}

\subsection{Glosario de abreviaturas}
\begin{table}[htb]
    \centering
    \caption{Descripción de abreviaturas}
    \begin{tabular}{>{\columncolor{lightgreen!20}}p{7cm} p{10cm}}
    \toprule
    \rowcolor{darkgreen!50}
    \textbf{Abreviatura} & \textbf{Descripción} \\
    \midrule
    DD & Día. \\
    \midrule
    MM & Mes. \\
    \midrule
    AAAA & Año. \\
    \bottomrule
    \end{tabular}
\end{table}




\subsection{Glosario de variables}\hypertarget{anexo:glosario_variables}{}
En esta sección se detallan los parámetros de configuración de usuario que se deben cumplir para poder utilizar la aplicación. Estos parámetros se encuentran en la tabla \ref{table:variables_requisitos}.
\begin{table}[htb]
    \centering
    \caption{Parámetros de configuración}
    \label{table:variables_requisitos}
    \begin{tabular}{>{\columncolor{lightgreen!20}}p{7cm} p{10cm}}
    \toprule
    \rowcolor{darkgreen!50}
    \textbf{Parámetro} & \textbf{Descripción} \\
    \midrule
    \hypertarget{confParam:gu-nombreUsuario}{}
    \textbf{GU\_NOMBRE\_USUARIO} & El nombre de usuario debe tener una longitud mínima de 6 caracteres y máxima de 20. Además, solo puede contener letras y números.  \\
    \midrule
    \hypertarget{confParam:gu-contrasena}{}
    \textbf{GU\_CONTRASEÑA} & La contraseña debe tener una longitud mínima de 8 caracteres y máxima de 20. Además, debe contener al menos una letra mayúscula, una letra minúscula, un número y un carácter especial. \\
    \midrule
    \hypertarget{confParam:gu-minEdad}{}
    \textbf{GU\_MIN\_EDAD} & La edad mínima para usar el sistema es 18 años. \\
    \midrule
    \hypertarget{confParam:gu-fechaNacimiento}{}
    \textbf{GU\_FECHA\_NACIMIENTO} & El formato  de la fecha es DD/MM/AAAA. \\
    \midrule
    \hypertarget{confParam:gu-rolDefecto}{}
    \textbf{GU\_ROL\_DEFECTO} & El rol por defecto será \textit{"STANDARD"}. \\
    \midrule
    \hypertarget{confParam:gu-imgPerfilDefecto}{}
    \textbf{GU\_IMG\_PERFIL\_DEFECTO} & La imagen de perfil por defecto será la de un avatar.  \\
    \midrule
    \hypertarget{confParam:gu-saldoDefecto}{}
    \textbf{GU\_SALDO\_DEFECTO} & El saldo por defecto será de 0 monedas. \\
    \midrule
    \hypertarget{confParam:gu-tokenSesion}{}
    \textbf{GU\_TOKEN\_SESION} & El token de sesión tendrá una duración de 1 hora. \\
    \midrule
    \hypertarget{confParam:cc-formatoIDCCarta}{}
    \textbf{CC\_FORMATO\_IDC\_CARTA} & El formato del identificador de colección de carta será CC-XXXX, donde XXXX es un número entero de cuatro dígitos. \\
    \bottomrule
    \end{tabular}
\end{table}