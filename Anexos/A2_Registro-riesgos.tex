En esta sección se presentan los resultados obtenidos tras el análisis de los riesgos identificados en el proyecto (ver la sección \coloredUnderline{\hyperlink{Risks:identificacion_riesgos}{\ref*{Risks:identificacion_riesgos} \nameref*{Risks:identificacion_riesgos}}}).

Los riesgos están ordenados en función de el valor obtenido al realizar el análisis según la probabilidad e impacto de cada uno de ellos.
Para cada riesgo se detalla su descripción, categoría, probabilidad e impacto, así como la respuesta y estrategia a seguir para tratar con él.

Se puede consultrar la descripción del proceso de análisis de riesgos en el \coloredUnderline{\hyperlink{anexo:plan_de_gestion_de_riesgos}{Plan de Gestión de Riesgos}}.
\begin{table}[H]
    \centering
    \caption{Riesgo 1. Falta de comunicación con el tutor del TFG}
    \label{table:risk_comunicacion-tutor}
    \begin{tabular}{>{\columncolor{rowcolor}}l l l}
    \toprule
    \rowcolor{lightgreen}
    \textbf{Identificador} & \multicolumn{2}{l}{1} \\
    \midrule
    \textbf{Nombre} & \multicolumn{2}{l}{Falta de comunicación con el tutor del TFG} \\
    \midrule
    \textbf{Descripción} & \multicolumn{2}{p{10cm}}{En proyectos académicos como un TFG, una comunicación inadecuada con el tutor puede llevar a desviaciones del objetivo del proyecto, retrasos y resultados que no cumplen con las expectativas académicas.} \\
    \midrule
    \textbf{Categoría} & \multicolumn{2}{l}{Riesgo de gestión del proyecto} \\
    \midrule
    \textbf{Probabilidad} & \multicolumn{2}{l}{Media} \\
    \midrule
    \textbf{Impacto} & Presupuesto & Bajo \\
    \cmidrule(lr){2-3}
    & Planificación & Alto \\
    \cmidrule(lr){2-3}
    & Alcance & Crítico \\
    \cmidrule(lr){2-3}
    & Calidad & Alto \\
    \cmidrule(lr){2-3}
    & Total & 0.45 \\
    \midrule
    \textbf{Respuesta} & \multicolumn{2}{p{10cm}}{Establecer un calendario de reuniones periódicas con el tutor para revisar el progreso del proyecto y asegurar que se cumplen los objetivos establecidos. Utilizar herramientas de comunicación como correo electrónico y Microsoft Teams con el fin de mantener una comunicación constante y efectiva. Documentar todas las decisiones y avances en un informe de progreso compartido con el tutor.} \\
    \midrule
    \textbf{Estrategia} & \multicolumn{2}{l}{Mitigar el riesgo} \\
    \bottomrule
    \end{tabular}
\end{table}

\begin{table}[H]
    \centering
    \caption{Riesgo 1. Seguridad de la información}
    \label{table:risk_seguridad-informacion}
    \begin{tabular}{>{\columncolor{rowcolor}}l l l}
    \toprule
    \rowcolor{lightgreen}
    \textbf{Identificador} & \multicolumn{2}{l}{2} \\
    \midrule
    \textbf{Nombre} & \multicolumn{2}{l}{Seguridad de la información} \\
    \midrule
    \textbf{Descripción} & \multicolumn{2}{p{10cm}}{La seguridad de la información se refiere a la protección de datos contra accesos no autorizados, alteraciones, robos o eliminaciones, tanto accidentalmente como de manera intencional. La falta de seguridad en el sistema puede resultar en la pérdida de datos, daños a la reputación de la empresa y sanciones legales. } \\
    \midrule
    \textbf{Categoría} & \multicolumn{2}{l}{Riesgo técnico} \\
    \midrule
    \textbf{Probabilidad} & \multicolumn{2}{l}{Alta} \\
    \midrule
    \textbf{Impacto} & Presupuesto & Medio \\
    \cmidrule(lr){2-3}
    & Planificación & Bajo \\
    \cmidrule(lr){2-3}
    & Alcance & Bajo \\
    \cmidrule(lr){2-3}
    & Calidad & Alto \\
    \cmidrule(lr){2-3}
    & Total & 0.39 \\
    \midrule
    \textbf{Respuesta} & \multicolumn{2}{p{10cm}}{Durante la fase de puesta en producción, se contratarán servicios de seguridad gestionados a un proveedor externo especializado. Este proveedor será responsable de implementar y mantener las medidas de seguridad necesarias para proteger los datos y asegurar el cumplimiento con las normativas vigentes.} \\
    \midrule
    \textbf{Estrategia} & \multicolumn{2}{l}{Transferir el riesgo} \\
    \bottomrule
    \end{tabular}
\end{table}


\begin{table}[H]
    \centering
    \caption{Riesgo 3. Intento de fraude por parte de los usuarios finales}
    \label{table:risk_fraude}
    \begin{tabular}{>{\columncolor{rowcolor}}l l l}
    \toprule
    \rowcolor{lightgreen}
    \textbf{Identificador} & \multicolumn{2}{l}{3} \\
    \midrule
    \textbf{Nombre} & \multicolumn{2}{l}{Intento de fraude por parte de los usuarios finales} \\
    \midrule
    \textbf{Descripción} & \multicolumn{2}{p{10cm}}{Los intentos de fraude, como la creación de cuentas falsas o transacciones fraudulentas, pueden tener un impacto financiero y de reputación significativo.} \\
    \midrule
    \textbf{Categoría} & \multicolumn{2}{l}{Riesgo técnico} \\
    \midrule
    \textbf{Probabilidad} & \multicolumn{2}{l}{Alta} \\
    \midrule
    \textbf{Impacto} & Presupuesto & Medio \\
    \cmidrule(lr){2-3}
    & Planificación & Bajo \\
    \cmidrule(lr){2-3}
    & Alcance & Bajo \\
    \cmidrule(lr){2-3}
    & Calidad & Alto \\
    \cmidrule(lr){2-3}
    & Total & 0.39 \\
    \midrule
    \textbf{Respuesta} & \multicolumn{2}{p{10cm}}{Implementar sistemas de verificación de la identidad y medidas para reducir la posibilidad de que un usuario realice transacciones fraudulentas.} \\
    \midrule
    \textbf{Estrategia} & \multicolumn{2}{l}{Mitigar el riesgo} \\
    \bottomrule
    \end{tabular}
\end{table}


\begin{table}[H]
    \centering
    \caption{Riesgo 4. Errores en las estimaciones de tareas}
    \label{table:risk_estimaciones}
    \begin{tabular}{>{\columncolor{rowcolor}}l l l}
    \toprule
    \rowcolor{lightgreen}
    \textbf{Identificador} & \multicolumn{2}{l}{4} \\
    \midrule
    \textbf{Nombre} & \multicolumn{2}{l}{} \\
    \midrule
    \textbf{Descripción} & \multicolumn{2}{p{10cm}}{Las estimaciones incorrectas de la duración de las tareas pueden resultar en retrasos en el proyecto y en la necesidad de reajustar el cronograma.} \\
    \midrule
    \textbf{Categoría} & \multicolumn{2}{l}{Riesgo de gestión del proyecto} \\
    \midrule
    \textbf{Probabilidad} & \multicolumn{2}{l}{Media} \\
    \midrule
    \textbf{Impacto} & Presupuesto & Medio \\
    \cmidrule(lr){2-3}
    & Planificación & Alto \\
    \cmidrule(lr){2-3}
    & Alcance & Alto \\
    \cmidrule(lr){2-3}
    & Calidad & Medio \\
    \cmidrule(lr){2-3}
    & Total & 0.28 \\
    \midrule
    \textbf{Respuesta} & \multicolumn{2}{p{10cm}}{Consultar con el tutor las estimaciones realizadas y realizar varias iteraciones para afinarlas. Realizar revisiones periódicas del cronograma del proyecto para identificar desviaciones tempranas y ajustar las estimaciones según sea necesario.} \\
    \midrule
    \textbf{Estrategia} & \multicolumn{2}{l}{Mitigar el riesgo} \\
    \bottomrule
    \end{tabular}
\end{table}

\begin{table}[H]
    \centering
    \caption{Riesgo 5. Conciliación entre responsabilidades académicas y laborales}
    \label{table:risk_conciliacion}
    \begin{tabular}{>{\columncolor{rowcolor}}l l l}
    \toprule
    \rowcolor{lightgreen}
    \textbf{Identificador} & \multicolumn{2}{l}{5} \\
    \midrule
    \textbf{Nombre} & \multicolumn{2}{l}{Conciliación entre responsabilidades académicas y laborales} \\
    \midrule
    \textbf{Descripción} & \multicolumn{2}{p{10cm}}{La conciliación entre las responsabilidades académicas y laborales puede reducir significativamente la disponibilidad de tiempo para dedicar al proyecto. Esto, a su vez, puede disminuir la productividad y eventualmente provocar retrasos en la entrega del proyecto.} \\
    \midrule
    \textbf{Categoría} & \multicolumn{2}{l}{Riesgo de gestión del proyecto} \\
    \midrule
    \textbf{Probabilidad} & \multicolumn{2}{l}{Alta} \\
    \midrule
    \textbf{Impacto} & Presupuesto & Bajo \\
    \cmidrule(lr){2-3}
    & Planificación & Alto \\
    \cmidrule(lr){2-3}
    & Alcance & Medio \\
    \cmidrule(lr){2-3}
    & Calidad & Medio \\
    \cmidrule(lr){2-3}
    & Total & 0.33 \\
    \midrule
    \textbf{Respuesta} & \multicolumn{2}{p{10cm}}{Establecer un horario de trabajo fijo para dedicar tiempo al proyecto. Priorizar las tareas del proyecto y planificar con anticipación para evitar retrasos, apoyándose además de en la planificación temporal en tableros de trabajo como \href{https://trello.com/es}{Trello}.} \\
    \midrule
    \textbf{Estrategia} & \multicolumn{2}{l}{Mitigar el riesgo} \\
    \bottomrule
    \end{tabular}
\end{table}