
\section{CONCLUSIONES}
Este proyecto ha sido una experiencia muy enriquecedora, ya que se han aplicado muchos de los conocimientos adquiridos durante la carrera en un proyecto 
que ha supuesto un reto tanto a nivel técnico como organizativo.

En cuanto a la parte técnica, se ha profundizado en el desarrollo de aplicaciones web con React, lo que ha permitido adquirir nuevos conocimientos
sobre esta tecnología y mejorar las habilidades en el desarrollo de interfaces de usuario. Además, se ha trabajado con Sockets para la comunicación en tiempo real
entre el cliente y el servidor, lo que ha supuesto un reto añadido y ha permitido aprender cómo funcionan las comunicaciones en tiempo real.

Por otro lado, a nivel organizativo, también ha supuesto un reto llevar a cabo un proyecto de estas características de forma individual.
Se ha tenido que planificar el proyecto, establecer unos objetivos y cumplir con los plazos establecidos. Además, se ha tenido que gestionar el tiempo
de forma eficiente para poder cumplir con las tareas planificadas.


Es complicado destacar una única parte del proyecto, ya que todas las fases han sido importantes y han aportado un valor añadido al proyecto.
No obstante, si tuviera que destacar una parte, sería la de diseño de la interfaz de usuario, ya que ha sido una de las partes más creativas y en la que
se ha podido plasmar la idea inicial del proyecto y con la que me siento más satisfecha.


\newpage
\section{AMPLIACIONES} 
El proyecto se podría ampliar de diversas formas para mejorar la experiencia del usuario y añadir nuevas funcionalidades.
A continuación, se detallan algunas de las ampliaciones que se podrían llevar a cabo:

\begin{itemize}
    \item \textbf{Sistema de mensajería}: Se podría implementar un sistema de mensajería entre los usuarios para que puedan comunicarse entre ellos.
    \item \textbf{Sistema de amigos}: Se podría implementar un sistema de amigos para que los usuarios puedan añadirse como amigos y ver su colección de cartas.
    \item \textbf{Sistema de torneos}: Se podría implementar un sistema de torneos en el que los usuarios puedan competir entre ellos con las cartas de su colección.
    \item \textbf{Sistema de logros}: Se podría implementar un sistema de logros para incentivar a los usuarios a completar determinadas acciones.
    \item \textbf{Mejoras de rendimiento}: Se podría implementar paginación en la obtención de resultados de la base de datos para mejorar el rendimiento de la aplicación.
    \item \textbf{Internacionalización}: Se podría implementar un sistema de internacionalización para que la aplicación esté disponible en varios idiomas.
\end{itemize}
