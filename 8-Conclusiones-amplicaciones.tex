\newpage
\section{AMPLIACIONES} 
El proyecto se podría ampliar de diversas formas para mejorar la experiencia del usuario y añadir nuevas funcionalidades.
A continuación, se detallan algunas de las ampliaciones que se podrían llevar a cabo:

\begin{itemize}
    \item \textbf{Sistema de torneos}: Se podría implementar un sistema de torneos en el que los usuarios puedan competir entre ellos con las cartas de su colección.
    Los usuarios podrían usar las cartas de su colección, que ya tienen un nivel de ataque y defensa, para competir en torneos y ganar premios.
    Las cartas podrían subir de nivel con el uso, lo que incrementaría su valor.
    \item \textbf{Mejoras de rendimiento}: Se podría implementar paginación en la obtención de resultados de la base de datos para mejorar el rendimiento de la aplicación.
    \item \textbf{Internacionalización}: Se podría implementar un sistema de internacionalización para que la aplicación esté disponible en varios idiomas.
    La aplicación ya dispone de un botón de internacionalización adaptable a diferentes dispositivos y tamaños de pantalla,
    por lo que sería relativamente sencillo implementar esta funcionalidad.
    \item \textbf{Administración del sistema}: Se podría ampliar la aplicación para que los administradores puedan gestionar los usuarios, la tienda y las cartas.
    Los administradores podrían añadir nuevos sobres a la tienda, gestionar los usuarios, descatalogar cartas o añadir nuevas cartas a la base de datos.
    Estas funcionalidades serían relativamente sencillas de implementar, ya que la aplicación se ha diseñado contemplando la posibilidad de añadir nuevas funcionalidades.
    Por ejemplo, descatalogar una carta sería tan sencillo como cambiar un campo en la base de datos.
    \item \textbf{Cierre de subastas automático}: Se podría implementar un \textit{cron job} que cierre las subastas automáticamente cuando llegue la fecha de cierre
    o un \textit{trigger} en la base de datos que cierre la subasta cuando se alcance el tiempo límite.
    Esta funcionalidad permitiría automatizar el cierre de las subastas y evitar que los usuarios administradores tengan que estar pendientes de cerrarlas manualmente.
    El método de cierre de subastas ya está implementado en la aplicación, por lo que sería relativamente sencillo implementar esta funcionalidad.
    No se ha llevado a cabo debido a las limitaciones de presupuesto, ya que automatizar esta tarea significa tener un servidor en funcionamiento las 24 horas del día.

\end{itemize}






\section{CONCLUSIONES}
Este proyecto ha sido una experiencia muy enriquecedora, ya que se han aplicado muchos de los conocimientos adquiridos durante la carrera en un proyecto 
que ha supuesto un reto tanto a nivel técnico como organizativo.

En cuanto a la parte técnica, se ha profundizado en el desarrollo de aplicaciones web con React, lo que ha permitido adquirir nuevos conocimientos
sobre esta tecnología y mejorar las habilidades en el desarrollo de interfaces de usuario. Además, se ha trabajado con Sockets para la comunicación en tiempo real
entre el cliente y el servidor, lo que ha supuesto un reto añadido y ha permitido aprender cómo funcionan las comunicaciones en tiempo real.

Por otro lado, a nivel organizativo, también ha supuesto un reto llevar a cabo un proyecto de estas características de forma individual.
Se ha tenido que planificar el proyecto, establecer unos objetivos y cumplir con los plazos establecidos. Además, se ha tenido que gestionar el tiempo
de forma eficiente para poder cumplir con las tareas planificadas.


Es complicado destacar una única parte del proyecto, ya que todas las fases han sido importantes y han aportado un valor añadido al proyecto.
No obstante, si tuviera que destacar una parte, sería la de diseño de la interfaz de usuario, ya que ha sido una de las partes más creativas y en la que
se ha podido plasmar la idea inicial del proyecto y con la que me siento más satisfecha.



